\documentclass{openjournal}

\let\tablehead\relax

%%%%%%%%%%%%%%%%%
\usepackage{amsmath}
\usepackage{amssymb}
\usepackage{gensymb}
\usepackage{makeidx}
\usepackage{color}
\usepackage[colorlinks]{hyperref}
%%
%USER Definitions
\def\ave#1{\left< #1 \right>}
\def\gs{\mathrel{\raise1.16pt\hbox{$>$}\kern-7.0pt %                            
\lower3.06pt\hbox{{$\scriptstyle \sim$}}}}         %                            
\def\ls{\mathrel{\raise1.16pt\hbox{$<$}\kern-7.0pt %                            
\lower3.06pt\hbox{{$\scriptstyle \sim$}}}}         %   
%%%%%%%%%%%%
\def\NEW#1{{\color{blue}#1}}
\def\OLD#1{{\color{green}#1}}
\def\COMM#1{{\color{magenta}#1}}
\def\rund#1{\left( #1 \right)}
\def\eck#1{\left\lbrack #1 \right\rbrack}
\def\wave#1{\left\lbrace #1 \right\rbrace}
\def\ave#1{\left\langle #1 \right\rangle}
\def\abs#1{\left\vert #1 \right\vert}
\def\vp{\varphi}
\newcommand{\bea}{\begin{eqnarray}}
\newcommand{\eea}{\end{eqnarray}}
\newcommand{\be}{\begin{equation}}
\newcommand{\ee}{\end{equation}}
\newcommand{\bc}{\begin{center}}
\newcommand{\ec}{\end{center}}
\newcommand{\bfig}{\begin{figure}}
\newcommand{\bfigs}{\begin{figure}\sidecaption}
\newcommand{\efig}{\end{figure}}
\newcommand{\bi}{\begin{itemize}}
\newcommand{\ei}{\end{itemize}}
\newcommand{\ben}{\begin{enumerate}}
\newcommand{\een}{\end{enumerate}}
\newcommand{\bd}{\begin{description}}
\newcommand{\ed}{\end{description}}
\def\d{{\rm d}}
\renewcommand*\vec[1]{\ensuremath{\boldsymbol{#1}}}
\newcommand*\tens[1]{\ensuremath{\mathsf{#1}}}
%%%%%%%%%%%%
\newcommand*{\glossaryname}{Dictionary}

\usepackage[nonumberlist,nopostdot]{glossaries}

\newglossary[rvw2]{review}{rvws2}{rvwo2}{Review}
\newglossary[dct2]{dictionary}{dcts2}{dcto2}{Dictionary}

\newcommand{\dictentry}[3]{%
  \newglossaryentry{#2}{type={#1},name=#2,description={#3}}%
  \glslink{#2}{}%
}

\makeglossaries
\makeindex
%%%%%%%%%%%%%%%%%
\begin{document}

\title{Euclid Weak Lensing Encyclopaedia}

\author{P. Paykari$^{1}$; T. D. Kitching$^{1}$; 
M. Brown$^2$; 
F. Courbin$^3$; 
M. Cropper$^{1}$; 
A. Kiessling$^{4,5}$; 
M. Kilbinger${6,7}$; 
D. Kirk$^8$; 
B. Joachimi$^8$; 
A. Leonard$^8$; 
R. Nakajima$^9$; 
J. Rhodes$^{4,5}$; 
P. Schneider$^9$; 
A. Taylor$^{10}$; 
L. Whittaker$^2$}
\email{p.paykari@ucl.ac.uk}
\email{t.kitching@ucl.ac.uk}
\affil{$^{1}$University College London, Mullard Space Science Laboratory, Holmbury Hill Rd, Dorking RH5 6NT, UK}
\affil{$^{2}$Jodrell Bank Centre for Astrophysics, School of Physics and Astronomy, University of Manchester, Oxford Road, Manchester M13 9PL}
\affil{$^3$Laboratoire d'astrophysique, Ecole Polytechnique Federale de Lausanne, Observatoire de Sauverny, CH-1290 Versoix, Switzerland}
\affil{$^4$California Institute of Technology, MC 350-17, 1200 East California Boulevard, Pasadena, CA 91125, USA}
\affil{$^5$Jet Propulsion Laboratory, California Institute of Technology, Pasadena, CA 91109, USA}
\affil{$^6$Laboratoire AIM, UMR CEA-CNRS-Paris 7, Irfu, Service d'Astrophysique, CEA Saclay, F-91191 Gif-sur-Yvette, France}
\affil{$^7$Institut d'Astrophysique de Paris, CNRS UMR 7095 \& UPMC, 98 bis, boulevard Arago, 75014 Paris, France}
\affil{$^8$Department of Physics \& Astronomy, University College London, Gower Street, London, WC1E 6BT, UK}
\affil{$^9$Argelander-Institut fur Astronomie, Auf dem Hugel 71, 53121 Bonn, Germany}
\affil{$^{10}$Institute for Astronomy, University of Edinburgh, Royal Observatory, Blackford Hill, Edinburgh EH9 3HJ, UK}

\begin{abstract}
  Weak lensing is an effect where the images of distant galaxies are
  distorted by the gravitational potential caused by matter
  perturbations along the line of sight. It is a method that majority
  of wide-field imaging surveys are, or will, use to determine
  cosmological parameters through the dependency of this effect on the
  geometry of the Universe and the growth of structure. It is also an
  effect that can be observed around galaxy clusters and individual
  galaxies, and used to make maps of the gravitational potential and
  inferred matter density. Weak lensing studies, as a sub-field of
  cosmology, contain significant specialised terms and definitions
  that are used commonly throughout the weak lensing literature. In
  this document we provide a reference for some of those specialised terms in
  an easily referenceable manner, as well as commonly used terms that
  have a special meaning in weak lensing studies. This document is a live 
  document that will be updated in response to community input that can be 
  submitted here \url{http://github.com/Weak-Lensing/Encyclopedia}. 
\end{abstract}

%----------------------------------------------------------------------------------------
% Purpose
%----------------------------------------------------------------------------------------
\newpage
\section{Purpose}
The purpose of this document is to create a glossary of terms for the
study of weak lensing. In particular, this document can provide a basis for common
definitions of previously undefined or ambiguously defined terms in
the literature, to elucidate terms that are in common parlance in the
weak lensing community but have not yet been defined formally, and to
highlight special use of common phrases (i.e. jargon). Throughout we
have attempted to provide as concise a definition as possible, and
reference applicable papers where the reader can find more detailed
definitions.

To reflect the continuing development of the field of weak lensing,
and to allow for alternative and additional definitions to be
submitted from the community, this document is a live document, with a
repository here: \url{http://github.com/Weak-Lensing/Encyclopedia}. 
Any suggested changes, clarifications, new or amended definitions 
can be raised as issues on this site, and a new version will be submitted 
periodically.

This document covers only those terms in common usage in the weak
lensing literature covering aspects in that field of gravitational
lensing theory, cosmology, statistical inference and survey-specific
definitions. In this dictionary we will not redefine words that are
already defined in commonly dictionaries, except in the cases that the
definition is changed or supplemented in its use in the field of weak
lensing. We will use the following documents as reference documents
for definitions: Oxford Dictionary of Astronomy (Ridpath, 2012),
Oxford Companion to Cosmology (Liddle \& Loveday, 2014), Oxford
Dictionary of Physics (2015).
%
%----------------------------------------------------------------------------------------
% Format
%----------------------------------------------------------------------------------------
\section{Format}
This document is not a review article of the field. To serve the purposes described above, it
is organised in two parts. 
Part I is in a dictionary-style, where a list of terms with short/concise definitions are provided. 
Where appropriate, a list of references to relevant papers/review articles that are useful in clarifying the
definitions are also provided. This list of definitions is provided in an alphabetical
order so that phrases are easy to lookup and reference.
Part II contains fewer entries, but each have longer and more in-depth 
definitions of the terms which may also have an entry in Part I. Part II has 
a {\it review} style, where general understating of the field of weak lensing  
can be found. 
%

%----------------------------------------------------------------------------------------
% Inputs
%----------------------------------------------------------------------------------------

\vspace{-104cm}
%%%%%%%%%%%%%%%%
\dictentry{dictionary}{Fiducial Cosmology}{
A well established standard of cosmology, established by an ensemble of previous observations/surveys, which is usually specified by a set of cosmological parameters with set values (and their confidence intervals). In the context of cosmological simulations, this may imply the input cosmology (e.g. a six-parameter $\Lambda$CDM), that was used to simulate the cosmological data.
}%

%%%%%%%%%%%%%%%%
\dictentry{dictionary}{Angular-Diameter Distance}{\ADDnew}
\newcommand\ADDnew{
The angular-diameter distance $D_{\rm A}$ to a source is defined as
the square root of the ratio of its physical surface area $\d A$ and the solid angle $\d\omega$ it subtends on the observer's sky, $D_{\rm A}=\sqrt{\d A/\d \omega}$. In general, the angular-diameter distance is affected by the tidal gravitational field along the line-of-sight, leading to (isotropic) focusing or anisotropic shearing of the light bundle. For a homogeneous, isotropic universe, the angular-diameter distance is a function of the cosmological redshift of a source
-- {\scshape{\footnotesize see} \gls{Friedman--Robertson--Walker Models}}.
}%

%%%%%%%%%%%%%%%%
\dictentry{dictionary}{Luminosity Distance}{
This is defined in terms of the intrinsic luminosity $L$ of an object at redshift $z_2$ to
the flux $S$ that it received by an observer at redshift $z_1$
\setcounter{equation}{0}
\renewcommand{\theequation}{LD.\arabic{equation}}
\be
D_{L}(z_1,z_2) = 
\sqrt{\left(\frac{L}{4\pi S} \right)}\;.
\ee
In a metric cosmology, such as those described by general relativity, the luminosity
distance can be determined by other measures of distance, such as the comoving distance $D_M(z_1,z_2)$
\be
D_L(z_1,z_2) = 
\frac{1+z_2}{(1+z_1)^2}D_M(z_1,z_2)\;,
\ee
and the angular diameter distance $D_A(z_1,z_2)$
\be
D_L(z_1,z_2) = 
\frac{(1+z_2)^2}{(1+z_1)^2}\; D_A(z_1,z_2)\;.
\ee
As an example, fitting the luminosity of the CMB to a blackbody spectrum 
has allowed for the determatio of the relation between $D_L$ and $D_A$.
}%
%%%%%%%%%%%%%%%%
\dictentry{dictionary}{Environment (-al dependence)}{
This refers to the dependency of measured galaxy, or inferred dark matter properties, on the properties that can be used to characterise a particular spatial location. For example local gravitational potential, 
matter density or morphology. This is particularly used in galaxy-galaxy lensing and galaxy clustering studies 
to characterise the dependency of galaxy properties on local dark matter properties e.g. 
\cite{2013MNRAS.431.1439G} and \cite{2016arXiv160407233B}.
}%

%%%%%%%%%%%%%%%%
\dictentry{dictionary}{Halo Occupation Distribution}{
This is a prescription of how galaxies populate dark matter halos. They typically indicate 
the number of galaxies as function of halo mass, $n(M)$. They also describe how galaxies 
above a mass/luminosity threshold can be biased with respect to the underlying 
dark matter \cite{2003ApJ...593....1B}.
}%

%%%%%%%%%%%%%%%%
\dictentry{dictionary}{HOD}{
Acronym for Halo Occupation Distribution; {\scshape{\footnotesize see} \gls{Halo Occupation Distribution}}.
}%

%%%%%%%%%%%%%%%%
\dictentry{dictionary}{Group-Scale}{
This corresponds to the typical virial mass $M$ or virial radius $R$ of galaxy groups, i.e.~$M \sim 10^{13} M_\odot$ and $R \sim 1$ Mpc $h^{-1}$. Groups are typically identified as a distinct class of object, between individual galaxies and galaxy clusters, in galaxy-galaxy lensing studies where the dark matter 
environment and galaxy formation in groups can be studies e.g. \cite{2015MNRAS.452.3529V}. Group-scale 
also can pertain to the physical distances that encapsulate the Local Group i.e. 
that refers to the Milky Way and its immediate neighbouring galaxies.
}%

%%%%%%%%%%%%%%%%
\dictentry{dictionary}{Group (of Galaxies)}{
This is a gravitationally bound structure normally with fewer than about $50$ galaxies. They are 
the smallest and the most common aggregates of galaxies in the Universe --- 
comprising at least 50\% of the galaxies in the local universe. The Milky Way galaxy is 
a member of the Local Group. Larger collection of galaxies are called 
galaxy clusters and these clusters can themselves be clustered, into superclusters of galaxies. 
Also {\scshape{\footnotesize see} \gls{Group-Scale}}.
}%

%%%%%%%%%%%%%%%%
\dictentry{dictionary}{Halo}{\Halo}
\newcommand\Halo{
This is a commonly used term in weak lensing and generally refers to any hypothetical 
component that surrounds or encapsulates a galaxy, galaxy group or galaxy cluster. 
Typically it is used in two contexts: 
\begin{enumerate}
\item A dark matter halo; which is a dark matter structure within which luminous objects such as 
a galaxy, or galaxies, are embedded. A dark matter halo typically extends beyond the edge 
of the visible galaxy/galaxies and its mass dominates the total mass of the luminous plus dark matter 
structure. Dark matter halos may have substructures wihinin them that are referred to as 
`subhalo' or `satellite' halos. Dark matter halos are one of the ingredients that influence 
galaxy formation and evolution models by providing a dark matter environment within which galaxy 
formation occurs {\scshape{\footnotesize see} \gls{Environment (-al dependence)}}.
\item A galactic halo; which is a roughly spherical component of a galaxy which extends beyond the main, visible component. It can refer to old population-II stars (or globular clusters) which have small or no 
mean rotation around the galactic centre. It can also refer to high temperature gas around the galaxy
\end{enumerate}
}%

%%%%%%%%%%%%%%%%
\dictentry{dictionary}{Halo Model}{\HaloModel}
\newcommand\HaloModel{
A Halo Model is one that describes the of distribution of dark matter around galaxies on large scales. 
The main assumption of a halo model is that all matter resides in halos i.e. distinct clouds of dark matter. 
The main ingredients of a halo model are an average halo density profile $\rho(r)$, a halo 
mass function $n(M, z)$ that describes the number of halos as a function of mass and redshifts, 
a halo bias $b_{\rm h}(M, z)$ function that relates the number of galaxies to the number of dark matter haloes 
as a function of mass and redshift, and how halos cluster together in 
the form of the two-point correlation functions $\xi_{\rm h}(r)$ or power spectra as a function of seperation; 
as well as higher-order correlations and correction. The halo model has been applied to describe the 
distribution of dark matter, e.g. \cite{2003MNRAS.341.1311S} and \cite{2012ApJ...761..152T}, and also has been 
extended to include the distribution of the baryonic component in the Universe and the interaction between 
this baryonic component and the dark matter e.g. \cite{2014JCAP...04..028F}, \cite{2011MNRAS.417.2020S} and 
\cite{2015MNRAS.454.1958M}.
}%

\dictentry{dictionary}{Density Profile}{\DP}
\newcommand\DP{
A density profile describes how the density of the distribution of matter changes as a function of radial 
seperation from the centre of mass of the matter distribution.  
There exists three commonly used models for dark matter density profiles that are used in weak lensing. 
All assume a spherically symmetric mass distribution with the follow functional forms for the density of 
dark matter as a function of radius:  
\setcounter{equation}{0}
\renewcommand{\theequation}{DP.\arabic{equation}}
\begin{enumerate}
\item Pseudo-isothermal halo \citep{1972ApJ1761G} is defined as 
\be
\rho(r) = \rho_o \left[ 1 + \left(\frac{r}{r_c}\right)^2 \right]^{-1} \;,
\ee
where $\rho_o$ is the central density and $r_c$ is the core radius. This model is a good fit to most rotation curve data, but is only an approximation. 

\item NFW (Navarro-Frenk-White) profile \citep{NFW} is defined as 
\be
\rho(r)=\frac{\rho_c \delta_c}{(1/r_s)(1+r/r_s)^2}\;,
\ee
where $r_s$ is a scale radius, $\delta_c$ is a characteristic density, and $\rho_c$ is the critical over density defined as $\rho_c=3H^2/8\pi G$. The NFW profile works for a large range of halo masses and sizes, from individual galaxies to the halos of galaxy clusters. 

\item Einasto profile (investigated in \cite{2006AJ....132.2685M}) is defined as 
\be
\rho(r)=\rho_e \exp\left[-d_n ((r/r_c)^1/n)-1) \right]\;,
\ee
where $r$ is the spatial (i.e., not projected) radius. The term $d_n$ is a function of $n$ such that $\rho_e$ is the density at the radius $r_e$ that defines a volume containing half of the total mass. High resolution computer simulations lead to this model.
\end{enumerate}
For a review on dark matter halo models see e.g. \cite{2011AdAst2011E...6T}.
}%

%%%%%%%%%%%%%%%%
\dictentry{dictionary}{High Mass (clusters)}{
This refers to galaxy clusters with a virial mass $M$ that is large compared to the mean of the 
observed distribution of cluster masses in the Universe. A typically mass range to warrant this classification 
would be $M \gtrsim 10^{15} M_{\odot}$. See for e.g. \cite{2010PASJ...62..811O}.
}%

%%%%%%%%%%%%%%%%
\dictentry{dictionary}{Bullet Cluster (-like)}{\Bullet} \newcommand\Bullet{The
  bullet cluster (1E 0657$-$588), \cite{2006ApJ...648L.109C}, consists
  of two galaxy clusters, that are at a redshift of $0.296$. It is of
  particular interest in weak lensing because the mass distribution
  inferred from weak and strong lensing measurements is offset from
  the distribution of the hot intracluster gas; and spatially
  coincident with the distribution of cluster galaxies. This offset is thought to be due to
  the collision between the clusters where the dark matter did not
  interact but the X-ray emitting gas did -- causing a shock
  `bullet'-like feature in the X-ray emitting gas distribution. This offset can
  be used place a limit on the dark matter self-interaction
  cross-section. `Bullet-like' is used to refer to any cluster, or
  cluster merging events, in which similar offsets between the
  inferred dark matter distribution and the X-ray emitting gas in the
  clusters is observed. These systems
  provide emperical evidence for the existence of weakly interacting dark matter.
}%

%%%%%%%%%%%%%%%%
\dictentry{dictionary}{Baryonic Feedback}{The transfer of energy and momentum
  from a non-dark matter-related physical process (or, more generally, physical process 
  beyond those that can 
  attributed to local gravitational interactions) to the surrounding
  matter (both dark matter and baryonic matter). The subsequent
  changes in the matter environment may lead to further changes
  in the non-dark matter physics, hence a feedback loop may be
  caused. Sources of such feedback processes include the outflows from
  supernovae, AGNs and star formation. The investigation of the impact
  of these effects on weak lensing observations is an active area of
  research.  The physics of these feedback loops at the
  relevant scales are 
  unknown and, therefore, baryonic feedback is a major source of
  systematic errors in weak lensing measurements
  e.g. \cite{2004APh....22..211W}, \cite{2011MNRAS.417.2020S} and
  \cite{2015ApJ...806..186O}. 
  }%

%%%%%%%%%%%%%%%%
\dictentry{dictionary}{Total matter}{
This refers to the sum of dark matter and baryonic matter mass or mass 
density in a galaxy, galaxy group, galaxy cluster or 
for the Universe. On cosmological scales this is used to refer to the sum of the dimensionless matter 
densities: $\Omega_{\rm  m}  =   \Omega_{\rm  dm}  + \Omega_{\rm b}$.
}%

%%%%%%%%%%%%%%%%
\dictentry{dictionary}{Large-Scale Structure (LSS)}{
Refers to the structure in the distribution of matter --- whether baryonic or dark --- on
cosmological scales of $\gtrsim10\;\rm Mpc$, and its evolution. Matter is not 
randomly distributed throughout the Universe but rather it is formed into structures, 
such as filaments, walls, and clusters. There are also regions called voids 
where distribution of matter is scarce. The evolution of the LSS is primarily driven by
gravitational collapse. The distribution of baryonic matter roughly traces the total matter distribution. The
mismatch between the baryonic matter distribution and the total matter distribution is called galaxy bias. The 
weak lensing effect caused by the general LSS is known as Cosmic Shear {\scshape{\footnotesize see} \gls{Cosmic Shear}}.
}%

%%%%%%%%%%%%%%%%
\dictentry{dictionary}{Low Mass (clusters)}{
This refers to galaxy clusters with a virial mass $M$ that is small compared to the mean of the
observed distribution of cluster masses in the Universe. A typically mass range to warrant this classification
would be $M \lesssim 10^{13} M_{\odot}$. See e.g. \cite{2015MNRAS.451.1460K}.
}%

%%%%%%%%%%%%%%%%
\dictentry{dictionary}{Linear Alignment (model)}{
{\scshape{\footnotesize see} \gls{Intrinsic Alignment}}.
}%

%%%%%%%%%%%%%%%%
\dictentry{dictionary}{Quadratic Alignment (model)}{
{\scshape{\footnotesize see} \gls{Intrinsic Alignment}}.
}%

%%%%%%%%%%%%%%%%
\dictentry{dictionary}{Linear (Matter Power Spectrum)}{\Linear}
\newcommand\Linear{
The linear power spectrum, or linear part of the matter power spectrum, refers to the part of the 
power spectrum of matter overdensity fluctuations $P_{\delta}(k,z)$ 
(as a function of scale $k$ and redshift $z$) that is governed by physical effects that can be described using 
linear equations i.e. that part that can be computed from first order perturbation 
theory of the matter overdensity field and initial conditions. 
In the standard cosmological model, the linear power spectrum can be written as
\setcounter{equation}{0}
\renewcommand{\theequation}{LMPS.\arabic{equation}}
\be
 P_{\rm lin}(k,z)=
 P_0\;D_+(z)^2\;k^{n_{s}}\;T(k)^2\;,
\ee
where $n_{\rm s}$ is the power-law index of a primordial matter overdensity power spectrum, $P_0$
is the spectrum normalisation, $T(k)$ is a transfer function that described the amplitude of power spectrum as 
a function of scale, and 
\be
D_+(z) \propto H(a(z))\; \int_0^{a(z)}\frac{{\rm d}a}{a^3\;H(a)^3}\;,
\ee 
is the linear growth factor to be scaled to $D_+=1$ at $z=0$. The linear
growth factor and transfer function depend on the particular set of
cosmological parameters, such as the matter density or the parameter
of the dark energy equation of state. Commonly, the normalisation
$P_0$ is determined by specifying the variance $\sigma_8$ of (linear)
matter density fluctuations inside a sphere of comoving radius 
$R=8\;h^{-1}\;\rm Mpc$ at a redshift of $z=0$ as additional cosmological
parameter. Hence $P_0$ is chosen to satisfy
\be
\sigma_8^2=
\frac{1}{2\pi^2}\; \int_0^\infty\;{\rm d}k\;k^2\;|W(k\;R)|^2\;P_{\rm lin}(k,0)\;,
\ee
for a given $\sigma_8$ and for the window function $W(x)=3\;x^{-3}\;(\sin{x}-x\;\cos{x})$. The linear matter power spectrum describes fluctuations in the matter overdensity on large scales, 
where structure on these scales is less affected by late-time gravitational clustering and non-linear 
gravitational feedback effects or evolution. {\scshape{\footnotesize see} \gls{Correlation Functions and Power Spectra}}.
}%
%%%%%%%%%%%%%%%%
\dictentry{dictionary}{Wings (of the PSF)}{
The `wings' of a PSF refer to the extended (and most commonly smoothly varying) component of the point spread function (PSF), i.e. away 
from the mean (center) of the intensity distribution. 
For example \cite{2005MNRAS.361..160H} for a reference to this in weak lensing where the wings 
of the Hubble Space Telescope PSF are investigated.
}%
%%%%%%%%%%%%%%%%
\dictentry{dictionary}{3D Fast}{
A software suite\footnote{\url{https://tomkitching.wordpress.com/2014/01/27/3dfast/}} used to 
\begin{enumerate}
\item generate 3D cosmic shear power spectra,
\item create Fisher matrix predictions for cosmic shear surveys,
\item search cosmological parameter likelihood using an MCMC
  Metropolis-Hastings algorithm. 
\end{enumerate}
}%

%%%%%%%%%%%%%%%%
\dictentry{dictionary}{SUNGLASS} {
Simulated UNiverses for Gravitational Lensing Analysis and Shear Surveys (SUNGLASS) is a pipeline for generating simulated universes for weak lensing and cosmic shear analysis. SUNGLASS performs tomographic cosmic shear analysis using line-of-sight integration through a suite of N-body simulations which can be used for shear analysis, power spectrum estimation and cosmological parameters 
estimation \citep{2011MNRAS.414.2235K,2011MNRAS.416.1045K}.
}%
    
%%%%%%%%%%%%%%%%%

\dictentry{dictionary}{CFHT}{
The acronym for the Canada-France-Hawaii Telescope\footnote{\url{http://www.cfht.hawaii.edu/en/}}. 
The CFH observatory hosts a 3.6 meter optical/infrared telescope, which is located on the island of Hawaii.
}%

%%%%%%%%%%%%%%%%
\dictentry{dictionary}{CFHTLenS}{
The Canada France Hawaii Lensing Survey (CFHTLenS)\footnote{\url{http://www.cfhtlens.org/}}. Using data accumulated over five years by the CFHT Legacy Survey (CFHTLS), the CFHTLenS team has analysed the images of over 10 million galaxies for weak lensing statistics. CFHTLenS is a 154 square degree multi-colour optical survey in ugriz incorporating all five years worth of data from the Wide, Deep and Pre-survey components on the CFHT Legacy Survey.
}%

%%%%%%%%%%%%%%%%
\dictentry{dictionary}{CFHTLS}{
{\scshape{\footnotesize see} \gls{CFHT}; \gls{CFHTLenS}}.
}%

%%%%%%%%%%%%%%%%
\dictentry{dictionary}{Chromatic Effects}{\ChromaticEffects}
\newcommand\ChromaticEffects{
Chromatic effects are those resulting from dispersion in which different colours have different 
convergence points in an optical system, and hence different wavelengths are focused in different positions. 
It occurs when the medium through which photons pass has a wavelength-dependent refractive index.
The refractive index of transparent materials decreases with
increasing wavelength, which is unique to each material. 
In weak lensing, this is important because; 
\begin{enumerate}
\item Any finite aperture has a diffraction effect that depends on wavelength.  
\item The material of the CCD produces PSF chromaticity; see e.g. \cite{2015ExA....39..207N}.
\item Atmosphere produces atmospheric differential chromatic refraction; see e.g. \cite{2015ApJ...807..182M}.
\item Atmospheric seeing is wavelength dependent; see e.g. \cite{2012MNRAS.421..381H} .
\end{enumerate}
For space-based telescopes, such as Euclid, only the first two elements are relevant.}%

%%%%%%%%%%%%%%%%
%%%%%%%%%%%%%%%%
\dictentry{dictionary}{Fsky}{
$f_{\rm sky}$ is the fraction of the sky observed by a survey. For example, Euclid will observe $15000$ square degrees, 
so that it has $f_{\rm sky}\simeq35\%$. In practice the fraction of usable data less than this value as there will be 
areas any observed region that need to be masked due to saturated pixels for example.).
}%

%%%%%%%%%%%%%%%%
\dictentry{dictionary}{Window Function}{
A mathematical function with zero values outside a desired range/interval. For example, a top-hat window function 
is a function that equals $1$ inside an interval and zero elsewhere. Window functions are used to zero-value a 
function in the unwanted range.
}%

%%%%%%%%%%%%%%%%
\dictentry{dictionary}{Weyl Tensor}{
The Weyle tenor is a measure of the curvature of spacetime (a pseudo-Riemannian manifold). The Weyl tensor expresses 
the tidal force that a body feels when moving along a geodesic and is used to quantify how the shape of the 
body is distorted by the tidal force --- i.e. a measure image distortions effects in weak and strong 
lensing (it is the trace-free, anti-symmetric part of the decomposition of the Riemann tensor).
}%

%%%%%%%%%%%%%%%%
\dictentry{dictionary}{Field}{\Field}
\newcommand\Field{
Field has several definitions that are dependent on context in which it is used: 
\begin{enumerate}
\item 
In physics in general, a field is a mathematical construct for analysis of remote effects, such as the effect of the 
gravitational force of an object on its surrounding objects. 
\item 
It could also refer to a spatially distributed set with certain properties. For example, a scalar field has scalar 
value to every point in a space and a vector field has a vector assigned to every point in a space. 
\item 
In astronomical imaging, a field refers to a region of the sky where data is being collected, {\scshape{\footnotesize see} \gls{FoV}}.
\item 
`The field' in cosmology is used to refer to areas of the sky that do not contain a large overdensity of galaxies, such as 
a galaxy cluster, {\scshape{\footnotesize see} \gls{Field Star/Galaxy}}.
\item 
Field is also used to refer to a subject, for example `weak lensing is a field of cosmology'.  
\end{enumerate}
The usage of this words is usually obvious depending on the context.
}%

%%%%%%%%%%%%%%%% 
\dictentry{dictionary}{Field Star/Galaxy}{
Loosely defined expression for a star/galaxy that is either in the 
field of view of star/galaxy cluster, but
it is not a member of that  
cluster as it is radially at a different (larger) distance with
respect to the cluster; or is not part of a star cluster or a
    galaxy group of cluster member, see
e.g. \cite{2004AIPC..743..129B}.}%

%%%%%%%%%%%%%%%%
\dictentry{dictionary}{Fiducial}{
The word fiducial indicates that an entity or a set of parameters is being used as a basis/standard of reference.
}%

%%%%%%%%%%%%%%%%
\dictentry{dictionary}{eth Differential Operator}{
This is a differential operator acting (sometimes written as \textit{edth}) on the surface of 
a sphere, relating quantities of different spin. It is usually denoted by the symbol $\eth$ (\textbackslash eth in latex). 
Combinations of 
$\eth$ form raising and lowering operators for spin spherical harmonics. For more information, 
please refer to \cite{2005PhRvD..72b3516C} and references therein.
}%

%%%%%%%%%%%%%%%%
\dictentry{dictionary}{Hankel Transform}{\hankel}
\newcommand\hankel{
The Hankel transform, and inverse Hankel transform, of a quantity $a(k)$ in general are defined as 
\setcounter{equation}{0}
\renewcommand{\theequation}{HT.\arabic{equation}}
\begin{eqnarray}
a(r) & = & \int\limits_0^\infty {\rm d} k \; r \; (r k)^{-q} {\rm J}_\mu(k r) \hat a(k); \nonumber\\
\hat a(k) & = & \int\limits_0^\infty {\rm d} r \; k \; (r k)^{q} {\rm J}_\mu(k r) a(r), 
\end{eqnarray}
respectively, where $J$ are Bessel functions of the first kind, and $r$ is the Fourier transform variable of $k$. 
If the bias $q=0$, the above indicate the unbiased Hankel transform. 
In weak lensing a Hankel transform relates cosmic shear power spectra $C(\ell)$ to their respective correlation functions 
\begin{equation}
\xi_{+/-}(\theta)=\int_0^\infty {\rm d}\ell\ell J_{0/4}(\ell\theta)C(\ell), 
\end{equation}
see for example \cite{2015RPPh...78h6901K}, where $\theta$ are angular coordinates on the sky and $\ell$ are angular 
wavenumbers.
}%

%%%%%%%%%%%%%%%% 
\dictentry{dictionary}{Higher-order statistics}{
Correlations of random variable $x$ that depend on $x^n$ with $n>2$. Examples are the third and 
fourth moments (skewness and kurtosis). 
Higher-order statistics are often used in cosmology to measure deviations from Gaussian distributions -- examples 
are bispectrum and trispectrum estimations, see e.g. \cite{2011MNRAS.411.2241M,2011MNRAS.416.1629M}.
}%

%%%%%%%%%%%%%%%% 
\dictentry{dictionary}{Higher-order moments}{
Moments $\langle x^n \rangle$ of a random variable $x$ with $n>2$; special case of a higher-order statistic, 
{\scshape{\footnotesize see} \gls{Higher-order statistics}}. 
Higher-order moments explore non-linearities in the data and can be used for estimation of further shape parameters.
}%

%%%%%%%%%%%%%%%% 
\dictentry{dictionary}{Bias}{\bias}
\newcommand\bias{
Bias is the deviation of the expectation value of an estimator from the quantity the estimator was designed to estimate. 
Biases occur in many places in weak lensing for example 
\begin{enumerate}
\item 
Shape measurement of galaxies may be biased, {\scshape{\footnotesize see} Multiplicative and Additive Bias of Shear}
\item 
Cross-correlations of weak lensing measurement with galaxy positions can measure 
the `galaxy bias' \citep{2012ApJ...750...37J,2012MNRAS.421.1073B,2016arXiv160707910S,2016MNRAS.459.3203C}. 
Galaxy bias parameterises the hypothesised tendancy for galaxies to not be accurate tracers of underlying 
dark matter distribution. 
\item 
Cosmological parameters can be offset, or biased, with respect to
another experiment. For a recent example of this see  
\cite{2015MNRAS.451.2877M,2015PhRvD..92b3003D,2016MNRAS.459..971K,2016arXiv160605338H}
\end{enumerate}
The usage of this word is usually obvious depending on the context.
}%

%%%%%%%%%%%%%%%% 
\dictentry{dictionary}{Bispectrum}{\bispectrum}
\newcommand\bispectrum{
The 2-nd order polyspectra $P^{(2)} (\equiv B)$, as explained in Equation \ref{eq:CFS5}. 
The bispectrum, gives more information with respect to the power spectrum and is normally used to search 
for non-linear (non-Gaussian) effects. For an application in weak
lensing see
e.g. \cite{2011MNRAS.411.2241M,2011MNRAS.416.1629M}. 
{\scshape{\footnotesize see} \gls{Correlation Functions and Power Spectra}; \gls{Three-point}; \gls{3-point Statistics}}.
}%

%%%%%%%%%%%%%%%% 
\dictentry{dictionary}{Beat Coupling}{
The coupling of the density contrast to super-survey modes and the effect that this induces 
on the power spectrum. It is also known as ``super-sample
covariance'',  {\scshape{\footnotesize see} \gls{Super-Sample Covariance}}.}%

%%%%%%%%%%%%%%%%
\dictentry{dictionary}{Auto-Correlation Function}{
A mathematical tool for measuring the similarity between observations
of the {\it same} quantity as a function of their spatial and/or
temporal separation.  
{\scshape{\footnotesize see} \gls{Correlation Functions and Power Spectra}}.
}%

%%%%%%%%%%%%%%%%
\dictentry{dictionary}{Tomography}{
The  division of some observable into spatial sections. In cosmology and weak lensing, this refers to dividing a population of 
galaxies into (possibly overlapping) redshift bin. In weak lensing `tomographic' bins are defined by the position of the source 
background galaxies, but due to the line-of-sight integrated nature of weak lensing, probe projected slices through the Universe.
}%

%%%%%%%%%%%%%%%%
\dictentry{dictionary}{Tomographic Bin}{
In tomography, the observable is divided into spatial sections in the radial (redshift) direction. 
The sections are referred to as tomographic bins.
}%

%%%%%%%%%%%%%%%%
\dictentry{dictionary}{Two-point}{
The correlation between some variable at two points in space, or with in transformed space such as Fourier space.
{\scshape{\footnotesize see} \gls{Correlation Functions and Power Spectra}}. 
}%

%%%%%%%%%%%%%%%%
\dictentry{dictionary}{Three-point}{
The correlation between some variable at three points in space, or with in transformed space such as Fourier space. 
{\scshape{\footnotesize see} \gls{Correlation Functions and Power Spectra}; \gls{Two-point}; \gls{Bispectrum}}.
}%

%%%%%%%%%%%%%%%%
\dictentry{dictionary}{Theta}{\itheta}
\newcommand\itheta{In weak lensing this is used to refer to the the great circle distance between two galaxies on the celestial sphere. 
As given in, for two galaxies $i = 1,2$ at  right ascension and declination ($\alpha_i, \delta_i$), their great 
circle distance can be calculated using, 
\setcounter{equation}{0}
\renewcommand{\theequation}{TH.\arabic{equation}}
\begin{equation}
\cos \theta = \cos(\alpha_2 - \alpha_1) \cos \delta_1 \cos \delta_2 + \sin \delta_1 \sin \delta_2\;.
\ee
See for example \cite{2015RPPh...78h6901K}.
}%

%%%%%%%%%%%%%%%%
\dictentry{dictionary}{Tension}{
When measurements of parameters from different probes or experiments disagree with each other at a statistical level 
that is considered by the person using this word to be significant. For a recent examples of this in weak lensing see 
\cite{2015MNRAS.451.2877M,2015PhRvD..92b3003D,2016MNRAS.459..971K,2016arXiv160605338H}.
}%

%%%%%%%%%%%%%%%%
\dictentry{dictionary}{Systematic Error (Effect)}{
Systematics are effects that bias or contaminate the true astrophysical or cosmological signal. Systematics can originate from 
instruments, calibration, measurements, inaccurate modelling or even 
astrophysics (e.g. intrinsic alignments). {\scshape{\footnotesize see} \gls{Accuracy}; \gls{Bias}}.
}%

%%%%%%%%%%%%%%%%
\dictentry{dictionary}{Super-Sample Covariance}{
Large-scale modes that exist outside of the survey  area are known as super-survey modes. Nonlinear evolution couples smaller-scale  modes with the super-survey modes, which produces off-diagonal terms in the power spectrum  covariance, known as the super-sample covariance. Super-sample covariance has also been known as ``beat coupling'' in the literature. For more information see for example \cite{2013PhRvD..87l3504T}. {\scshape{\footnotesize see} \gls{Beat Coupling}}.
}%

%%%%%%%%%%%%%%%%%

\dictentry{dictionary}{Angle-Averaging}{
Taking the average value of a 2D-function over a circle of equal angular 
separation in real space or equal wavenumber 
in Fourier space. Also referred to as azimuthal averaging.}%

%%%%%%%%%%%%%%%%
\dictentry{dictionary}{Accuracy}{The difference between the results of a
measurement, or a combination of a set of
measurements, and the true value. The differences are associated with
systematic effects, rather than statistical errors. The statistical
literature commonly uses the term ``bias'' instead of accuracy; meaning a highly accurate result has a low bias
{\scshape{\footnotesize see}. \gls{Bias}}. 
}%

%%%%%%%%%%%%%%%%%

\dictentry{dictionary}{3D}{
Term usually applied to cosmic shear analysis when the radial information is explicitly made use of, and usually with moderate to high resolution.
}%

%%%%%%%%%%%%%%%%
\dictentry{dictionary}{1-point Statistics}{
Statistical term referring to 1-point statistical properties of a
continuous or discrete stochastic process, such as the matter density field or
galaxy distribution, using values of the field at a single
point or wavenumber. Examples are moments of the field or the distribution 
}%

%%%%%%%%%%%%%%%%
\dictentry{dictionary}{2-point Statistics}{
Statistical term referring to 2-point statistical properties of a
continuous or discrete stochastic process, such as the matter density field or
galaxy distribution, using values of the field at two distinct points
or wavenumbers. Examples are the 2-point correlation function or the
power spectrum 
{\scshape{\footnotesize see} \gls{Correlation Functions and Power Spectra}}
}%
  
%%%%%%%%%%%%%%%%
\dictentry{dictionary}{3-point Statistics}{
Statistical term referring to 3-point statistical properties of a
continuous or discrete stochastic process, such as the matter density field or
galaxy distribution, using values of the field at a three distinct
points or wavenumbers. Examples are the 3-point correlation function
or the bispectrum 
{\scshape{\footnotesize see} \gls{Correlation Functions and Power Spectra}; \gls{Bispectrum}}
}%
  
%%%%%%%%%%%%%%%%
\dictentry{dictionary}{Margin}{
Margin is an allocated fraction of a specific type of error within the maximally allowed error budget, that allows for changes in 
specification that decrease performance to still enable scientific objectives to be achieved\footnote{Systems engineering definitions are provided here \url{http://www.ecss.nl}.}.
}%

%%%%%%%%%%%%%%%%
\dictentry{dictionary}{Line-of-Sight}{
The fiducial light path(s) from, or to, the point of an observer passing through a particular angular coordinate(s) 
on the celestial sphere $\vec\theta$.
}%

%%%%%%%%%%%%%%%%
\dictentry{dictionary}{Line-of-Sight Integration}{
Often, effects in only the transverse direction as function of
$\vec\theta$ are observed, and effects that take place along the
line-of-sight must be integrated to calculate the predicted
(gravitational lensing) effects. For example, in the Born approximation of light
propagation, the integration is performed along the imaginary path of
a unperturbed fiducial light ray in the line-of-sight direction.
}%

%%%%%%%%%%%%%%%%
\dictentry{dictionary}{Limber Approximation}{\LimberApproximation}
\newcommand\LimberApproximation{The Limber approximation \citep{1953ApJ...117..134L} is employed for estimates of an angular correlation 
function (angular domain) or power spectrum (Fourier domain) of spatial random 
fields $\delta[f_{\rm K}(\chi){\vec\theta},\chi]$ projected on the celestial sphere. 
Here $\chi$ is the distance of a comoving position $(f_{\rm K}(\chi)\vec\theta,\chi)$ from the observer, 
and $\vec\theta$ is the line-of-sight direction relative to a fiducial direction. The 
approximation assumes that the random field quickly decorrelates such that 
correlations between positions with different radial distances from the observer can effectively be 
evaluated at the same $\chi$. For example, it assumes for a two-point correlation function that 
\setcounter{equation}{0}
\renewcommand{\theequation}{LM.\arabic{equation}}
\begin{equation}
C_{\delta\delta}=
\ave{
  \delta[f_{\rm K}(\chi){\vec\theta},\chi]
  \delta[f_{\rm K}(\chi^\prime){\vec\theta^\prime},\chi^\prime]
}
\approx
\ave{
  \delta[f_{\rm K}(\chi){\vec\theta},\chi]
  \delta[f_{\rm K}(\chi){\vec\theta^\prime},\chi]
}\;.
\ee
Importantly, the Limber approximation assumes small angular separations on the sky, or large angular wave 
numbers in the case of power spectra \citep{2007A&A...473..711S}.  Thereby we arrive at the following 
integral for the angular two-point correlation function $C_{12}(\theta)$ of homogeneous and isotropic random fields $\delta$. For this let
\begin{equation}
 g_i(\vec\theta)=
 \int_0^\infty\;{\rm d}\chi\;q_i(\chi)\;\delta[f_{\rm K}(\chi)\vec\theta,\chi]\;,
\ee
be projections $i$ of $\delta$ on the sky along $\vec\theta$ with projection kernel $q_i(\chi)$. The Limber approximation then yields for the cross correlation between $g_i(\vec\theta)$ and $g_j(\vec\theta^\prime)$ the integral
\begin{equation}
C_{ij}(\theta)=
\int_0^\infty{\rm d}\chi\;q_1(\chi)\;q_2(\chi)\;
\int\;{\rm d}(\Delta\chi)\;
C_{\delta\delta}(\sqrt{f_{\rm K}^2(\chi)\theta^2+(\Delta\chi)^2},\chi)\;,
\ee
for separations $\theta=|\vec\theta-\vec\theta^\prime|$. More frequently employed in gravitational lensing is the Limber approximation of the angular power spectrum at angular wave number $\ell$, namely
\begin{equation}
P_{12}(\ell)=
\int{\rm d}{\vec\theta}\;C_{12}(\theta)\;{\rm e}^{{\rm i}\vec\ell\cdot\vec\theta}=
\int_0^\infty{\rm d}\chi\;
\frac{q_1(\chi)\;q_2(\chi)}{f_{\rm K}^2(\chi)}\;
P_\delta\left(\frac{\ell}{f_{\rm K}(\chi)},\chi\right)\;,
\ee
where $P_\delta(k,\chi)$ is the spatial power spectrum of $\delta$ at spatial wave number $k$ and distance $\chi$, 
and we use the notation of \cite{2010A&A...523A...1J}. For an 
exposition of the Limber approximation in the case of projected fields see \cite{2008PhRvD..78l3506L}.
}%





%%%%%%%%%%%%%%%%
\dictentry{dictionary}{Weight Map}{ A weight map is a map which attaches a weight
  to each pixel on the map.  See e.g. \cite{2013MNRAS.433.2545E} for
  details of a weight map to describe masking
  effects in a data analysis chain used for weak lensing. 
  }%

%%%%%%%%%%%%%%%%
\dictentry{dictionary}{Exposure}{
A period of observation on specific part of the sky using a telescope. 
One exposure in Euclid refers to observation in one dither
position. {\scshape{\footnotesize see} \gls{Dither}}.  
}%

%%%%%%%%%%%%%%%%
\dictentry{dictionary}{Band}{
The characterisation of a photometric system in terms of its sensitivity to incident radiation as a function of wavelength/frequency. 
Hence bands are typically wavelength filters used in a survey over a particular range. The sensitivity usually depends on the 
optical system, detectors and filters. {\scshape{\footnotesize see} \gls{Narrowband}; \gls{Broadband}}. 
}%

%%%%%%%%%%%%%%%%
\dictentry{dictionary}{In-band}{
Light transmission within an optical or IR band.}%

%%%%%%%%%%%%%%%%
\dictentry{dictionary}{Broadband}{
Broadband photometry refers to photometric observations and data for
which the wavelength interval $\Delta\lambda$ over which the
telescope/instrument/filter/detector system has appraciable
sensitivity is not small compared to the central (or mean) wavelength
$\lambda_0$. Typically, $\Delta\lambda/\lambda \sim 0.1$ for broadband
observations. Typical examples are the Sloan filters u, g, r, i, z, or
the Johnson-Cousins filters U, B, V, R, I, J, H, K. {\scshape{\footnotesize see} \gls{Band}; \gls{Narrowband}}.
Weak lensing literature typically refers to the u, g, r, i, z filters as `broadbands'.
}%
    
%%%%%%%%%%%%%%%% 
\dictentry{dictionary}{Euclid}{\Euclid}
\newcommand\Euclid{
Euclid\footnote{\url{http://euclid-ec.org}} \cite{2011arXiv1110.3193L} is an M2-class (medium-class) ESA mission, 
which is part of ESA's `Cosmic Vision' (2015--2025) scientific program. Euclid was chosen in 
October 2011 by ESA as of the leading probes of cosmology. Euclid will map the geometry of the dark Universe and aims to shed light on dark energy and dark matter by accurately measuring the acceleration of the universe. The mission will investigate the distance-redshift relationship and the evolution of cosmic structures by measuring shapes and redshifts of galaxies and clusters of galaxies out to redshifts $z \simeq2$ --- there are two main surveys in Euclid; weak lensing and galaxy clustering. Euclid aims to observe about 1.5 billion galaxy images. It is scheduled to be launched in 2020. 
Euclid is named after 3rd century BC Greek mathematician Euclid of Alexandria, the `Father of Geometry'.
}%
         
%%%%%%%%%%%%%%%%
\dictentry{dictionary}{Astrometry}{
The branch of astronomy involving precise measurements of the positions and movements of stars and other celestial bodies.
}%
    
%%%%%%%%%%%%%%%%
\dictentry{dictionary}{Athena}{\Athena}
\newcommand\Athena{
This can refer to: 
\begin{enumerate}
\item The Advanced Telescope for High-ENergy Astrophysics (ATHENA)\footnote{\url{http://www.the-athena-x-ray-observatory.eu}} 
is an X-ray telescope selected as the second L-class mission in ESA's Cosmic Vision 2015-25 plan, with a launch foreseen in 2028. Athena aims to unravel mysteries of two major components of the Cosmos; The Hot Universe -- revealing the properties of the hot gas and relating its evolution to large-scale structure and the cool components in galaxies and stars; The Energetic Universe -- unravel the physics of black hole growth, energy output and its evolution to the highest redshifts.
\item A 2D-tree code\footnote{\url{http://www.cosmostat.org/software/athena/}} for estimating second-order correlation functions from galaxy catalogues. It computes the shear-shear correlations (cosmic shear), position-shear correlation (galaxy-galaxy lensing) and position-position (galaxy correlation). 
\item A grid-based code\footnote{\url{https://trac.princeton.edu/Athena/}} for astrophysical magnetohydrodynamics (MHD), which was developed primarily for studies of the interstellar medium, star formation, and accretion flows. 
\end{enumerate}
The reference is usually obvious in context.
}%
        
%%%%%%%%%%%%%%%%
\dictentry{dictionary}{LSST (Large Synoptic Survey Telescope)}{\lsst}
\newcommand\lsst{
The Large Synoptic Survey Telescope (LSST)\footnote{\url{http://www.the-athena-x-ray-observatory.eu}} 
is a wide-field survey reflecting telescope with an 8.4-meter primary mirror, currently under construction, 
that will photograph the entire available sky every few nights. The telescope has a very wide 3.5-degree diameter field of view, feeding a 3.2 gigapixel CCD imaging camera. Commissioning is expected to start in 2018. 
}%

%%%%%%%%%%%%%%%%
\dictentry{dictionary}{HSC}{
The Hyper-Supreme Camera (HSC)\footnote{\url{http://subarutelescope.org/Observing/Instruments/HSC/}} 
is a digital camera for the 8.2 meter Subaru telescope, built by National Astronomical Observatory of Japan. 
The FoV of the camera is 1.5 degrees diameter. 
}%

%%%%%%%%%%%%%%%%%

\dictentry{dictionary}{Absolute Photometry}{
Absolute photometry ties the measured
count rate of an observation in a specific photometric band to an
absolute, calibrated flux scale. This is normally done by observing
the object through multiple filters and also observing a number of
photometric standard stars.}%

%%%%%%%%%%%%%%%%
\dictentry{dictionary}{Stellar Calibration Field}{
An observed field of galactic stars that will be  visited at intervals throughout the lifetime of a weak lensing survey 
to calibrate the optical system and PSF properties; {\scshape{\footnotesize see} \gls{PSF}}. 
These periodic observations enable tracking and characterisation of changes in instrument response.
}%

%%%%%%%%%%%%%%%%
\dictentry{dictionary}{ugriz}{
A photometric system with filters ranging from the UV (u $\sim$ 350nm) to the near-infrared (z  $\sim$ 900nm). 
This system was used in the Sloan Digital Sky 
Survey (SDSS)\footnote{\url{http://skyserver.sdss.org/dr1/en/proj/advanced/color/sdssfilters.asp}}.
}%
    
    
    
    
    
    
    
%%%%%%%%%%%%%%%%
\dictentry{dictionary}{Early and Late Type (Galaxies)}{\elg}
\newcommand\elg{
According to the Hubble sequence, galaxies can be divided into two main populations. These are referred to in weak lensing for several 
reasons, some of these are: their intrinsic alignment properties may be different which is important for 
cosmic shear studies, their dark matter environments may be different which can be inferred using galaxy-galaxy lensing studies. 
\begin{enumerate}
\item Early-type galaxies are elliptical galaxies which have approximately spheroidal systems. They usually 
have red colour in optical wavelengths, and are hypothesised to have pressure-supported stellar systems. 
\item Late-type galaxies are spiral galaxies which have disc systems. They usually have blue colours 
optical wavelengths and are hypothesised to have rotation-supported stellar systems. Irregular galaxies also fall into this category.  
\end{enumerate}
The origin of the terms ``early-type'' and ``late-type'' lies with the historical interpretation of the 
Hubble tuning-fork diagram, which assumed galaxies evolved \emph{from} ellipticals (early-type) galaxies \emph{to} 
spiral (late-type) galaxies. The names have no particular basis in the 
modern picture of formation or evolution of galaxies.
}%

%%%%%%%%%%%%%%%%
\dictentry{dictionary}{Late Type (Galaxies)}{
{\scshape{\footnotesize see} \gls{Early and Late Type (Galaxies)}}.
}%

%%%%%%%%%%%%%%%%
\dictentry{dictionary}{Early Type (Galaxies)}{
{\scshape{\footnotesize see} \gls{Early and Late Type (Galaxies)}}.
}%

%%%%%%%%%%%%%%%%
\dictentry{dictionary}{Light Profile}{\LightProfile}
\newcommand\LightProfile{
Galaxies observed in the sky have a finite extent. Their observed 
distribution of light is described by a surface brightness distribution $I(\vec\theta)$, i.e. a `light profile'. These have specific 
characteristics that define the morphology of galaxies on the sky. 

For weak gravitational lensing, the light profile and substructure might need to be taken into 
account when measuring shapes because the bias in estimators of galaxy shapes usually depend, 
among other things, on the light profile. A model for the radial light profile of a galaxy 
can be found by averaging over all angles 
\setcounter{equation}{0}
\renewcommand{\theequation}{LP.\arabic{equation}}
\be
I(s):=
2\pi s \int_0^{2\pi} {\rm d}\phi\;s\;I(s\;{\rm e}^{{\rm i}\phi})\;.
\ee
A commonly used functional form that has been found to be a good approximations to observed galaxy light profiles
is known as the `S\'ersic' model \cite{1963BAAA....6...41S}
\be
I_n(s):=
\exp{\left(-k\;\left[\frac{s}{s_{\rm h}}\right]^{\frac{1}{n}}\right)}
\;,
\ee
which is determined by the `S\'ersic index' $n$ and the half-light
radius $s_{\rm h}$. \cite{1989woga.conf..208C} finds $k\sim 1.992n-0.3271$. Special cases of this 
function form are the exponential profile that has $n=1$, and the `de Vaucoleur' \citep{1948AnAp...11..247D} profile which has $n=4$. 
These are used as approximations for early and late-type galaxies with the exponential
profile for late-type galaxies, and the de Vaucouleur profile for early-type galaxies.
}%

%%%%%%%%%%%%%%%%
\dictentry{dictionary}{AGN}{
Acronym for Active Galactic Nucleus;
{\scshape{\footnotesize see} \gls{Active Galactic Nucleus}} .
}%

%%%%%%%%%%%%%%%%
\dictentry{dictionary}{Active Galactic Nucleus}{\agns}
\newcommand\agns{An active galactic nucleus (AGN) is a compact region in a galaxy,
mostly situated at its centre, where a substantial amount of radiation
over a wide range of the electromagnetic spectrum is emitted. A galaxy
hosting an AGN at its nucleus is called an `active' galaxy.
Observationally, an AGN is often difficult to distinguish from a
central star burst. Physically, the radiation from an AGN is not due
to the stars or hot gas in the galaxy, but powered by accretion of
matter onto a supermassive black hole at the centre of its host
galaxy. AGNs are classified into different categories: Seyfert
galaxies, radio galaxies, quasars, QSOs, and BL Lac Objects
\citep{2013pss6.book..305P}. AGNs are
relevant for weak lensing because they affect the
distribution of matter through baryonic feedback processes
{\scshape{\footnotesize see} \gls{Baryonic Feedback}} that occur during the formation and
evolution of galaxies. Furthermore, if the luminosity of an AGN is
sufficiently large, it can outshine the radiation of its host galaxy
and may thus be misidentified as a point-like source on images,
thereby biasing the determination of the point-spread function. However,
spectral information or multi-band imaging can be used to remedy this.
 }%

%%%%%%%%%%%%%%%%
\dictentry{dictionary}{Blue Galaxies}{
{\scshape{\footnotesize see} \gls{Blue and Red Galaxies}}.
}%

%%%%%%%%%%%%%%%%
\dictentry{dictionary}{Red Galaxies}{
{\scshape{\footnotesize see} \gls{Blue and Red Galaxies}}.
}%

%%%%%%%%%%%%%%%%
\dictentry{dictionary}{Blue and Red Galaxies}{
The colour-absolute magnitude diagram of low-redshift galaxies shows a
distinctive pattern. If the colour is measured in two bands, located
on either side of the $4000\AA$ break, then a concentration of red
galaxies at high luminosities, and one of blue galaxies at lower
luminosities is seen. Most of the luminous galaxies belong to one of
these two concentrations, constituting the {\it red sequence} and {\it
  blue cloud} of galaxies; the space between these two
concentration in the colour-magnitude diagram is called {\it green
  valley}, but it is populated with only a small fraction of galaxies
\citep[see, e.g.,][]{2009ARA&A..47..159B}. There is a strong
correlation between the galaxy colours and their morphological
classification, in that most blue-cloud galaxies are spirals, whereas
the majority of red-sequence galaxies are early-type galaxies
(ellipticals and S0's), or early-type spirals. The blue and red
populations of galaxies appears to also extend to higher redshift. The
importance of the distinction between red and blue galaxies for weak
lensing comes from their different intrinsic alignment properties and
the difficulty to perform a morphological classification for galaxies
with the apparent magnitudes typical for weak lensing source galaxies:
red-sequence galaxies seem to have considerably higher intrinsic
alignment amplitudes than blue-cloud galaxies \citep[see, e.g.,][and
references therein]{2016arXiv160603216H}.
}%

%%%%%%%%%%%%%%%%%

\dictentry{dictionary}{CLASH}{
Cluster Lensing And Supernova survey with Hubble (CLASH)\footnote{\url{http://www.stsci.edu/~postman/CLASH/Home.html}} 
is a survey aiming to observe $25$ massive galaxy clusters over a $3$ year period using the Hubble Space Telescope (HST).
}%

%%%%%%%%%%%%%%%%
\dictentry{dictionary}{Colour Gradient}{\colgrad}
\newcommand\colgrad{
A colour gradient is a spatially varying spectral energy distribution 
within a galaxy profile {\scshape{\footnotesize see} \gls{Light Profile}}, manifest as 
a change in the mean colour as a function of radius of the galaxy light profile. This can cause an 
important potential systematic effect in weak lensing, because it is necessary to measure galaxy shapes with great accuracy, 
which in turn requires a detailed model of the 
point spread function (PSF). In general, the PSF varies with wavelength and therefore the PSF integrated over an 
observing filter depends on the spectrum of the object. For a typical galaxy the spectrum varies across the galaxy image, 
thus the PSF depends on the position within the image. Therefore colour gradients within galaxies can nessitate a position-dependent 
PSF correction to be applied when measuring the shapes of galaxies. This effect has been investigated in 
\cite{2012MNRAS.421.1385V} and \cite{2013MNRAS.432.2385S}.}%

%%%%%%%%%%%%%%%%
\dictentry{dictionary}{Comoving Distance}{
Comoving distance is the distance to an object where the expansion of
the Universe is factored out. This gives a distance to a comoving observer that does 
not change in time due to the expansion of space-time. 
Comoving distance is normalised so that it is a distance between two
events in space at the present cosmological time. For objects moving with the Hubble flow, it remains constant in time. 
The comoving distance from an observer to a distant object (e.g. galaxy) can be computed by 
\setcounter{equation}{0}
\renewcommand{\theequation}{CD.\arabic{equation}}
\be
\chi = \int_{t_e}^t c \; \frac{dt^\prime}{a(t^\prime)}, 
\ee 
where $a(t^\prime)$ is the scale factor, $t_e$ is the time of emission of the photons detected by the observer, 
$t$ is the present time, and $c$ is the speed of light in vacuum. 
{\scshape{\footnotesize see} \gls{Friedman--Robertson--Walker Models}}}%

%%%%%%%%%%%%%%%%
\dictentry{dictionary}{Convolution}{
A convolution (sometimes known as `folding') is an integral that expresses the amount of overlap of one 
function as it is shifted over another function. It therefore `blends' one function with another. 
Convolution of two functions $f$ and $g$ is given by 
\setcounter{equation}{0}
\renewcommand{\theequation}{CV.\arabic{equation}}
\be
[f*g](t)=\int f(\tau)\, g(t-\tau)\; \d\tau\;.
\ee
A convolution in real space becomes a multiplication in Fourier space
and vice versa. This is a particularly important mathematical concept 
in weak lensing because the finite aperture of a telescope, and the atmosphere, act to convolve an image with a Point Spread Function (PSF) 
that blurs images of galaxies. This blurring needs to be accounted for when measuring the weak lensing effect. 
{\scshape{\footnotesize see} \gls{Convolved/Convolutive}; \gls{Non-convolutive}}.
}%

%%%%%%%%%%%%%%%%
\dictentry{dictionary}{Correlation Function}{\corrfunc}
\newcommand\corrfunc{
Correlation functions describe how two variables co-vary with one
another on average; it is a measure of the excess probability for
any random sample (e.g. distribution of galaxies) within some given
separation. In cosmology the random variables are usually position
dependent fields. 
\\
\\
Correlation fucntion is a general term for any two variables, however
there is some nomenclature referring to auto-correlation and cross-correlations that is
clarified in {\scshape{\footnotesize see} \gls{Correlation Functions and Power Spectra}; \gls{Auto-Correlation Function}\gls{Cross-Correlation Function}}.
\\
\\
Correlation is often used as a synonym for covariance; {\scshape{\footnotesize see} \gls{Covariance}}.
 }%

%%%%%%%%%%%%%%%%
\dictentry{dictionary}{COSEBI}{
Acronym for Complete Orthogonal Sets of E-/B-mode integrals that were 
introduced and investigated in \cite{COSEBI}, \cite{2012A&A...542A.122A} and \cite{2016arXiv160100115A}. 
For a measurement of the shear correlation functions
  $\xi_\pm(\theta)$ over the range $\theta_{\rm min}\le
  \theta\le\theta_{\rm max}$, one defines 
\setcounter{equation}{0}
\renewcommand{\theequation}{COS.\arabic{equation}}
%
\be
E_n={1\over 2}\int_{\theta_{\rm min}}^{\theta_{\rm
    max}}\d\theta\;\theta\, \eck{T_{+n}(\theta)\xi_+(\theta)
+T_{-n}(\theta)\xi_-(\theta)}\; ;\quad
B_n={1\over 2}\int_{\theta_{\rm min}}^{\theta_{\rm
    max}}\d\theta\;\theta\, \eck{T_{+n}(\theta)\xi_+(\theta)
-T_{-n}(\theta)\xi_-(\theta)}\; .
\label{eq:COS1}
\ee
%
The functions $T_{-n}(\theta)$ are related to the $T_{+n}(\theta)$
through
%
\be
T_{-n}(\vartheta)=T_{+n}(\vartheta)+\int_{\theta_{\rm min}}^\vartheta
\d\theta\;\theta\, T_{+n}(\theta)\rund{{4\over \vartheta^2}
-{12\theta^2\over\vartheta^4}} \;.
\label{eq:COS2}
\ee
Provided that the $T_{+n}(\theta)$ satisfy the conditions
\be
\int_{\theta_{\rm min}}^{\theta_{\rm max}}\d\theta\;\theta\,
T_{+n}(\theta) = 0 =
\int_{\theta_{\rm min}}^{\theta_{\rm max}}\d\theta\;\theta^3\,
T_{+n}(\theta) \;,
\label{eq:COS3}
\ee
$E_n$ is sensitive only to E-modes, whereas $B_n$ is sensitive only to
B-modes of the shear. One can now construct a complete set of
functions $T_{+n}$ on the interval $\theta_{\rm min}\le
  \theta\le\theta_{\rm max}$ which satisfy the conditions
  (\ref{eq:COS3}); this then yields a set of COSEBIs $E_n$, $B_n$
  which contain all E-/B-mode information of the correlation functions
  on that interval \citep[see][]{COSEBI}.
}%

%%%%%%%%%%%%%%%%
\dictentry{dictionary}{Cosmic Shear}{Cosmic shear is the gravitational lensing
  effect caused by the large-scale structure along and around
  the line of sight to distant objects. This large-scale structure can
  induce both a shear and a magnification in the images of
    distant sources.}%

%%%%%%%%%%%%%%%%
\dictentry{dictionary}{Cosmology-sensitive}{
A result that is dependent on the cosmological model, or values of
cosmological parameters, that have been assumed in deriving the
results.  
If results are solely data-driven without any assumptions made about the Universe we live in, the results are cosmology-insensitive.
}%

%%%%%%%%%%%%%%%%
\dictentry{dictionary}{Covariance}{\covariance}
\newcommand\covariance{
Covariance is a statistical measure that indicates the extent to which two or more variables co-vary with one another on average. 
\\
\\
In the community, {\it correlation} is often used as a synonym for {\it covariance}. 
{\scshape{\footnotesize see} \gls{Correlation Function}}.}%

%%%%%%%%%%%%%%%%
\dictentry{dictionary}{CTE}{\cte} \newcommand\cte{ Charge Transfer Efficiency
  (CTE) measures the fraction of the charge in a given pixel in a CCD
  that is transferred to the next row during the readout. The CTE is
  always less than unity because some of the charge is caught in
  `traps' in the pixel and is left behind during transfer of that
  pixel to the readout amplifier. The fraction of charge lost depends
  on the signal levels of both the source and the background. Since in
  a CCD, charge must be transferred many times before reaching the
  readout register, even a small decrease in CTE can have a large
  effect on the measured count rate near the centre of the
  detector. For $n$ transfers, the fraction of detected charge will be
  $\textrm{CTE}^n$.  Charge Transfer Inefficiency (CTI) measures the
  loss; i.e. CTI = 1 - CTE. Most of the charge lost during a transfer
  will reappear during a later transfer, which causes `charge tails'
  away from the sources towards the read-out register. These trails impact photometry, noise, shape and
  astrometry of sources. CTI trails result in the following problems;
\begin{enumerate}
\item They remove flux from the central pixel, hence degrading the expected S/N for an observation; 
\item They bias measurements of sources along the trail, impacting astrometry; 
\item CTI from warm pixels and cosmic rays introduce noise in observations.
\end{enumerate}
For a space-based telescope, CTE degrades over time, as the
bombardment by 
energetic particles generate more defects in the lattice structure of the CCDs. 
The impact of CTE, or CTI, on weak lensing has been investigated in 
\cite{2007ApJS..172..203R,2010PASP..122..439R,2010MNRAS.401..371M,2010MNRAS.409L.109M,2014MNRAS.439..887M,2015MNRAS.453..561I}.
}%

%%%%%%%%%%%%%%%%
\dictentry{dictionary}{CTI}{
{\scshape{\footnotesize see} \gls{CTE}}.
}%

%%%%%%%%%%%%%%%%
\dictentry{dictionary}{Deflection Angle}{\DeflectionAngle}
\newcommand\DeflectionAngle{
The angle between an incoming and outgoing ray, and deflection by lens, for a geometrically-thin
lens is called the deflection angle, which is defined as 
\setcounter{equation}{0}
\renewcommand{\theequation}{DA.\arabic{equation}}
\be
\hat{\alpha}=-\frac{2}{c^2}\int \nabla_{\bot} \Phi dr \; ,
\ee
i.e. gradient of the potential $\Phi$ is taken perpendicular to the
light path. The deflection angle is twice the classical prediction in
Newtonian dynamics if photons were massive particles with velocity $c$ 
{\scshape{\footnotesize see} \gls{Surface Density}; \gls{Weak Lensing Equations}}.
}%

%%%%%%%%%%%%%%%% 
\dictentry{dictionary}{DEIMOS}{
DEIMOS \cite{DEIMOS} is a moment-based method for weak lensing shear estimation from galaxies, 
that does not make some of the assumptions or approximations that are made in applying the KSB algorithm \cite{1995ApJ...449..460K}. 
DEIMOS directly estimates the lensed moments from the measured moments, which are affected by PSF convolution 
and the application of a weighting function. One improvement over KSB is that an exact deconvolution from the PSF is 
performed that requires only the knowledge of PSF moments of the same order as the galaxy moments to be corrected.
}%

%%%%%%%%%%%%%%%%
\dictentry{dictionary}{DES}{
The Dark Energy Survey (DES\footnote{\url{http://www.darkenergysurvey.org}}) 
will image 5000 square degrees of the southern sky in 5 optical filters using the DECam instrument on 
the CTIO. DES started taking data in August of 2013 and 
will continue for five years to record information from 300 million galaxies for a redshift range of $0.2<z<1.3$.
}%

%%%%%%%%%%%%%%%%
\dictentry{dictionary}{DETF}{
The Dark Energy Task Force (DETF) was a subcommittee established in February 2005 by the Astronomy and Astrophysics 
Advisory Committee (AAAC) and the High Energy Physics Advisory Panel (HEPAP) in USA to 
advise NSF (National Science Federation), NASA and DOE (Department of Energy) on the future of dark energy research. 
The DETF had a remit to help the agencies to identify actions for a dark energy program and understanding the 
nature of dark energy. The report of DETF can be found in \cite{DETF}, and a second 
report from the same team was performed in 2009 \citep{2009arXiv09010721A}. {\scshape{\footnotesize see} \gls{Stage (dark energy)}}.
}%

%%%%%%%%%%%%%%%%
\dictentry{dictionary}{Dither}{\Dither}
\newcommand\Dither{
A technique that consists of slightly moving the telescope between `exposures', i.e. within each slew. One purpose of `dithering' 
is to reduce some types of CCD sensor noise such as pattern noise. A relatively standard method is to randomly move the 
telescope, known as random dithering. Dithering causes the resultant image to land on a different sets of pixels from one dither to the next. 

The advantage of dithering is chip artefacts can be minimised when the different `sub' exposures are combined. 
The best signal-to-noise ratio improvement that can be achieved, without data smoothing, is the mean combine.  
For $n$ frames, a mean combine, improves signal-to-noise by $\sqrt{n}$. 
However a mean combine does nothing to eliminate artefacts, except by reducing their visibility by $1/n$.  
A median combine improves signal-to-noise by $\sqrt{2n/3}$, or about $20\%$ less than the mean but does 
reduce some artefacts. Another combine method is MMClip\footnote{\url{http://www.hiddenloft.com/notes/dithering.htm}} 
which improves signal-to-noise by $\sqrt{n-2}$.
}%

%%%%%%%%%%%%%%%%
\dictentry{dictionary}{Down-sizing}{
First suggested by \cite{CowieEtAl96} to refer to a scenario of galaxy formation where massive galaxies were 
formed first in the history of the Universe, and completed their star formation process more rapidly than the low-mass galaxies. 
This scenario is in contrast with other scenarios in which a simple hierarchical structure formation is assumed: 
where large galaxies are formed from the merging of smaller galaxies
and hence are formed later in the history of the Universe.}%

%%%%%%%%%%%%%%%%
\dictentry{dictionary}{E-mode}{
{\scshape{\footnotesize see} \gls{EB-mode Decomposition}}.
}%

%%%%%%%%%%%%%%%%
\dictentry{dictionary}{Filter Function}{
A function or procedure which removes unwanted parts of a signal or transforms the signal into another form. 
For example, one can take Fourier transform of a signal into frequency space, perform the filtering operation there (i.e. down-weighting 
or removing particular frequencies), then transform back into the original space.
}%

%%%%%%%%%%%%%%%%
\dictentry{dictionary}{Fisher Matrix}{\FisherMatrix}
\newcommand\FisherMatrix{
The Fisher (information) matrix is a way of measuring the amount of information that an observable, or random variable, 
carries about an unknown parameter upon which the probability of the variable depends. 
In cosmology and weak lensing, the Fisher matrix is generally used to determine the sensitivity of a 
particular survey to a set of parameters and has been used largely 
for forecasting and optimisation. It is defined as the ensemble
average of the \textit{curvature} of the likelihood function 
\setcounter{equation}{0}
\renewcommand{\theequation}{FM.\arabic{equation}}
\be
F_{\alpha\beta}=\left\langle -\frac{\partial^{2}\ln\mathcal{L}}{\partial\theta_{\alpha}\partial\theta_{\beta}}\right\rangle \label{eq:General_FM}\;,
\ee
where $\mathcal{L}$ is the likelihood function and $\theta_{\alpha}$ and $\theta_{\beta}$ are cosmological parameters. 
Its inverse is an approximation of the covariance matrix of the parameters, by analogy with a 
Gaussian distribution in the $\theta_{\alpha}$, for
which is exact --- this was investigated for CMB in \citep{1997ApJ...480...22T}. 
Therefore, using the assumption of Gaussianity in the data, it allows us to 
estimate the errors on parameters without
having to cover the whole parameter space. The authors of \cite{BJK} have compared
the Fisher matrix analysis with the full likelihood function analysis
and found there was great agreement between the two methods if the
likelihood function is approximately Gaussian near the peak.
}%

%%%%%%%%%%%%%%%%
\dictentry{dictionary}{Flat-Sky}{
Considered when the area of sky under consideration is small enough that geometric effects caused by observing on the 
celestial sphere are negligible and can be ignored. 
The limit and area of the sky that can be considered flat may change depending on the function that is being calculated.
}%

%%%%%%%%%%%%%%%%
\dictentry{dictionary}{Flexion}{\Flexion}
\newcommand\Flexion{
Flexion is used to refer to effects caused by second-order derivatives of the deflection angle, 
or third-order derivatives of the lens potential in the lens equation. While shear gives rise to elliptical image distortions, 
the combined effect of shear and flexion gives rise to a arc-like (also referred to as arclet, or `bannana'-like) 
image distortions \cite{2002ApJ...564...65G,2005ApJ...619..741G}. Flexion statistics are in principle sensitive to 
local variations of the potential field and hence can in principle probe small and non-linear scales. 
In application there are several systematics that need to be overcome \cite{2006MNRAS.365..414B}, for example
Poisson noise may be higher for flexion, and flexion may be more sensitive to photon pixel noise than shear 
\cite{2013MNRAS.435..822R}.
}%

%%%%%%%%%%%%%%%%
\dictentry{dictionary}{Focal Plane}{\FocalPlane}
\newcommand\FocalPlane{For a lens, or a spherical/parabolic mirror, 
the focal plane is the plane of points onto which light, parallel to the axis, is focused. 
The distance between the lens or mirror and the focus is called the focal length. 
In telescopes cameras are placed at the focal plane to collect data from the telescope.
}%

%%%%%%%%%%%%%%%%
\dictentry{dictionary}{FoM}{
Acronym for Figure of Merit; {\scshape{\footnotesize see} \gls{Figure of Merit}}.
}%

%%%%%%%%%%%%%%%%
\dictentry{dictionary}{Figure of Merit}{
A figure of merit (FoM) is a quantity that is used to characterize the performance of a survey relative to its alternatives. 
FoMs should allow for a determination of the ability of any given cosmological probe or individual experiment 
to measure the properties of interest e.g. parameters for the dark energy equation of state. 
It should, ideally, be represented by one number or metric. In weak lensing studies and cosmology, FoMs have been defined as 
some function of the Fisher matrix of the parameters of interest. Some example are: 
\begin{itemize}
\item A-optimality $=\log(\textrm{trace}(\mathbf{F}))$\\
trace of the Fisher matrix (or its $\log$) and is proportional to
sum of the variances. This prefers a spherical error region, but may
not necessarily select the smallest volume. 
\item D-optimality $=\log\left(\left|\mathbf{F}\right|\right)$ \\
determinant of the Fisher matrix (or its $\log$), which measures
the inverse of the square of the parameter volume enclosed by the
posterior. This is a good indicator of the overall size of the error
over all parameter space, but is not sensitive to any degeneracies
amongst the parameters. 
\item Entropy (also called the Kullback-Leibler divergence) 
\setcounter{equation}{0}
\renewcommand{\theequation}{FOM.\arabic{equation}}
\begin{eqnarray}
E & = & \int d\theta\; P(\theta|\hat{\theta},e,o)\log\frac{P(\theta|\hat{\theta},e,o)}{P(\theta|o)}\nonumber \\
 & = & \frac{1}{2}\left[\log\left|\mathbf{F}\right|-\log|\mathbf{\Pi}|-\textrm{trace}(\mathbb{I}-\mathbf{\Pi}\mathbf{F}^{-1})\right]\;,
\end{eqnarray}
where $P(\theta|\hat{\theta},e,o)$ is the posterior distribution
with Fisher matrix $\mathbf{F}$ and $P(\theta|o)$ is the prior distribution
with Fisher matrix $\mathbf{\Pi}$. 
\end{itemize}
For dark energy surveys there was a particular recommendation from the Dark Energy Task Force \citep{DETF}, 
{\scshape{\footnotesize see} \gls{Stage (dark energy)}}.
}%

%%%%%%%%%%%%%%%%
\dictentry{dictionary}{Foreground}{
Part of a view that is radially closer to the observer relative to the desired source. Normally `foregrounds' are used to refer to unwanted non-cosmological signals, 
and need to be removed from the data, in particular in CMB analysis see e.g. \cite{2016arXiv160603606D}. In weak lensing 
the most dominant foreground contaminant is zodiacal dust at low ecliptic latitudes. Intrinsic 
alignments are occasionally referred to as a `foreground' contaminant. 
Foreground estimation and characterisation and removal is an essential
step in the data analysis pipeline.}%

%%%%%%%%%%%%%%%%%
\dictentry{dictionary}{Fourier Space (in reference to cosmology)}{
This refers to the Fourier, spherical harmonic, or spherical Bessel transforms of a field in cosmology, 
for example the matter overdensity, interpolated 
galaxy overdensity, ellipticity or CMB temperature or polarisation
fields. In the angular direction 
the Fourier wavenumber is denoted $\ell$ (which is a dimensionless quantity) and in the radial 
(comoving distance) direction as $k$ (with units of $h$Mpc$^{-1}$). The article may also refer to $\ell$-space, 
or $k$-space to refer to the coordinate systems after such a transform.
{\scshape{\footnotesize see} \gls{Fourier Transform}}.
}%

%%%%%%%%%%%%%%%%
\dictentry{dictionary}{Fourier Space (in reference to PSF)}{
In reference to a Point-Spread-Function estimation, Fourier space refers to the 2D Fourier transform of the optical wavefront, see 
e.g. \cite{2011PASP..123..596J}.
{\scshape{\footnotesize see} \gls{Fourier Transform}}.
}%

%%%%%%%%%%%%%%%%
\dictentry{dictionary}{FoV}{
Acronym for Field of View; {\scshape{\footnotesize see} \gls{Field of View}}.
}%

%%%%%%%%%%%%%%%%
\dictentry{dictionary}{Field of View}{
Field of View (FoV) is a solid angle through which a detector is sensitive to electromagnetic radiation.
}%

%%%%%%%%%%%%%%%%
\dictentry{dictionary}{Future (Survey)}{
This phrase is used to refer to a (weak lensing) survey that has not yet started observations. 
In literature from c. 2000 to 2005 such a phrase refers to Stage II like surveys such as CFHTLS; 
up until c. 2010 in reference to DES, HSC, and KiDS; and up until the present 
time (c. 2016) to LSST, Euclid, SKA etc.}%

%%%%%%%%%%%%%%%%
\dictentry{dictionary}{FWHM}{
Full width at half maximum (FWHM) is the distance between points on a curve where the function has 
half its maximum value. This parameter is commonly used to describe the width of the curve or width of a function.
}%

%%%%%%%%%%%%%%%%
\dictentry{dictionary}{Galaxy}{
A gravitationally bound system of stars, interstellar gas and dust which is hypothesised to live in a dark matter halo. 
Galaxies can have from a few thousand stars to one hundred trillion stars. Galaxies have historically been categorised 
according to their morphology and include elliptical, spiral, and irregular. Many galaxies are hypothesised to have blackholes at their centres.
}%

%%%%%%%%%%%%%%%%
\dictentry{dictionary}{Galaxy Size}{\gsize}
\newcommand\gsize{
Galaxy size can refer to several different properties of a galaxy surface brightness distribution, 
all of which have some relation to the 
projected angular extent of a galaxy as observed. Commonly used definitions include 
\begin{itemize} 
\item 
The sum of the Cartesian axis quadrupole moments $R^2=Q_{11}+Q_{22}$ {\scshape{\footnotesize see} \gls{Quadrupole Moments}; \gls{R-squared}}, 
see e.g. \cite{2013MNRAS.429..661M}.  
\item 
The FWHM (full width half maximum) of the surface brightness distribution, 
\item 
A measure of the size through a model fitting procedure that has a free parameter controlling the radial width of the model, 
such as an exponential profile with a particular scale length, see e.g. \cite{2007MNRAS.382..315M}. 
\end{itemize} 
These definitions are also used in PSF fitting to refer to PSF, or stellar, size. 
}%

%%%%%%%%%%%%%%%%
\dictentry{dictionary}{Galaxy Useful for Weak Lensing}{
A galaxy `useful for weal lensing' is one which has been deemed by a particular study, using observational data, 
to have passed any selection criteria used to remove galaxies that may have systematic effects present in their derived quantities. 
Selection criteria typically include signal-to-noise, galaxy size, galaxy colours, the quality of a model fit 
(e.g. a $\chi^2$ measurement), the size of the galaxy relative to the size of the estimated PSF, or similar. 
There is no universally agreed criteria for this definition, and each study typically justifies its own selection 
based on the quality of the data and the measurement techniques it uses.
}%

%%%%%%%%%%%%%%%%
\dictentry{dictionary}{Galaxy-Galaxy Lensing}{
Galaxy-galaxy lensing (GGL) can be weak or, occasionally, strong gravitational lensing, where the 
lens is an individual galaxy. GGL, therefore, 
correlates shapes of source galaxies with positions of the lens galaxies, and hence gives information about 
the mass of the lens galaxy (or sample of galaxies). Hence it probes galaxy halos on scales of about 
a few kpc to a few Mpc, providing insights about halo masses and density profiles as function of 
stellar mass, luminosity, type, or environment \cite{2006MNRAS.368..715M}. GGL produces shear 
correlations of about 1\%, which is weaker than cluster lensing, but stronger than the signal due to cosmic shear.
}%

%%%%%%%%%%%%%%%%
\dictentry{dictionary}{Galaxy-Galaxy-Galaxy lensing}{
Galaxy-galaxy-galaxy lensing (G3L) refers to third-order correlations
between mass and galaxy ellipticities, which can involve 
two source galaxies and one lens galaxy, or one source galaxy and two lens galaxies 
\cite{2005A&A...432..783S,2013MNRAS.430.2476S}. This technique is very useful in understanding 
the mass and environments of close correlated galaxies.
}%

%%%%%%%%%%%%%%%%
\dictentry{dictionary}{Gaussian Noise}{
A statistical noise with a probability density function (PDF) equal to Gaussian distribution, 
{\scshape{\footnotesize see} \gls{Probability Distribution}}. 
White Gaussian noise is an identically and independently distributed noise, meaning it is sampled from a distribution with a constant power and zero correlation.
}%

%%%%%%%%%%%%%%%%
\dictentry{dictionary}{Intrinsic Ellipticity Distribution}{
The intrinsic ellipticity distribution is used to refer to an estimate of the probability density function of galaxy ellipticities, in a population of galaxies, before the the lensing effect. These distributions are referred to in, for example \citet{2007MNRAS.382..315M}.
}%

%%%%%%%%%%%%%%%%
\dictentry{dictionary}{Inter-pixel Responsivity}{
The responsively of a pixel in a CCD varies from pixel-to-pixel across the CCD. This is due to differences in the manufacturing 
of the pixels, such as pixel area or thickness of pixel surface layers. These inter-pixel variations in a CCD should be kept small for 
weak lensing measurement to be possible, see e.g. \cite{2016PASP..128i5001K}.
}%

%%%%%%%%%%%%%%%%
\dictentry{dictionary}{Intrinsic}{\Intrinsic}
\newcommand\Intrinsic{
An `intrinsic' property is a property of a system itself, un-affected by external influences. 
For example, in gravitational lensing, the ellipticity of a galaxy \emph{before} it is lensed is 
called the intrinsic ellipticity of that galaxy; {\scshape{\footnotesize see} \gls{Intrinsic Alignment}}. 

A property that is not inherent is called an extrinsic property. An example in physics 
is the density of an object, which is an intrinsic property, whereas the 
weight of the object is an extrinsic property as it depends on the strength of the gravitational field on the object.
}%

%%%%%%%%%%%%%%%%
\dictentry{dictionary}{Kaiser-Squires Mass Reconstruction}{
This is used to refer to the inverse method of estimating a map covergence 
from estimates of galaxy shear, which was first described in \citet{1993ApJ...404..441K}; 
{\scshape{\footnotesize see} \gls{Mass Map}; \gls{3D Mass Map}; \gls{Kappa}; \gls{Potential Map}}.
}%

%%%%%%%%%%%%%%%%
\dictentry{dictionary}{Kappa}{\kappaa}
\newcommand\kappaa{This refers to the `convergence' of the lensing potential, which is denoted using the symbol 
$\kappa$ and referred to in shorthand as `kappa', i.e. in `kappa map' means `weak lensing covergence map'; 
{\scshape{\footnotesize see} \gls{Mass Map}; \gls{3D Mass Map}; \gls{Kappa}; \gls{Potential Map}}. The convergence is related to the projected lensing potential by 
\setcounter{equation}{0}
\renewcommand{\theequation}{KP.\arabic{equation}}
\be 
\kappa(\theta)=\frac{1}{2}\eth\tilde\eth\phi(\theta),
\ee
where $\theta$ is an angular coordinate on the sky, $\phi$ is the projected lensing potential 
and $\eth$ is a complex derivative on the sky ({\scshape{\footnotesize see} \gls{eth Differential Operator}}) which 
are simplified covariant derivatives in a Cartesian coordinate system, see e.g. \citep{2005PhRvD..72b3516C}. The 
convergence is a spin-0 quantity.
{\scshape{\footnotesize see} \gls{Surface Density}; \gls{Weak Lensing Equations}}.
}%

%%%%%%%%%%%%%%%%
\dictentry{dictionary}{KiDS}{\kids}
\newcommand\kids{
The Kilo-Degree Survey KiDS\footnote{\url{http://kids.strw.leidenuniv.nl/index.php}} 
is one of the public surveys on the VST telescope (operated and maintained by ESO). 
KiDS will survey two areas of extragalactic sky, which are roughly 750 square degrees each, 
to a median redshift of 0.7. KiDS observed in four bands (u,g,r,i). 
}% 

%%%%%%%%%%%%%%%%
\dictentry{dictionary}{VIKING}{
VIKING\footnote{\url{https://www.eso.org/sci/observing/PublicSurveys/sciencePublicSurveys.html}} 
is one of the near-infrared public surveys planned on the VISTA telescope. VIKING 
complements KiDS, {\scshape{\footnotesize see} KiDS}, by observing the same area at five different infrared bands (Z,Y,J,H,K). 
}%

%%%%%%%%%%%%%%%%
\dictentry{dictionary}{Knot}{\knot}
\newcommand\knot{
This is used to refer to a topological feature in a two or three-dimensional field that occurs where several other 
lower-dimensional feature (e.g. lines, filaments or planes) intersect. This occur in several instances in weak lensing 
literature, for example:  
\begin{enumerate}
\item  
Knots in the matter distribution of the Universe, the cosmic web, can occur and 
form an approximately spherically symmetric overdensity. Such topological features can be studied using 
galaxy-galaxy lensing, see e.g. \cite{2016arXiv160407233B}.
\item 
Knots in strong lensing maps can also occut in the observed caustic features of strong lensing. 
\end{enumerate}}%

%%%%%%%%%%%%%%%%
\dictentry{dictionary}{Knowledge Requirement}{
Refers to the need to know the error distribution of a desired quantity for use in a weak lensing measurement. 
For example for a quantity $A$, a `knowledge requirement' refers to the need to know the value of $A$ to high precision; 
in the case that $A$ is expected to be Gaussian distributed it would be sufficient to specify the mean 
and variance of $A$; in other cases the full probability distribution. A particular case is 
that of shape measurement biases where the known error on a methods bias is required. Such requirements are 
discussed in \cite{2013MNRAS.429..661M}.
}%

%%%%%%%%%%%%%%%%
\dictentry{dictionary}{KSB}{
A perturbative approach for a direct measurement of galaxies ellipticities 
\cite{1995ApJ...449..460K}. Improvements of the KSB method have also been 
achieved by \cite{1997ApJ...475...20L,1998NewAR..42..137H,2000ApJ...537..555K,2002AJ....123..583B}, and improvements by 
extension to higher perturbation order have been achieved by \cite{2009ApJ...699..143O}.
}%

%%%%%%%%%%%%%%%%
\dictentry{dictionary}{l-mode}{
$\ell$-mode is the angular wavenumber that corresponds to an angular scale via $\ell \simeq \pi/\theta$.
{\scshape{\footnotesize see} \gls{Fourier Transform}; \gls{Fourier Space (in reference to cosmology)}}.
}%

%%%%%%%%%%%%%%%%
\dictentry{dictionary}{Large Scales}{
Refers to a range of physical scales, or Fourier wavenumbers, that are deemed to be `large(r)' 
than some other set of scales that are being referred to. A common use is in reference to the 
matter power spectrum where `large scales' refers to the part of the matter power spectrum 
that can be computed using linear perturbation theory of the matter overdensity distribution, 
{\scshape{\footnotesize see} \gls{Linear (Matter Power Spectrum)}}. This is also used as a synonym for linear scales
}%

%%%%%%%%%%%%%%%%
\dictentry{dictionary}{LCDM}{\LCDM}
\newcommand\LCDM{
Lambda cold dark matter, or $\Lambda$CDM, model is currently the standard model of cosmology. 
This is a parametrisation of the Big Bang cosmological model and in such Universe there is 
a cosmological constant, $\Lambda$ and cold dark matter (CDM). The $\Lambda$CDM is the favourite 
model currently as it is the simplest model that can explain many of the observations such as: 
\begin{enumerate}
\item the cosmic microwave background; 
\item the large-scale structure; 
\item the abundances of hydrogen, helium, and lithium; 
\item the current accelerating expansion of the Universe. 
\end{enumerate}
The model assumes that general relativity is the correct theory of gravity on cosmological scales. For a review of the 
cosmological parameters that describe this model see \cite{2006JPhG...33....1Y,2014arXiv1401.1389L}.
}%

%%%%%%%%%%%%%%%%
\dictentry{dictionary}{Lens}{
In cosmology a lens can be a galaxy, galaxy cluster, matter overdensity, or any other mass perturbation 
that is causing gravitational lensing of background objects along the line of site.
}%

%%%%%%%%%%%%%%%%
\dictentry{dictionary}{Lens Equation}{
An equation that relates the angular position of light ray at which it would have been 
observed in the absence of gravitational lensing $\beta$, to the actual angular observed $\theta$, and the deflection 
angle in the path of the light ray caused by a gravitational lens $\alpha$
\setcounter{equation}{0}
\renewcommand{\theequation}{LE.\arabic{equation}}
\be
\beta=\theta-\alpha\frac{D_{LS}}{D_S}\;,
\ee
where $D_{LS}$ is the angular diameter distance between the source and the lens, and $D_S$ is the 
angular diameter distance to the source. {\scshape{\footnotesize see} \gls{Distortion Matrix}; 
\gls{Weak Lensing Equations}}.
}%

%%%%%%%%%%%%%%%%
\dictentry{dictionary}{Lens Plane}{
The transverse 2D projection along the line-of-sight at which it is assumed the 
change in light paths geodesic is affected by the presence of a massive object causing gravitational 
lensing; i.e. the plane at which the lens sits.}%

%%%%%%%%%%%%%%%%
\dictentry{dictionary}{Lensfit}{
A likelihood-based shear measurement method proposed by \cite{2007MNRAS.382..315M,2008MNRAS.390..149K}. 
The method is based on creating pixel-based models of the varying point spread function (PSF) in each image exposure. 
It fits PSF-convolved models to measure the ellipticity of each of the galaxies in the image, while marginalising 
over galaxy position, size, brightness and bulge fraction. The method can optimally deal with joint measurement of 
multiple dithered image exposures. It takes into account imaging distortions and possible alignment of the 
multiple measurements.
}%

%%%%%%%%%%%%%%%%
\dictentry{dictionary}{Lensing}{
Used as shorthand to refer to gravitational lensing. It is ambiguous
in reference to 
weak, strong, or micro lensing; but is usually clear within context. In general lensing means 
`gravitational light deflection'.}%

%%%%%%%%%%%%%%%%
\dictentry{dictionary}{Lensing Kernel}{
The lensing kernel, or lensing efficiency, defines the efficiency of
lensing for a distribution of sources and 
lenses with respect to the observers. The kernel is broad and it is most sensitive to structure halfway 
between the observer and the source. The kernel is defined as the combination of distances that appears in the 
len equation {\scshape{\footnotesize see} Lens Equation}
}%

%%%%%%%%%%%%%%%%
\dictentry{dictionary}{Mask}{
Mask is normally a binary (multi-dimensional) matrix, which specifies the regions on the sky that one should, or should not, 
take into account for further analysis. Normally, the unwanted data has zero elements in the corresponding matrix. 
The mask need not be binary. In some cases, for e.g. the CMB data, the mask is apodised, meaning there is a 
smooth function that defines the transition between pixels with 1s and 0s. A mask can cause aliasing for power 
spectrum measurements that can be accounted for using Pseudo-Cl analyses, see e.g. \cite{MASTER} and 
{\scshape{\footnotesize see} \gls{Pseudo-Cl}}.
}%

%%%%%%%%%%%%%%%%
\dictentry{dictionary}{Mass Map}{
Or kappa map ($\kappa$ map) is the map of the total matter, which is 
reconstructed from a shear map in lensing surveys. 
{\scshape{\footnotesize see} \gls{Kappa}; \gls{Potential Map}; \gls{3D Mass Map}}.
}%

%%%%%%%%%%%%%%%%
\dictentry{dictionary}{Matched Filter}{
Usually obtained by correlating a template with an unknown signal to detect the presence of the template in the unknown signal. 
The matched filter is the optimal linear filter for maximising the signal to noise ratio (SNR) in the presence of 
additive stochastic noise. In weak lensing matched filters have been used to identify galaxy clusters, see e.g. 
\cite{2015MNRAS.447.1304F}.
}%

%%%%%%%%%%%%%%%%
\dictentry{dictionary}{Minkowki Functional}{
In mathematics, in the field of functional analysis, a Minkowski functional is a function that recovers a notion 
of distance on a linear space. In cosmology, they are the main tools to characterise the large-scale galaxy 
distribution in the Universe. A tutorial on the use of Minkowki Functionals in cosmology is provided by \cite{1996dmu..conf..281S}. 
In weak lensing Minkowski functionals may be used to test modified gravity 
theories \citep{2015PhRvD..91j3511P,2014ApJ...786...43S,2013PhRvD..88l3002P}.
}%

%%%%%%%%%%%%%%%%
\dictentry{dictionary}{Simulation}{
This is the imitation of the operation of a process or system over time, such as the Universe. 
Numerical simulations play a significant role in cosmology. The first simulations started in 1960s and 1970s 
\citep{1963MNRAS.126..223A,1970AJ.....75...13P,1974ApJ...187..425P}. The important factors that should be 
taken into account for simulations are 1) setting the initial conditions, 2) setting the equations of evolution of 
fluctuations that were set in the initial conditions. Some of the simulations of the Universe include\\
1. Millennium \url{https://wwwmpa.mpa-garching.mpg.de/galform/virgo/millennium/}\\
2. Illustris \url{http://www.illustris-project.org} \\
3. EAGLE \url{http://icc.dur.ac.uk/Eagle/}.\\
Simulations are also extensively used to capture the complexity of different steps in the 'end-to-end' 
pipelines in different experiments. For example, to estimate cosmological parameters, simulations are used to 
estimate the covariance matrix. Simulations are also used to test different methods and algorithms available 
before taking data in experiments. See e.g. \cite{2000astro.ph..5502K} for a review. 
Simulations of the imaging process in weak lensing can also be performed. 
{\scshape{\footnotesize see} \gls{STEP}; \gls{GREAT}}.
}%

%%%%%%%%%%%%%%%%
\dictentry{dictionary}{Single-epoch}{
Weak lensing observations that were taken a single time; the
singular case of `multi-epoch'. See {\scshape{\footnotesize see} \gls{Multi-Epoch}}.}%

%%%%%%%%%%%%%%%%
\dictentry{dictionary}{Singular Isothermal Sphere}{\SingularIsothermalSphere}
\newcommand\SingularIsothermalSphere{
Singular Isothermal Sphere (SIS) is the simplest symmetrical parameterisation of 
the matter distribution in astronomical systems, such as galaxies. The density distribution is defined as 
\setcounter{equation}{0}
\renewcommand{\theequation}{SIS.\arabic{equation}}
\be
\rho(r) = \frac{\sigma_{V}^{2}}{2\pi G r^{2}} \; ,
\ee
where $\sigma_{V}^{2}$ is the velocity dispersion. The SIS profile is unphysical because of the singularity at 
zero radius and the fact that the total mass calculated by integrating the function out to infinite radius 
does not converge (i.e., it is infinite). However, it is commonly utilised in the literature due to the simplicity of its form.
In lensing, SIS is one of the simplest axially symmetric analytical models to describe the matter distribution of 
extended lenses. The main advantage of using axially symmetric lenses is that their surface density 
is independent on the position angle with respect to lens centre and this reduces the lensing equations a 
one-dimensional form.
}%

%%%%%%%%%%%%%%%%
\dictentry{dictionary}{Size Magnification}{
Gravitational lensing conserves surface brightness, a consequence from Liouville's theorem which holds in any 
passive optical system. Since the apparent size of resolved background objects change, their flux changes as well. 
These two effects are manifestations of gravitational magnification, and can be used as weak-lensing 
observables in addition to the deformation (shear) of galaxy shapes. 
See e.g. \cite{2013MNRAS.430.2844C} for a recent study on this. 
{\scshape{\footnotesize see} \gls{Magnification}}.
}%

%%%%%%%%%%%%%%%%
\dictentry{dictionary}{SKA}{\SKA}
\newcommand\SKA{
The Square Kilometre Array (SKA)\footnote{\url{https://www.skatelescope.org/project/}} 
is an international radio telescope project that will be built in Australia and South Africa, 
with a total collecting area of approximately one square kilometre, with receiving stations extending out to distance of at 
least 3,000 kilometres from a central core. SKA will be 50 times more sensitive than any other radio instrument. It will 
cover a wide range of frequencies, using thousands of dishes and up to a million antennas. It will be able to survey the sky more than ten thousand times faster than ever before. Construction of the SKA is scheduled to begin in 2018 for initial observations by 2020.
}%

%%%%%%%%%%%%%%%%
\dictentry{dictionary}{Small-Scales}{
Scales that are not defined, within context, as `large scales' ({\scshape{\footnotesize see} \gls{Large Scales}}). Usually used 
in reference to the part of the matter power spectrum that cannot be computed using linear perturbation theory of the 
matter over-density field, and requires N-body simulations to determine its properties and behaviour; 
{\scshape{\footnotesize see} \gls{Linear (Matter Power Spectrum)}}.}%

%%%%%%%%%%%%%%%%
\dictentry{dictionary}{Source}{
In lensing source galaxies are the ones whose images are lensed.
}%

%%%%%%%%%%%%%%%%
\dictentry{dictionary}{Source Clustering}{
When the source galaxies have intrinsic clustering due to their local environment and 
gravitational attraction between the local galaxies. \cite{1998A&A...338..375B} have studied the effect of source clustering on weak lensing statistics.
}%

%%%%%%%%%%%%%%%%
\dictentry{dictionary}{Source Plane}{
In lensing a hypothetical surface at the comoving distance upon which source galaxies lie. 
{\scshape{\footnotesize see} \gls{Lens Plane}; Image Plane}.
}%

%%%%%%%%%%%%%%%%
\dictentry{dictionary}{Source Population}{
Population of source galaxies on a source plane, or series of source planes.
}%

%%%%%%%%%%%%%%%%
\dictentry{dictionary}{Specz}{
Shorthand for `a spectroscopic redshift', where a prism is used to measure the intensity of light across frequency (or wavelength) of characteristic spectral lines. The shift of these lines with respect to their laboratory positions gives the redshift of the object.
}%

%%%%%%%%%%%%%%%%
\dictentry{dictionary}{Photoz}{
Shorthand for `a photometric redshift'. {\scshape{\footnotesize see} \gls{Photometric Redshifts (Photo-z; photo-zee; photo-zed)}}.
}%

%%%%%%%%%%%%%%%%
\dictentry{dictionary}{Specz-Photz Plot}{
A scatter plot of spectroscopic redshift estimates, for a population of galaxies, 
versus an estimator for the photometric redshifts of the same galaxies. Typically used to show 
graphically the ability of a photometric redshift estimation code to determine the spectroscopic redshifts of galaxies, 
see e.g. \cite{2010A&A...523A..31H} for use of this representation to determine photometric redshift accuracy and precision; 
{\scshape{\footnotesize see} \gls{Photometric Redshifts (Photo-z; photo-zee; photo-zed)}}.
}%

%%%%%%%%%%%%%%%%
\dictentry{dictionary}{Spherical Harmonics}{\sphh}
\newcommand\sphh{
A series of functions defined on the surface of a sphere used to solve some kinds of differential equations. 
They are similar to Fourier transform, but operate on the surface of a sphere. Spherical 
harmonics are defined as the angular portion of a set of solutions to Laplace's equation in three dimensions. Represented 
in a system of spherical coordinates with $\theta$ and $\phi$, Laplace's spherical harmonics $Y_\ell^m$ are 
a specific set of spherical harmonics that forms an orthogonal system, first introduced 
by Pierre Simon de Laplace in 1782. They form a complete orthonormal set on the unit sphere and are defined as
\setcounter{equation}{0}
\renewcommand{\theequation}{SH.\arabic{equation}}
\be
Y_\ell^m = \sqrt{\frac{2\ell + 1}{4\pi} \frac{(l-m)!}{(l + m)!}} P_\ell^m(\cos\theta)e^{im\phi} \; , 
\ee
where $\ell$ is the multipole representing angular scale $\alpha$ as $\alpha \simeq \pi / \ell$, with $\ell = 0,...,\infty$. 
The order $m$ ranges as $-\ell \le m \le \ell$ and $P_\ell^m$ are the Legendre polynomials. Spherical harmonics 
are important in cosmology for representation of gravitational fields, planetary bodies and 
stars, and characterisation of the cosmic microwave background radiation. In weak lensing 
the cosmic shear field can be described using spherical harmonics supplemented by a radial Bessel function transform, 
see e.g. \cite{2005PhRvD..72b3516C}.
}%

%%%%%%%%%%%%%%%%
\dictentry{dictionary}{Spin}{\Spin}
\newcommand\Spin{
In quantum mechanics and particle physics, spin is an intrinsic form of angular momentum carried 
by elementary particles, composite particles (hadrons), and atomic nuclei. The conventional definition of 
the spin quantum number is $s = n/2$, where n can be any non-negative integer. 
Hence the allowed values of $s$ are $0,\; 1/2,\; 1,\; 3/2,\; 2$, etc.

Parallels are used in weak lensing where the shearing of images of objects is a spin-2 field, meaning under 
a rotation of $\pi$ in the coordinate system the field is left unchanged. A rotation by $\pi/2$ 
changes $\gamma_1$ to $\gamma_2$ and $\gamma_2$ to $-\gamma_1$. A rotation of counterclockwise 
$\phi$ changes $(\gamma_1 + i \gamma_2) \rightarrow  (\gamma_1 + i\gamma_2) \exp(-2i\phi)$; see e.g.  
\cite{2015RPPh...78h6901K}. 
The weak lensing ellipticity 
field is therefore a spin-2 field, which is locally symmetric under 180 degree transformations, whereas the 
convergence field is a spin-0 field (i.e. a scalar).
}%

%%%%%%%%%%%%%%%%
\dictentry{dictionary}{Reduced Shear g}{\redg}
\newcommand\redg{Weak lensing affects the {\it shapes} of distant objects, inducing a change in the image of a galaxy 
that is a combination of the shear and convergence. 
In a measurement of 
the ellipticity, or inference of shear, it is the `reduced shear' that is measured. This is defined as
\setcounter{equation}{0}
\renewcommand{\theequation}{RS.\arabic{equation}}
\be
g = \frac{\gamma}{1-\kappa} \;.
\ee
Where $\gamma$ is the (not reduced) shear, and $\kappa$ is the convergence. 
This has the same properties as shear such as spin-2 transformation properties. Weak lensing is 
the regime where the effect of gravitational lensing is very small, with both the 
convergence and the shear much smaller than unity. Therefore, $\gamma$ is a good approximation of $g$ to linear order. 
The total reduced shear $g_{\rm tot}$ contains contributions from both the gravitational reduced shear and intrinsic 
reduced shear due to the intrinsic alignments of galaxies.}%

%%%%%%%%%%%%%%%%
\dictentry{dictionary}{Redshift z}{\Redshift}
\newcommand\Redshift{
Redshift quantifies the amount by which the light from a distant object, such as a galaxy, 
is moved to the redder end of the spectrum. Galaxies have redshifts caused by 
the expansion of the Universe. The redshift is defined by 
\setcounter{equation}{0}
\renewcommand{\theequation}{ZZ.\arabic{equation}}
\be
z = \frac{\lambda_{\text{obs}} - \lambda_{\text{emit}}}{\lambda_{\text{emit}}}\;,
\ee
where $\lambda_\text{obs}$ is the observed wavelength and $\lambda_\text{emit}$ is the emitted/absorbed wavelength.
The redshift can be related to the dimensionless scale factor in the following way 
\be 
z= \frac{a(t_{\rm obs})}{a(t_{\rm emit})}-1
\ee
where $a(t)$ are the scale factors at the cosmic time $t$ (the proper time measured 
by an observer at rest to the local matter distribution, or `substratum'). 
Because, by convention, $a(t_{\rm obs}) = 1$ the redshift is usually related to the 
scale factor simply by $a = (1 + z)^{-1}$. For local galaxies the redshift is related to the apparent 
recessional velocity by $z\approx v/c$, where $v$ is the recessional velocity and $c$ is the speed of light in 
vacuum. Note that local galaxies could have a redshift (or a blueshift) due to their local peculiar velocity. 
}%

%%%%%%%%%%%%%%%%
\dictentry{dictionary}{Ray Tracing}{\rays}
\newcommand\rays{A technique of calculating the light propagation through large-scale structure in 
N-body simulations. The path along the line of sight is divided into several planes and light rays are 
followed from one plane to next plane along the deflected direction, which is calculated on 
the current lens plane. This approach takes into account non-linear couplings between lens planes, so there is no 
one-to-one mapping between the light cone of emitted rays from an abject at high redshift and the observer's field-of-view. 
Light rays are traced backwards from the observer to the emitting object plane to make sure each photon reaches the observer; 
see e.g. \cite{1986ApJ...310..568B,2000ApJ...530..547J}.
}%

%%%%%%%%%%%%%%%%
\dictentry{dictionary}{Raw Data}{
Raw data, or primary data, is collected data from a source that has not been processed or reduced.
}%

%%%%%%%%%%%%%%%%
\dictentry{dictionary}{Radial Profile (of a Galaxy)}{
The radial profile of galaxy refers to the projected intensity of light as a function of radius 
from the projected centre of a galaxy.
}%

%%%%%%%%%%%%%%%%
\dictentry{dictionary}{R-squared}{
A particular estimate for the size of a galaxy, obtained from galaxy images, using quadrupole moments. It is defined as 
\setcounter{equation}{0}
\renewcommand{\theequation}{RS.\arabic{equation}}
\be
R^2=(Q_{11}Q_{22}-Q_{12}^2) \;.
\ee 
In some weak lensing literature this is referred to as $A$, since it is also related to the projected area of the 
galaxy in question \citep{2015arXiv151205591I}. {\scshape{\footnotesize see} \gls{Ellipticity (observed)}; \gls{Quadrupole Moments}; \gls{Galaxy Size}}.
}%

%%%%%%%%%%%%%%%%
\dictentry{dictionary}{Quasi-linear (Matter Power Spectrum)}{
The part of the matter power spectrum at $k$-modes that are slightly larger than those that 
can be computed using linear perturbation theory of the matter over-density field. These regimes have 
approximately equal contributions from the linear and non-linear parts of the matter power spectrum. At 
a redshift of zero these 
are typically $0.1\ls k \ls 1 h$Mpc$^{-1}$; 
{\scshape{\footnotesize see} \gls{Linear (Matter Power Spectrum)}; \gls{Non-linear (Matter Power Spectrum)}}
}%

%%%%%%%%%%%%%%%%
\dictentry{dictionary}{Quadrupole Moments}{\QuadrupoleMoment}
\newcommand\QuadrupoleMoment{
The quadrupole moment is a second-rank tensor. 
It represents the ellipsoidal shape of an object. In weak lensing it is used to quantify the 
observed ellipticity of an object, such as a galaxy or a star, and is defined as 
\setcounter{equation}{0}
\renewcommand{\theequation}{QM.\arabic{equation}}
\be
Q_{ij} = \frac{\int{\rm d}^2x (x_i - \bar{x}_i) (x_j - \bar{x}_j) I(\mathbf{x}) W(\mathbf{x})} 
{\int{\rm d}x^2 I(\mathbf{x})W(\mathbf{x})}\;,
\ee
where $\{i , j\} = \{1, 2\}$ and $\mathbf{x}=(x_i,x_j)$ is the 2D
position of the object in the image. The weight function
$W(\mathbf{x})$ is typically assumed to be a multivariate Gaussian and
$I(\mathbf{x})$ is the flux of the image centred at
$(\bar{x}_i,\bar{x}_j)$. The three quadrupole moments $\mathbf{Q} = \left(Q_{11}, Q_{22}, Q_{12}\right)$ can be related to the ellipticity of an object in two ways;
\begin{enumerate}
\item third eccentricity 
\be
\mathbf{\chi} \equiv \chi_1 + i\chi_2 \equiv \frac{Q_{11} - Q_{22} + 2iQ_{12}}{Q_{11} + Q_{22}} \;.
\ee
{\scshape{\footnotesize see} Third Eccentricity}.
\item third flattening 
\be
\mathbf{\epsilon} \equiv \epsilon_1 + i\epsilon_2 \equiv \frac{Q_{11} - Q_{22} + 2iQ_{12}}{Q_{11} + Q_{22}+2R} \;,
\ee
where $R=(Q_{11}Q_{22}-Q_{12}^2)^{1/2}$ is a measure of the size of the image (in some weak lensing literature this is referred to as $R^2$; {\scshape{\footnotesize see} \gls{R-squared}}). {\scshape{\footnotesize see} \gls{Third Flattening}}.
\end{enumerate}
There is a simple relation between third flattening and third eccentricity $\mathbf{\chi} = 2 \mathbf{\epsilon}/(1 + | \epsilon |^2)$.
The definitions above reflect ellipticity as a complex quantity, defined by an amplitude $|\epsilon|$ or $|\chi |$ and a position angle $\theta$ such that, e.g., $\epsilon=|\epsilon|e^{2\mathrm{i}\theta}$, reflecting the fact that ellipses are symmetrical under a rotation of $\pi$.
{\scshape{\footnotesize see} \gls{Ellipticity (observed)}; \gls{Galaxy Size}}.
}%

%%%%%%%%%%%%%%%%
\dictentry{dictionary}{Monte Carlo}{\MonteCarlo}
\newcommand\MonteCarlo{
Refers to a process in which samples are chosen randomly from some distribution. A common example is a Monte-Carlo Markov Chain used to sample a probability distribution where at each step in the chain a sample is drawn randomly from some `proposal' distribution. However this phrase is used more colloquially to mean any random sample, for example ``just Monte-Carlo it'' would be a suggestion to run the process in question many times where each time a random set of variables in question are chosen. 

Monte Carlo methods/experiments/simulations are a broad class of computational algorithms that rely on repeated random sampling to obtain numerical results. They are often used in physical and mathematical problems and are most useful when it is difficult or impossible to use other mathematical methods. Monte Carlo methods are mainly used in three distinct problem classes: optimisation, numerical integration, and generating draws from a probability distribution.
In physics-related problems, Monte Carlo methods are quite useful for simulating systems with many coupled degrees of freedom.
In principle, Monte Carlo methods can be used to solve any problem having a probabilistic interpretation. By the law of large numbers, integrals described by the expected value of some random variable can be approximated by taking the empirical mean (a.k.a. the sample mean) of independent samples of the variable. When the probability distribution of the variable is too complex, mathematicians often use a Markov Chain Monte Carlo (MCMC) sampler. The central idea is to design a judicious Markov chain model with a prescribed stationary probability distribution. By the ergodic theorem, the stationary probability distribution is approximated by the empirical measures of the random states of the MCMC sampler
}%

%%%%%%%%%%%%%%%%
\dictentry{dictionary}{Newtonian Potential}{
Refers to the gravitational potentials. For example in an expanding universe in the following equation
\setcounter{equation}{0}
\renewcommand{\theequation}{NP.\arabic{equation}}
\be
{\rm d}s^2 = \left(1+\frac{2\Psi}{c^2}\right) c^2 {\rm d}t^2 - a^2(t)\left(1-\frac{2\Phi}{c^2}\right){\rm d}l^2 \;,
\ee
where ${\rm d}s$ is the line element, $a$ is the scale factor, $c$ is the speed of light, $\Psi$ is the Newtonian potential and $\Phi$ is the Newtonian curvature. {\scshape{\footnotesize see} \gls{Friedman--Robertson--Walker Models}}.
}%

%%%%%%%%%%%%%%%%
\dictentry{dictionary}{NFW}{
The Navarro-Frenk-White (NFW) profile describes the spatial distribution of dark matter halos that was fitted to N-body simulations by \cite{NFW}. The NFW profile is one of the most commonly used models in cosmology and it defines the density of dark matter as a function of radius as
\setcounter{equation}{0}
\renewcommand{\theequation}{NFW.\arabic{equation}}
\be
\rho(r)=\frac{\rho_c \delta_c}{(1/r_s)(1+r/r_s)^2} \;,
\ee
where $r_s$ is a scale radius, $\delta_c$ is a characteristic density, and $\rho_c$ is the critical over density defined as $\rho_c=3H^2/8\pi G$. The NFW profile works for a large range of halo masses and sizes, from individual galaxies to the halos of galaxy clusters. {\scshape{\footnotesize see} \gls{Halo Model}}.
}%

%%%%%%%%%%%%%%%%
\dictentry{dictionary}{PanSTARRS}{\PanSTARRS}
\newcommand\PanSTARRS{
The Panoramic Survey Telescope \& Rapid Response System (PanSTARRS)\footnote{\url{http://ps1sc.org/KeyScience.shtml}} is a wide-field imaging facility developed at the University of Hawaii's Institute for Astronomy. PanSTARRS can observe the entire available sky several times each month, meaning surveying the sky for moving objects on a continual basis.  It also has accurate astrometry and photometry of already detected objects. The first Pan-STARRS telescope (PS1) started taking data in May 2010 and was completed in April 2014. As of mid-2014 the second telesceop, PS2, was in the process of being commissioned\footnote{\url{http://pan-starrs.ifa.hawaii.edu/public/}}.
}%

%%%%%%%%%%%%%%%%
\dictentry{dictionary}{PCA}{\PCA}
\newcommand\PCA{
Principal component analysis (PCA) is a statistical procedure that uses an orthogonal transformation to convert a set of observations of possibly correlated variables into a set of values of linearly uncorrelated variables called principal components. The number of principal components is less than or equal to the number of original variables. This transformation is defined in such a way that the first principal component has the largest possible variance (that is, accounts for as much of the variability in the data as possible), and each succeeding component in turn has the highest variance possible under the constraint that it is orthogonal to the preceding components. The resulting vectors are an uncorrelated orthogonal basis set. The principal components are orthogonal because they are the eigenvectors of the covariance matrix, which is symmetric. PCA is sensitive to the relative scaling of the original variables.
PCA can be done by eigenvalue decomposition of a data covariance (or correlation) matrix or singular value decomposition of a data matrix, usually after mean centering (and normalizing or using Z-scores) the data matrix for each attribute.

The covariance/Fisher matrix is a symmetric $n\times n$ matrix and therefore, can be diagonalised
using its eigenvectors. This has the form $\mathbf{C}=\mathbf{E^{T}}\mathbf{\Lambda}\mathbf{E}$,$\;$where
$\mathbf{C}$ is the covariance matrix, $\mathbf{{\normalcolor E}}$
is an orthogonal matrix with the eigenvectors of ${\normalcolor \mathbf{C}}$
as its rows and ${\normalcolor \mathbf{\Lambda}}$ is the diagonal
matrix with the eigenvalues of ${\normalcolor \mathbf{C}}$ as its
diagonal elements%
\footnote{It is common to construct the covariance matrix for PCA. However,
the Fisher matrix can be used instead; the eigenvectors stay the same, but
eigenvalues are reciprocals.%
}. This constructs a new set of variables $\mathbf{{\normalcolor X}}$
that are orthogonal to each other and are a linear combination of
the original parameters ${\normalcolor \mathbf{O}}$, through the eigenvectors
\setcounter{equation}{0}
\renewcommand{\theequation}{PCA.\arabic{equation}}
\be
\mathbf{X}=\mathbf{EO}\;.
\ee
The $X_{i}$ are called the \emph{principal components} of the experiment
and are ordered so that so that $X_{1}$ has the smallest
eigenvalue and $X_{n}$ the largest. In this construction, the eigenvalues
are the variances of the new parameters, so $X_{1}$ and $X_{n}$ are the best- and worst-measured components respectively. The eigenvectors have
been normalised so that $\sum_{j}e_{j}^{2}=1$, where $e_{j}$ are
the elements of $E_{i}$. We list some properties of PCA below:
\begin{enumerate}
\item The main point of PCA is to assess the degeneracies (correlations) amongst the parameters
that are not resolved by the experiments, be they fundamental as from 
cosmic variance or due to the noise and coverage of the experiment.
In our case, it will especially help us to see the correlation amongst
the bins of the primordial PS, and between the bins and the cosmological
parameters.
\item The eigenvalues obtained measure the performance of the experiment
--- a larger number of small eigenvalues means a better experiment.
Another measure of the performance of the experiments is to see how
they mix physically independent parameters such as, say, $n_{s}$,
the spectral index, and $\Omega_{b}$. This sort of mixture may be
improved by improving the experiment's noise properties or increasing its area or volume.%
\footnote{However, the so-called `geometrical degeneracy'
cannot be improved by improving the experiments; two models with
same primordial PS, the same matter content, and the same comoving
distance to the surface of last scattering produce identical CMB PS.%
}
\end{enumerate}
PCA is a special case of Singular Value Decomposition (SVD).
}%

%%%%%%%%%%%%%%%%
\dictentry{dictionary}{Poisson Noise}{\PoissonNoise}
\newcommand\PoissonNoise{
Poisson noise -- sometimes referred to as shot noise -- is a type of noise that can be modelled by a Poisson process, 
which is applied when the event in question can independently be counted in whole numbers. For example, in electronics shot noise is due to the discrete nature of electric charge; in photon counting in optical devices, such as CCD, shot noise arises due to the particle nature of light; in galaxy power spectrum estimation, shot noise arises due to the discrete sampling of the galaxy density field. 

For large numbers, at the point where the events (photons, electrons, etc.) can no longer be individually observed, the Poisson distribution approaches a normal (Gaussian) distribution.
}%

%%%%%%%%%%%%%%%%
\dictentry{dictionary}{Polarisability}{
The weak lensing shear is also referred to as polarisability, see e.g. \cite{1995ApJ...449..460K}.
}%

%%%%%%%%%%%%%%%%
\dictentry{dictionary}{Position}{
The coordinate of object, or data point. In the context of weak lensing 2-point statistics 
this refers to the set of 3D angular and comoving galaxy positions that can be can 
correlated with other properties, for example shear estimates to create a 
shear-position 2-point cross-correlation statistic, or position-position to create a 2-point statistic of the 3D galaxy field.
}%

%%%%%%%%%%%%%%%%
\dictentry{dictionary}{Shapelets}{
Shapelets\footnote{See e.g. \url{http://community.dur.ac.uk/r.j.massey/shapelets/}.} are a complete and orthonormal set of 2D basis functions (like Fourier basis), which can be used to model galaxy image; i.e. galaxy images are decomposed into several shape components, where each of them provides independent estimates of the local shear. Their use in astronomy was first introduced in \cite{2003MNRAS.338...35R,2003MNRAS.338...48R}.
}%

%%%%%%%%%%%%%%%%
\dictentry{dictionary}{NGMIX}{
A method for estimating galaxy shapes for weak lensing. It was first used in DES in \cite{2015arXiv150705598B}. Both the galaxy and the PSF profile are modelled using mixtures of Gaussians for 2D images. Convolutions are analytically dealt with, which makes the method faster than some other methods. For the galaxy model, exponential disks, de Vaucouleurs and S\'ersic profiles are also supported. For more information please refer to \cite{2015ascl.soft08008S} and visit the Github repository \url{https://github.com/esheldon/ngmix}.
}%

%%%%%%%%%%%%%%%%
\dictentry{dictionary}{Potential Map}{
An inferred map of the lensing potential, derived using a kappa map. 
{\scshape{\footnotesize see} \gls{Mass Map}; \gls{3D Mass Map}; \gls{Kappa}}.
}%

%%%%%%%%%%%%%%%%
\dictentry{dictionary}{Convergence Map}{
{\scshape{\footnotesize see} \gls{Mass Map}; \gls{3D Mass Map}; \gls{Kappa}; \gls{Potential Map}}.
}%

%%%%%%%%%%%%%%%%
\dictentry{dictionary}{Power Spectrum}{\PowerSpectrum}
\newcommand\PowerSpectrum{
The power spectrum is the Fourier counterpart of the two-point correlation function.
{\scshape{\footnotesize see} \gls{Correlation Functions and Power Spectra}}.

The power can also be in the harmonic space. Given a harmonic transform of a quantity, the power spectrum is the average of the quadrature of the harmonic transform coefficients. An example in weak lensing is the spherical harmonic transform of the shear field $\gamma(\theta,z)$ 
on a redshift slice $z$ 
\setcounter{equation}{0}
\renewcommand{\theequation}{PS.\arabic{equation}}
\be
\gamma^m_{\ell}(z)=\sum_{g\in z}\gamma(\theta,z)_2Y^m_{\ell}(\theta)\;,
\ee 
that results in a power spectrum 
\be
\langle \gamma^m_{\ell}(z)\gamma^{m'*}_{\ell'}(z')\rangle=C_{\ell}(z,z')\delta_{\ell\ell'}\delta_{mm'}\;,
\ee
where $C_{\ell}(z,z')$ is the power spectrum that assumes homogeneity and isotropy. 
{\scshape{\footnotesize see} \gls{Shear Power Spectrum}}.
}%

%%%%%%%%%%%%%%%%
\dictentry{dictionary}{Shear Power Spectrum}{\ShearPowerSpectrum}
\newcommand\ShearPowerSpectrum{
Tomographic cosmic shear power spectrum is the auto- and cross-correlation of shear of galaxies at different redshift bins
\setcounter{equation}{0}
\renewcommand{\theequation}{SP.\arabic{equation}}
\be
C_{ij}(\ell)=\int^{r_H}_{0}\textrm{d}r \; W_{ij}^{GG}(r) \; P_{\delta\delta}\left(k=\frac{\ell}{S_k(r)};r\right)\:,
\ee
%
where $P_{\delta\delta}\left(k=\ell/S_k(r);r\right)$ is the 3D density matter power spectrum, with $S_k(r)$ being the comoving angular diameter distance and $r$ being the comoving distance. The lensing weight function $W_{ij}^{GG}(r)$ is expressed as
\be
W_{ij}^{GG}(r)=\frac{q_i(r)q_j(r)}{S^2_k(r)}\:,
\ee
with kernel  
\be
q_i(r)=\frac{3H_0^2\Omega_m S_k(r)}{2a(r)} \int^{r_H}_r \textrm{d}r^\prime \; p_i(r^\prime) \; \frac{S_k(r^\prime-r)}{S_k(r^\prime)}\;,
\ee
where $ij$ subscripts refer to redshift bins, $r_H$ is the horizon distance and $a$ is the scale factor. The comoving source galaxy probability distribution $p_i(r)$ is given by $p_i(z)\propto z^2 \exp(-1.4z/z_m)^{1.5}$, where $z_m$ is the median redshift of the survey and $p_i(r)dr = p_i(z)dz$. {\scshape{\footnotesize see} \gls{Power Spectrum}}.
}%

%%%%%%%%%%%%%%%%
\dictentry{dictionary}{Precision}{
Precision refers to a statistical error of a measurement. It refers to how close a measurement is to the `true' value (that would have been inferred given no systematic effects and 
no noise in the data). In terms of confidence limits on parameters a precise measurement is one where the inferred limits on 
parameters are statistically consistent with the true values of the parameters. The opposite is an imprecise measurement. 
Also the inverse of a covariance matrix is also known as a precision matrix \citep{2013MNRAS.432.1928T}.
}%

%%%%%%%%%%%%%%%%
\dictentry{dictionary}{Projected Density}{
The projected density refers to the integrated mass density along a line of sight.
}%

%%%%%%%%%%%%%%%%
\dictentry{dictionary}{PSF}{\PSF}
\newcommand\PSF{
Acronym for Point Spread Function of telescope, which describes the response of the telescope to a point source {\scshape{\footnotesize see} \gls{Impulse Response Function (of telescope)}}. PSF is a convolutive function, which distorts sizes and shapes of galaxies. The observed source has a larger size of 
\setcounter{equation}{0}
\renewcommand{\theequation}{PSF.\arabic{equation}}
\be
R^2_{\rm{obs}} = R^2_{\rm{gal}} + R^2_{\rm{PSF}}\;,
\ee
and a perturbed ellipticity given by 
\be 
\epsilon_{\rm{obs}} = \epsilon_{\rm{gal}} + \frac{R^2_{\rm{PSF}}}{R^2_{\rm{gal}}+R^2_{\rm{PSF}}} \left( \epsilon_{\rm{PSF}} - \epsilon_{\rm{gal}}\right)\;.
\ee
Galaxy shapes have to be deconvolved for the effects of PSF, using point objects, such as stellar images. 
\cite{2013ApJS..205...12K} review several methods for estimating the PSF from images for weak lensing.
}%

%%%%%%%%%%%%%%%%
\dictentry{dictionary}{q}{
{\scshape{\footnotesize see} \gls{Multiplicative and Additive Bias of Shear}}.
}%

%%%%%%%%%%%%%%%%
\dictentry{dictionary}{QE}{
Acronym for Quantum Efficiency; {\scshape{\footnotesize see} \gls{Quantum Efficiency}}.
}%

%%%%%%%%%%%%%%%%
\dictentry{dictionary}{Quantum Efficiency}{
Quantum efficiency (QE) measures the effectiveness of a CCD to produce electronic charge when hit by incident photons; the percentage of photons that are actually detected (due to photoelectrons being produced) is known as the Quantum Efficiency (QE). For example, the human eye only has $\rm{QE}\simeq20\%$, a photographic film has $\rm{QE}\simeq10\%$. QE is wavelength dependent and it directly effects the $S/N$ of the CCD. Hence a CCD is generally chosen which has the highest QE in the desired wavelength range. The key factor in determining the QE of a CCD is the actual structure of the CCD. In Euclid CCDs, the QE range from $\rm{QE}\simeq50\%$ at wavelength $\lambda=900$ to $\rm{QE}\simeq90\%$ at wavelength $\lambda=750$.
}%

%%%%%%%%%%%%%%%%
\dictentry{dictionary}{Pseudo-Cl}{\PseudoCl}
\newcommand\PseudoCl{
The shear field can be expanded in terms of spherical harmonic functions $Y_{\ell m}$ as
\setcounter{equation}{0}
\renewcommand{\theequation}{PCL.\arabic{equation}}
\be
\gamma = \sum\limits_{\ell=0}^{\infty} \sum\limits_{m=-\ell}^{\ell} a_{\ell m} Y_{\ell m} \;,
\ee
with $a_{\ell m}$ being the spherical harmonic coefficients. For a Gaussian $\gamma$ with zero mean, $\langle a_{\ell m} \rangle = 0$, the power spectrum is the variance
\be
\langle a_{\ell m} a^*_{\ell^{\prime} m^{\prime}} \rangle = \delta_{\ell \ell^{\prime}} \delta_{mm^{\prime}} C_\ell^{\textrm{th}} \;,
\label{eq:ps}
\ee
where $C_\ell^{\textrm{th}}$ is the shear angular power spectrum. We only observe a realisation of this underlying power spectrum on our sky, which we can estimate using the \textit{empirical power spectrum estimator} defined as
\be
\widehat{C}^{\mathrm{th}}_\ell = \frac{1}{2\ell + 1} \sum\limits_{m=-\ell}^{\ell} | a_{\ell m} |^2\;,
\label{eq:empirical-ps}
\ee
where $\widehat{C}^{\mathrm{th}}_\ell$ is an unbiased estimator of the true underlying power spectrum; $\langle \widehat{C}^{\mathrm{th}}_\ell\rangle = C_\ell^{\textrm{th}}$, in the case of noiseless data over full sky. Applying a mask on the sky results in the following modification of the spherical harmonic coefficients:
\begin{equation}
	\tilde{a}_{\ell m} = \int {\rm d} \, \Omega\gamma(\Omega) W(\Omega) Y_{\ell m}^*(\Omega)\;,
\end{equation}
where $W(\Gamma)$ is the window function applied to the data. The presence of the window function induces correlations between the $a_{\ell m}$ coefficients at different $\ell$ and different $m$. One can define the \textit{pseudo power spectrum} $\widetilde{C}_{\ell}$ as the application of the empirical power spectrum estimator on the spherical harmonic coefficients of the masked sky. In case of data contaminated with additive Gaussian stationary noise, the pseudo power spectrum is
\be
\widetilde{C}_\ell = \frac{1}{ 2 \ell + 1 } \sum_{m = -\ell}^{\ell} | \tilde{a}_{\ell m} + \tilde{n}_{\ell m} |^2 \;,
\ee
where $\tilde{n}_{\ell m}$ are the spherical harmonic coefficients of the masked instrumental noise. 

Following the MASTER method from \citet{MASTER}, the pseudo power spectrum $\widetilde{C}_\ell$ and the empirical power spectrum $\widehat{C}^{\mathrm{th}}_\ell$ can be related through their ensemble averages:
\begin{equation}
\langle \widetilde{C}_\ell \rangle = \sum_{\ell^\prime} M_{\ell \ell^\prime}  {C}^{\mathrm{th}}_{\ell^\prime} +  \langle \widetilde{N}_\ell \rangle \;,
\end{equation}
where $M_{\ell \ell^\prime}$ describes the mode-mode coupling between modes $\ell$ and $\ell^\prime$ resulting from computing the transform on the masked sky.
{\scshape{\footnotesize see} \gls{Power Spectrum}}.
}%

%%%%%%%%%%%%%%%%
\dictentry{dictionary}{K-correction}{
This is a correction to the magnitude or flux of a source to convert it to its rest frame magnitude or flux. The K-correction can be defined as 
\setcounter{equation}{0}
\renewcommand{\theequation}{KC.\arabic{equation}}
\be
M = m - 5 \left( {\rm log}_{10} D_L -1  \right) -K_{\rm cor} \;,
\ee
where $D_L$ is the luminosity distance.  
K-correction is necessary when objects are observed through a single bandpass, as only a fraction of the total spectrum of the source is observed.
}%

%%%%%%%%%%%%%%%%
\dictentry{dictionary}{Vignetting}{
This is the spatial variation of the transmission of an optical system, generally (but not necessarily) with a reduction towards the perimeter, so that images appear darker in those regions. Vignetting may be caused by inadequate sizes of optical elements 
intermediate in the optical train, so that they do not intercept the full bundle of rays for objects far from centre of the 
field-of-view. Alternatively it could be the consequence of mechanical structures such as optical supports and baffles in 
the ray path blocking the rays. Finally, as the angle from the field centre increases the projected area of the optical system 
to the ray bundle decreases, but this is a small effect in most astronomical optical systems.
}%

%%%%%%%%%%%%%%%%
\dictentry{dictionary}{Impulse Response Function (of telescope)}{
The response of an optical system when observing a point-like object at a distance of infinity. 
A general term for the Fourier transform of the point spread function (PSF). 
{\scshape{\footnotesize see} \gls{PSF}}.
}%

%%%%%%%%%%%%%%%%
\dictentry{dictionary}{Footprint (of survey)}{
The specific area of the sky that is observed by a survey. See e.g. 
\url{http://lambda.gsfc.nasa.gov/toolbox/footprint/aladin/aladinLAMBDA.cfm} to see the footprint 
of some of the cosmological surveys such as DECam, HSC and BOSS.
}%

%%%%%%%%%%%%%%%%
\dictentry{dictionary}{Slew}{\slew}
\newcommand\slew{A slew is the rotation of a spacecraft from one pointing direction to another. Generally this 
is a relatively large rotation with acceleration, constant speed and deceleration phases. In the weak lensing 
this may refer to HST or Euclid missions. For Euclid this refers to a rotation of $0.7$ times the field-of-view, i.e. approximately the field of view of the instruments or more, and it entails the reaquisition of the knowledge of the pointing direction at its termination using an on-satellite star catalog. 

A dither is a smaller rotation, with pointing directions within-field, and sometimes as small as arc seconds. Knowledge of the pointing direction is generally maintained during the dither. Dithers are used to maximise the utility of multiple exposures, by moving the image with respect to the detector pixel grids, to recover spatial resolution in slightly under-sampled images. In the Euclid context the multiple exposures themselves (as opposed to the re-pointings) are often also called dithers, but the meaning is generally clear from the context. {\scshape{\footnotesize see} \gls{Dither}}. 
}%

%%%%%%%%%%%%%%%%
\dictentry{dictionary}{Scattered Light (Gegenschein)}{
This is the sunlight scattered by interplanetary dust in the ecliptic plane of the Solar system, 
in the anti-solar direction --- mostly at the L2 Earth-Sun Lagrangian point, due to the concentration of particles at L2. 
It forms an oval-shaped glow directly opposite the Sun within the band of zodiacal light. This is strong enough to be taken into account in the observations of surveys. 
{\scshape{\footnotesize see} \gls{Zodiacal Light}}.
}%

%%%%%%%%%%%%%%%%
\dictentry{dictionary}{Zodiacal Light}{
This is the sunlight scattered by interplanetary dust in the zodiacal cloud in the Solar system, extending up from the vicinity of the Sun along the ecliptic/zodiac. Its intensity covers the whole sky, but decreases with distance from the Sun. This is strong enough to be taken into account in the observations of surveys. {\scshape{\footnotesize see} \gls{Scattered Light (Gegenschein)}}.
}%

%%%%%%%%%%%%%%%%
\dictentry{dictionary}{Brighter-Fatter Effect}{\bfe}
\newcommand\bfe{This is an effect in which the profile of point
  sources ({\scshape{\footnotesize see} PSF}) as recorded on the detector is flattened 
and broadened to an extent depending on the flux recorded by a CCD detector 
(the effect is negligible in HeCdTe detectors used in the infrared). 

Within a CCD pixel, as absorbed photons with sufficient energy elevate
electrons into the conduction band of the Silicon, they collect in the
potential well created by the voltage applied to the electrodes. In
back-illuminated CCDs such as the CCD273 in Euclid, photons are
absorbed in a region between the back surface and the electrodes (the pixel boundaries are set by the field structure extending from the electrodes to the back surface). The accumulation of charge changes the
field structure, with the consequence that it modifies the effective
boundaries of the pixel. If its neighbouring pixels are collecting
fewer photons, this effectively reduces its size as the charge
accumulates. When remapped to a regular grid in which all pixel areas
are the same, as in an image, this pixel appears to be slightly
deficient in charge, and the neighbours appear to have excess
charge. The point-source profile is therefore flattened at the peak
and broadened, with a dependence on the overall photon flux captured
in the exposure.

The impact of this effect on weak lensing has been studied in \cite{2015ExA....39..207N,2015JInst..10C5032G,2015JInst..10C5015W}.
}%

%%%%%%%%%%%%%%%%
\dictentry{dictionary}{AOCS}{
An Attitude and Orientation Control System (AOCS) is a system in a satellite which controls the satellite pointing direction, the stability of this pointing, and the rotation of the satellite to point in a different direction. The AOCS is critical in ensuring the safety of the satellite consequent to an equipment failure or commanding error. The system contains sensors, such as Sun sensor, star trackers, fine guidance sensors and gyroscopes, and actuators such as reaction wheels, magnetorquers and thrusters of various powers and controllability. The AOCS operational lifetime is generally limited by consumables, such as hydrazine; {\scshape{\footnotesize see} \gls{Slew}}.
}%

%%%%%%%%%%%%%%%%
\dictentry{dictionary}{Cosmetic Maps}{
Cosmetic maps are maps of the imperfections in the detectors. These may be ``dead pixels'' with no sensitivity, or 
pixels which always have a large charge in them whatever the incident flux (known as ``hot pixels''). They may record the 
presence of dust particles. In some detectors, such as CCDs, the defect can extend to columns of pixels. 
The maps are used to flag these pixels for the subsequent data processing, at which time their information 
content is discarded.
{\scshape{\footnotesize see} \gls{Flat-Field (FF)}}.
}%

%%%%%%%%%%%%%%%%
\dictentry{dictionary}{Pocket Pumping}{\ppp}
\newcommand\ppp{Pocket pumping is a technique which is used to identify damage in the Si lattice in CCD detectors. Ions, generally Solar protons, but also heavier ions from the Sun and of cosmic origin, incident on the detector may displace atoms in the lattice, especially in areas where there impurities and the lattice is less elastic. These regions are called traps, because as the CCD is being read out at the end of an exposure, electrons in the conduction band may be trapped in the local irregularity. Several species of trap exist, with different trapping and release time constants. Traps with longer release time constants than the readout rate may release their trapped electrons into a subsequent pixel, giving rise to a trail behind the image. This directly modifies the shape of the image, which is important for measuring its shape, and must be corrected for weak gravitational lensing measurements. When pocket pumping, the charge is shuffled backwards and forwards without being read out during the exposure. If the shuffling frequency is resonant with the trap release time, charges in a pixel will be trapped and released in the adjacent pixel, so that an initially uniform image develops bright-faint pairs. These pairs identify the location of each trap with that associated trap release time constant. The shuffling frequency can then be changed to explore another species of trap in the next pocket pumping exposure. The initial uniform image can be generated by the artificial injection of charge 
or by optical illumination, for example in the case of Euclid, by a calibration source.
}%

%%%%%%%%%%%%%%%%
\dictentry{dictionary}{Pixel Clamping Bounce}{
Pixel clamping bounce is the difference between the measured and expected signal in a detector pixel after a sharp transition. Generally this refers to an overshoot, so that the first few faint pixels after bright pixels are measured to be fainter than they in reality are, or the first few bright pixels after faint pixels are measured to be brighter. The effect can arise in various parts of the detection chain electronics, including within the detector and from parasitic capacitance in the circuits connecting them to the electronics.
}%

%%%%%%%%%%%%%%%%
\dictentry{dictionary}{Cadence}{
The cadence is the frequency of sampling of the data. In a space-based
observatory, this is approximately the time interval between spacecraft 
slews divided by the number of dithers; {\scshape{\footnotesize see} \gls{Slew}}.
}%

%%%%%%%%%%%%%%%%
\dictentry{dictionary}{Flat-Field (FF)}{
A flat field (FF) is the response of a telescope to a uniform light across its area. Hence a flat field frame is an image of a field that has a uniform illumination. Flat field correction (flat fielding) is then applied to astronomical images to correct for artefacts in the optical path of the telescopes; these include inter-pixel responsivity in the CCD, vignetting etc. 
Flat fielding is particularly needed when photometry is required. See e.g. \cite{2013MNRAS.433.2545E} for an application of flat-fielding in a data analysis chain used for weak lensing.
}%

%%%%%%%%%%%%%%%%
\dictentry{dictionary}{Friends of Friends}{
Friends of Friends (FoF) is one of the many halo finder techniques used in cosmological simulations and for finding clusters 
in data. It uses particle positions to group spatially close particles -- $\Delta x < b$, where $b$ is called the linking parameter -- 
to detect halos, sub-haloes in the simulation. See e.g. \cite{2010ApJ...709..286M,2011A&A...531A.169P,2015ApJ...806....2M}.
}%

%%%%%%%%%%%%%%%%
\dictentry{dictionary}{Tree Rings}{
Tree rings are an effect in CCDs (resulting from the manufacturing of CCDS) that cause photoelectrons to slightly shift from 
the position the photon hit the detector to the point the electron was registered. 
This results in chip-position-dependant systematic bias, which need to be corrected for. See e.g. \cite{2014JInst...9C4001P}.
}%

%%%%%%%%%%%%%%%%
\dictentry{dictionary}{Cross-component Shear}{
Cross-component shear (or ellipticity) is the amplitude of 
the shear which at $45^{\degree}$ to the tangential shear (which is that which is perpendicularly aligned 
with a lensing source) {\scshape{\footnotesize see} \gls{Tangential Shear}}.}%

%%%%%%%%%%%%%%%%
\dictentry{dictionary}{Density Parameters}{
Density parameters in cosmology are the ratio of the density of component $i$ to the critical density $\rho_c$, i.e. the density at which the Universe expands asymptotically; $\Omega_i = \rho/\rho_c$. This could be cold dark matter $\Omega_c$, baryonic matter $\Omega_b$, cosmological constant $\Omega_\Lambda$ and etc. 
{\scshape{\footnotesize see} \gls{Friedman--Robertson--Walker Models}}.
}%

%%%%%%%%%%%%%%%%
\dictentry{dictionary}{Standard Cosmological Model}{
{\scshape{\footnotesize see} \gls{LCDM}}.
}%

%%%%%%%%%%%%%%%%
\dictentry{dictionary}{R200 or R500}{
$R_{200}$ or $R_{500}$ are the radii at which the enclosed mean density of a cluster is either 200 or 500 times the 
critical density of the Universe.
}%

%%%%%%%%%%%%%%%%
\dictentry{dictionary}{M200 or M500}{
The mass enclosed within radius $R_{200}$ or $R_{500}$. 
{\scshape{\footnotesize see} \gls{R200 or R500}}.
}%

%%%%%%%%%%%%%%%%
\dictentry{dictionary}{Gaia}{
Gaia\footnote{\url{http://sci.esa.int/gaia/}} is an ESA mission and the successor to the Hipparcos mission. It is an all-sky astrometric survey telescope with the aim of constructing a 3D map of one billion astronomical objects in the Milky Way (including stars, planets, comets, asteroids, and etc.), providing information about their motion, luminosity, effective temperature, gravity and elemental composition. This will provide information about the origin, structure and evolution of the Milky Way. Gaia was launched in December 2013 and currently operates around the L2 Lagrangian point. The expected completion date of 2021.
}%

%%%%%%%%%%%%%%%%
\dictentry{dictionary}{SED}{
The Spectral Energy Distribution (SED) is the energy (or wavelength) spectrum of a given astronomical object.
}%

%%%%%%%%%%%%%%%%
\dictentry{dictionary}{Cosmic Rays}{
Cosmic Rays (CRs) are high-energy particles (mainly protons and atomic nuclei) originating from outside the Solar System, but of unknown origin. They can saturate pixels in CCDs and even damage the lattice structure of the CCDs.
}%



%%%%%%%%%%%%%%%%
\dictentry{dictionary}{Probability Distribution}{\ProbabilityDistribution}%
\newcommand\ProbabilityDistribution{
Probability distribution or probability distribution function (PDF). Here we list some of the commonly used PDFs in weak lensing. 
\begin{enumerate}
%%%%%%%%
\item Gaussian (Normal) Distribution

This is the most common continuous probability distribution in cosmology. It is very useful because of the central limit theorem; physical quantities that are sum of many independent processes (such as measurement errors) often have nearly normal distributions. For data $x$, normal distribution is defined as 
\setcounter{equation}{0}
\renewcommand{\theequation}{PD.\arabic{equation}}
\be
\mathcal{N}\left( x;\mu,\sigma^2 \right)=\left(\frac{1}{2\pi\sigma^{2}}\right)^{1/2}\exp\left[ -\frac{1}{2}\frac{\left(x-\mu\right)^{2}}  {\sigma^{2}}\right] \;,
\ee
where `standard' normal distribution has $\mu=0$ and $\sigma^{2}=1$. For a collection of data $\left\{x_{n}\right\} $
\be
\mathcal{N}\left(x_n;\mu,\sigma^2 \right)=\left(\frac{1}{2\pi\sigma^{2}}\right)^{N/2}\exp\left[ -\frac{1}{2}\frac{\sum_{i}^{N}\left(x_{i}-\mu\right)^{2}}{\sigma^{2}}\right] \;.
\ee
For multi-variate case 
\be
\mathcal{N}\left(\mathbf{D}_{n};\mu_n,\mathbf{C} \right)=\left(\frac{1}{\left|2\pi \mathbf{C}\right|}\right)^{1/2}\exp\left[ -\frac{1}{2}\left(\mathbf{D}_{n}-\mu_n\right)\mathbf{C}^{-1}\left(\mathbf{D}_{n}-\mu_n\right)^{T}\right] \;,
\ee
where $C=\left\langle \left(\mathbf{D}_{n}-\mu_n\right)\left(\mathbf{D}_{n}-\mu_n\right)^{T}\right\rangle$.
\\
%%%%%%%%
\item Log-normal Distribution

This is a continuous probability distribution of a random variable whose logarithm is normally distributed --- a log-normally distributed random variable can only take positive real values. For data $x$, where $\log(x)$ has normal distribution
\be
\ln \mathcal{N} \left(x;\mu,\sigma^2 \right)=\left(\frac{1}{2\pi\sigma^{2}}x^2 \right)^{1/2}\exp\left[ -\frac{1}{2}\frac{\left(\ln x-\mu\right)^{2}}  {\sigma^{2}}\right] \;\;\;\;\;\; {\rm for} \;\; x>0 \;.
\ee
This is useful in weak lensing because the matter overdensity field is hypothesised to have an approximately log-normal distribution 
\citep{1991MNRAS.248....1C,2001ApJ...561...22K,2016MNRAS.459.3693X}.
\\
%%%%%%%%
\item $\chi^2$ Distribution 

For $x_{n}$ is drawn from `standard' normal distribution, then $X=\sum_{n}^{N}x_{n}^{2}$ has a $\chi^{2}$ distribution with $N$ degrees of freedom
\be
\chi_{N}^{2}\left(X\right)=\left(\frac{1}{2^{N}\Gamma^{2}\left(N/2\right)}\right)^{1/2}X^{N/2-1}\exp\left[ -\frac{1}{2}X\right] \ ,\label{eq:chi2_dist}
\ee
where $\Gamma(N/2)$ is Gamma function. If $x_{n}$ has mean $\mu$ and variance $\sigma^{2}$ then $X=\frac{1}{\sigma^{2}}\sum_{n}^{N}\left(x_{n}-\mu\right)^2$ has a $\chi^{2}$ distribution with $\left(N-1\right)$ degrees of freedom.
\\
%%%%%%%%
\item Poisson Distribution 

This probability distribution describes the distribution of a discrete independent point process
\be
\mathcal{P}_n = \frac{\lambda^n}{n} \exp{(-\lambda)} \; ,
\ee
where $\lambda$ is the average number of events per interval (e.g. in a volume of space), $n$ is the event (e.g. a galaxy). 
This is useful in weak lensing and cosmology because many processes have Poisson distribution for example 
photon noise. 
\\
%%%%%%%%
\item Wishart Distribution

For $x$ being an $N\times p$ matrix, with each row being independently drawn from a $p$-variate normal distribution with mean $\mu=0$, 
the $p\times p$ random scatter matrix $\mathbf{S}=X^TX$ has a Wishart distribution with $N$ degrees of freedom and covariance $\mathbf{C}$
\be
\mathcal{W}_p \left(\mathbf{C},N\right) = \frac{1}{2^{Np/2} \left|\mathbf{C}\right|^{N/2} \Gamma_p \left(N/2\right)}
\left|S\right|^{(N-p-1)/2}
\exp\left[-\frac{1}{2}\textrm{tr}\left(\mathbf{C}^{-1}\mathbf{S}\right)\right]\; ,
\label{eq:Wishart_dist}
\ee
where $\Gamma_p$ is the multivariate Gamma function. 
In cosmology and weak lensing, this distribution is useful because parameters covariance matrices have Wishart distributions.
 The Wishart distribution is the multivariate extension of the gamma distribution --- in case of integer degrees of freedom, 
it is a multivariate generalisation of the $\chi^2$distribution; i.e. for a single data point ($p=1$), 
the Wishart distribution is the reduced $\chi^2$-distribution. As the $\chi^2$distribution describes the sums 
of squares of $N$ draws from a univariate normal distribution, the Wishart distribution represents the sums 
of squares (and cross-products) of $N$ draws from a multivariate normal distribution. For examples of the use of this distribution in 
weak lensing see \cite{2013MNRAS.432.1928T,2014IAUS..306...99J}. 
\\
%%%%%%%%
\item Inverse Wishart Distribution

For $x$ being an $N\times p$ matrix, with each row being independently drawn from a $p$-variate normal distribution with mean $\mu=0$, 
the $p\times p$ random scatter matrix $\mathbf{S}=X^TX$ (note that this is the inverse of $\mathbf{S}$ in Wishart distribution) has a inverse Wishart distribution with $N$ degrees of freedom and covariance $\mathbf{C}$
\be
\mathcal{W}^{-1}_p\left(\mathbf{C},N\right) = \frac{1}{2^{Np/2}\left|\mathbf{S}\right|^{N+p+1/2}\Gamma_p \left(N/2 \right)}
\left| \mathbf{C} \right|^{N/2}
\exp \left[-\frac{1}{2}\textrm{tr}\left(\mathbf{C}\mathbf{S}^{-1}\right)\right]\; ,
\label{eq:Wishart_dist}
\ee
where $\Gamma_p$ is the multivariate Gamma function. 
In cosmology, they are useful as normally parameters precision matrices (inverse of covariance matrices) have inverse Wishart distributions. 
The Inverse-Wishart distribution is the multivariate extension of the inverse-gamma distribution --- in case of integer degrees of 
freedom, it is a multivariate generalisation of the inverse $\chi^2-$distribution; i.e. for a single data point ($p=1$), the 
inverse Wishart distribution is the inverse $\chi^2$ distribution. For examples of the use of this distribution in
weak lensing see \cite{2013MNRAS.432.1928T,2014IAUS..306...99J}
\\
%%%%%%%%
\item Cauchy (Lorentz) Distribution
Far data $x$, Cauchy distribution is defined as 
\be
f(x;x_{0},\gamma )={1 \over \pi \gamma }\left[{\gamma ^{2} \over (x-x_{0})^{2}+\gamma ^{2}}\right]\;,
\ee
where $x_0$ specifies the location of the peak of the distribution and $\gamma$ is the HWHM (half width half maximum). 
This distribution describes the resonance behaviour and the distribution of horizontal distances at which a line segment 
tilted at a random angle cuts the $x$-axis.
\\
%%%%%%%%
\item Marsaglia-Tin Distribution

A probability density function (PDF) for ratios of two random Gaussian variables with arbitrary means and correlation. 
This was first discussed in \cite{Marsaglia:1965:RNV} and \cite{10.2307/2283154}. This distribution is encountered 
in weak lensing, as estimating the ellipticity of a galaxy involves a ratio of  
such variables. For example, the ellipticity can be estimated using the semi-major and semi-minor axis of the galaxy shape 
$\epsilon = [(a-b)/(a + b)] \exp(2i\phi)$. The Marsaglia-Tin distribution was further generalised 
by \cite{Viola01042014} to the case of two ratios constructed from three correlated random variables, which is relevant 
for ellipticity measurements.
\end{enumerate}}%

%%%%%%%%%%%%%%%%
\dictentry{dictionary}{Weight Function}{\wf}
\newcommand\wf{
A weight function is a mathematical device used when performing a sum, integral, or average to give 
some elements more weight or influence on the result than other elements in the same set. 
The result of this application of a weight function is a weighted sum or weighted average. 
In weak lensing commonly specifically refers to the weight function used in a moment-based measurement of 
ellipticity; {\scshape{\footnotesize see} Quadrupole Moments}.
}% 

%%%%%%%%%%%%%%%%
\dictentry{dictionary}{Ellipticity}{\Ellipticity}
\newcommand\Ellipticity{
Ellipticity defines the 2D shape of an object, hence it is a spin-2 quantity which is usually defined as a complex quantity. The symbol $e$ is generically used, which could be ellipticity, polarisation or shear. In a two-dimensional (flat-sky) Cartesian coordinate system the real (imaginary) part of $e$ can be identified with the first (second) coordinate in that system, usually written as $e=e_1+{\rm i}e_2$; where $e_1$ is called {\it Ellipticity-1} and $e_2$ is called {\it Ellipticity-2}.

\setcounter{equation}{0}
\renewcommand{\theequation}{EL.\arabic{equation}}
The ellipticity of an object in an image can be defined in several ways, which can be based on 
\begin{enumerate}
\item Quardrupole moments $Q_{ij}$ ({\scshape{\footnotesize see} Quadrupole Moments});
\begin{enumerate}
\item Third Eccentricity \index{Third Eccentricity}
\be
\mathbf{\chi} \equiv \chi_1 + i\chi_2 \equiv \frac{Q_{11} - Q_{22} + 2iQ_{12}}{Q_{11} + Q_{22}} \;.
\ee
\item Third Flattening \index{Third Flattening}
\be
\mathbf{\epsilon} \equiv \epsilon_1 + i\epsilon_2 \equiv \frac{Q_{11} - Q_{22} + 2iQ_{12}}{Q_{11} + Q_{22}+2(Q_{11}Q_{22}-Q_{12}^2)^{1/2}} \;.
\ee
\end{enumerate}
\item Semi-major $a$ and semi-minor $b$ axes of the ellipse; 
\begin{enumerate}
\item Third Eccentricity
\be
\left| \chi \right| \equiv  \frac{a^2 - b^2}{a^2 + b^2}\;.
\ee
\item Third Flattening 
\be
\left| \epsilon \right| \equiv \frac{a - b}{a + b}\;.
\ee
This definition of the third flattening is sometimes referred to as {\it $\epsilon$-ellipticity}.
\end{enumerate}
\end{enumerate}
Please refer to \cite{Viola01042014} for more discussions. 
}%

%%%%%%%%%%%%%%%%
\dictentry{dictionary}{Ellipticity (observed)}{\EO}
\newcommand\EO{
There are different ways to measure the ellipticity of a galaxy/star in an image. Based on $2^{\rm nd}$ 
order moments $Q_{ij}$ ({\scshape{\footnotesize see} \gls{Quadrupole Moments}}) we can define 
\setcounter{equation}{0}
\renewcommand{\theequation}{EO.\arabic{equation}}
\begin{enumerate}
\item third eccentricity 
\be
\mathbf{\chi} \equiv \frac{Q_{11} - Q_{22} + 2iQ_{12}}{Q_{11} + Q_{22}} \;,
\ee
{\scshape{\footnotesize see} \gls{Third Eccentricity}} for this definition in terms of semi-major and semi-minor axes. 
\item third flattening 
\be
\mathbf{\epsilon} \equiv \frac{Q_{11} - Q_{22} + 2iQ_{12}}{Q_{11} + Q_{22}+2(Q_{11}Q_{22}-Q_{12}^2)^{1/2}} \;.
\ee
{\scshape{\footnotesize see} \gls{Third Flattening}} for this definition in terms of semi-major and semi-minor axes. 
\end{enumerate}
Both of these are measurements of ellipticity. 
In the case of weak lensing, when we compute the average ellipticity over an ensemble of galaxies whose orientations 
are assumed to be random, we find that:
\be
\left\langle\epsilon\right\rangle \simeq \frac{\left\langle\chi\right\rangle}{2}\simeq g \simeq \gamma\;,
\ee
where $g=\gamma/(1-\kappa)$ is the reduced shear. Therefore, either definition of ellipticity, 
under the assumptions outlined above, is taken to be an unbiased noisy estimate 
of the gravitational shear at the location of the galaxy. 
In practice, the observed, measured ellipticity will be convolved with a telescope (and possibly atmospheric) 
point spread function (PSF) which convolves the image and has the effect of smearing the galaxy images. 
This impacts the measured ellipticity:
\be
\chi^{o} = \frac{\chi + T\chi^{\rm PSF}}{1+T}\;,
\ee
where $\chi$ is the lensed galaxy ellipticity prior to convolution with the PSF, $\chi^{\rm PSF} $ 
is the ellipticity of the point spread function, $\chi^{o}$ is the measured, observed ellipticity of the galaxy after convolution with the PSF and 
\be
T = \frac{P_{11} + P_{22}}{Q_{11}+Q_{22}}.
\ee
In the above equation, $P_{ij}$ refers to the 2nd order moments of the PSF and $Q_{ij}$ refers to 
the 2nd order moments of the galaxy's brightness distribution, as before.
}%

%%%%%%%%%%%%%%%%
\dictentry{dictionary}{Epsilon-ellipticity}{
{\scshape{\footnotesize see} \gls{Ellipticity}}.
}%

%%%%%%%%%%%%%%%%
\dictentry{dictionary}{Ellipticity 1 and Ellipticity 2}{
{\scshape{\footnotesize see} \gls{Ellipticity}}.
}%

%%%%%%%%%%%%%%%%
\dictentry{dictionary}{IMCAT}{
The Image and Catalogue (IMCAT)\footnote{\url{http://www.ifa.hawaii.edu/faculty/kaiser/imcat/content.html}} manipulation 
software was developed for faint galaxy photometry for weak lensing studies \citep{2011ascl.soft08001K}.
}%

%%%%%%%%%%%%%%%%
\dictentry{dictionary}{GREAT}{\GREAT}
\newcommand\GREAT{
Gravitational lEnsing Accuracy Testing challenges are a series of challenges, with a goal of testing and facilitating the development of methods within the lensing community for analysing astronomical data in a blind way to find the best methods to measure weak gravitational lensing. So far there has been GREAT08, GREAT10 and GREAT3 each with different aims and challenges;
\begin{itemize}
\item The GREAT8 Challenge set a highly simplified version of the problem, using known PSFs, simple galaxy models, 
and a constant applied gravitational shear \citep{2009AnApS...3....6B,2010MNRAS.405.2044B}.
\item The GREAT10 Challenge increased the realism and complexity of its simulations over GREAT08 
by using cosmologically-varying shear fields and greater variation in galaxy model parameters and 
telescope observing conditions. Since imperfect knowledge of the PSF can also bias shear measurements, 
GREAT10 tested also PSF modelling in a standalone Star Challenge \citep{2010arXiv1009.0779K,2012MNRAS.423.3163K,2013ApJS..205...12K}.
\item The GREAT3\footnote{\url{http://www.great3challenge.info}} 
challenge included effects realistically complex galaxy models based on high-resolution imaging from space; 
spatially varying and physically-motivated blurring kernel; and combination of multiple different 
exposures \citep{2014ApJS..212....5M,2015MNRAS.450.2963M}. 
\end{itemize}
GREAT08 and GREAT10 were preceded by a number of internal challenges within the astrophysics community, 
known as the Shear Testing Programme, or STEP ({\scshape{\footnotesize see} \gls{STEP}}).
}%

%%%%%%%%%%%%%%%%
\dictentry{dictionary}{Atmospheric (effects)}{
The blurring and `twinkling' in the images of astronomical objects caused by turbulence in the Earth's atmosphere. 
Commonly known as ``astronomical seeing''.
}%

%%%%%%%%%%%%%%%%
\dictentry{dictionary}{Additive Bias of Shear}{
{\scshape{\footnotesize see} \gls{Multiplicative and Additive Bias of Shear}}.
}%

%%%%%%%%%%%%%%%%
\dictentry{dictionary}{Multiplicative Bias of Shear}{
{\scshape{\footnotesize see} \gls{Multiplicative and Additive Bias of Shear}}.
}%

%%%%%%%%%%%%%%%% 
\dictentry{dictionary}{Multiplicative and Additive Bias of Shear}{\mpc}
\newcommand\mpc{The estimated shear $\tilde{\gamma}$ is usually biased with respect to
the true shear $\gamma$. To a good approximation biases are expected to be linear
in $\gamma$ \citep{2013MNRAS.429..661M,2008A&A...484...67P,2009A&A...500..647P}, expressed through the relation
\setcounter{equation}{0}
\renewcommand{\theequation}{MC.\arabic{equation}}
\be
\tilde{\gamma} = m\;\gamma + c\;.
\ee
This was first used in \cite{2006MNRAS.368.1323H}, and has subsequently been extended to more 
general biases in ellipticity using a similar linear relation. 
$m$ is referred to as a multiplicative bias and $c$ as an additive bias. Both
may depend on characteristics of the source, such as signal-to-noise,
apparent size, light profile, or the colour gradient of galaxies. 
In \cite{2007MNRAS.376...13M,2012MNRAS.423.3163K} a quadratic term $q\gamma^2$ was also used.
}%

%%%%%%%%%%%%%%%%
\dictentry{dictionary}{STEP}{
The Shear Testing Program (STEP) was a series of weak lensing community challenges \cite{2006MNRAS.368.1323H,2007MNRAS.376...13M}, which started in 2004 (\url{http://www.roe.ac.uk/~heymans/step/cosmic_shear_test.html}). 
STEP1 involved analysis of ground based images, STEP2 added complex morphologies, STEP3 (SpaceSTEP) analysed space-based images and STEP4 was an abbreviated attempt at a gradual approach to adding increased complexity to simulated images. 
STEP naturally led into the GRavitational lEnsing Accuracy Testing (GREAT) program; 
{\scshape{\footnotesize see} \gls{GREAT}}.
}%

%%%%%%%%%%%%%%%%
\dictentry{dictionary}{Third Eccentricity}{
{\scshape{\footnotesize see} \gls{Ellipticity}; \gls{Ellipticity (observed)}; \gls{Quadrupole Moments}}.
}%

%%%%%%%%%%%%%%%%
\dictentry{dictionary}{Third Flattening}{
{\scshape{\footnotesize see} \gls{Ellipticity}; \gls{Ellipticity (observed)}; \gls{Quadrupole Moments}}.
}%


%%%%%%%%%%%%%%%%                                                                                               
\dictentry{dictionary}{Star}{
In weak lensing the word `star' can either refer to the physical object `a star' \citep[see][]{2013A&G....54a1.22R}, 
or in a different context to mean a `point-like object' in an image which 
is an indicator of the Point-Spread-Function (PSF) 
of the instrument, telescope and atmosphere (if present). For example `the stars in this image look broad' would 
mean the PSF inferred from the images of stars (which are point-like objects) has a large width, not the stars 
themselves are physically larger. Derivatives of this term are `stellar'; for example, 
the `stellar ellipticity' is a reference not to the ellipticity of the physical object, the star, 
but rather to the ellipticity of the observed PSF in an image.
}%

%%%%%%%%%%%%%%%%                                                                                               
\dictentry{dictionary}{Star-Galaxy}{\starg}
\newcommand\starg{
Star-galaxy is used in reference to statistics that involve the mutliplication of 
quantities derived from stars (point-like objects) in an image, that are a measure the PSF, and 
quantities derived from galaxy observations in the same image. A common statistic used 
is the star-galaxy cross-correlation which is the 2-point configuration space correlation function 
of the derived stellar ellipticity and the galaxy ellipticities. In \cite{2012MNRAS.427..146H} this 
statistic is described by defining an observed ellipticity $\epsilon^{\rm obs}$ as 
\setcounter{equation}{0}
\renewcommand{\theequation}{SG.\arabic{equation}}
\begin{equation}
\epsilon^{\rm obs}=\epsilon^{\rm int}+\gamma+\eta+A^T_{\rm sys}\epsilon^{\star}\;,
\end{equation} 
where $\epsilon^{\rm int}$ is the unlensed `intrinsic' ellipticity of a galaxy, $\gamma$ is the 
weak lensing shear, $\eta$ is a random component caused by noise in the images and $\epsilon^{\star}$ is 
the ellipticity of the PSF measured using images of stars ($A^T_{\rm sys}$ is an amplitude that 
characterises the propogation of PSF ellipticity into observed galaxy ellipticity). 
By taking the 2-point correlation function of this equation with $\epsilon^{\star}$ a statistic can be 
define 
\begin{equation}
\langle\xi_{\rm sg}\rangle=\langle\epsilon^{\rm int}\epsilon^{\star}\rangle
+\langle\gamma\epsilon^{\star}\rangle+\langle\eta\epsilon^{\star}\rangle
+CA^T_{\rm sys}\;,
\end{equation}
where $C_{ij}=\langle\epsilon^{\star}_i\epsilon^{\star}_j\rangle$ is the correlation of the PSF ellipticity 
between several exposures ($i$ and $j$). In the absence of systematic effects caused by the PSF this quantity 
should have a mean of zero and a distribution characterised by intrinsic ellipticity and shear distributions, 
and has hence been used a test for systematics in weak lensing data.
}%
%%%%%%%%%%%%%%%%
\dictentry{dictionary}{Stage (dark energy)}{\Stage}
\newcommand\Stage{
In \cite{DETF} the Dark Energy Task Force defined several different levels of dark energy experiments in order to rank their relative capabilities. 
The ranking was done using their defined Figure of Merit (FoM) which is the inverse of the area of the predicted 
$68\%$ confidence region for an experiment in the $(w_0,w_a)$ plane. This is commonly approximated in the 
majority of the literature by using a Fisher matrix where 
\setcounter{equation}{0}
\renewcommand{\theequation}{SDE.\arabic{equation}}
\begin{equation}
{\rm FoM}=\frac{1}{{\rm det}(S_{w_0w_a}[F])}\;,
\end{equation} 
where the denominator is the determinant of the $(w_0, w_a)$ Schur compliment of a Fisher matrix containing 
cosmological parameters. 
The dark energy stages were divided into four categories, where a Stage I experiment 
was the least constraining experiment and Stage IV was the most constraining. 
As of 2006, these stages were defined in \cite{DETF} as 
\begin{enumerate}
\item 
Stage I represents what is now known.
\item 
Stage II represents the anticipated state of knowledge upon completion of
ongoing projects that are relevant to dark energy.
\item 
Stage III comprises near-term, medium-cost, currently proposed projects.
\item 
Stage IV comprises a Large Survey Telescope (LST), and/or the Square
Kilometer Array (SKA), and/or a Joint Dark Energy (Space) Mission (JDEM).
\end{enumerate}
The acronyms used in the Stage IV definitions reflect proposed experiments at that time. The projects that 
will satisfy the Stage IV capabilities and are now (c. 2016) in construction are Euclid, LSST, SKA, DESI 
and WFIRST.
}%

%%%%%%%%%%%%%%%%
\dictentry{dictionary}{Spin Weight}{
This is the type of weighting in a statistic that the spin of the field in question has been implied. 
For example, ellipticity is a spin-2 ($s=2$) quantity and whenever the term $\exp(-2i\phi)$ appears in some statistics it would result in a spin-weighted quantity. As a specific example spherical harmonics needs to be 
modified for fields with a general spin, resulting in spin-weighted spherical harmonics 
\citep[see][]{2005PhRvD..72b3516C}.
}%

%%%%%%%%%%%%%%%%
\dictentry{dictionary}{Miscentering}{
Refers to a situation where the statistics is sensitive to inaccurate estimation of the centre/centre-of-mass/average, 
of a distribution. Such cases are referred to as a `miscentering problems'. 
Examples include galaxy cluster mass estimates being incorrect due to the estimation of the centre of the cluster mass distribution 
(e.g. \cite{2015MNRAS.452..998H}). 
Another example is if the estimation of galaxy or stellar ellipticity is sensitive to estimates of the central pixel 
of the objects, then the method involved may be referred to as being `sensitive to miscentering' (see for example 
\cite{2007MNRAS.382..315M} and \cite{2013MNRAS.429.2858M} for methods to mitigate this issue).
}%

%%%%%%%%%%%%%%%%
\dictentry{dictionary}{Mixing Matrix}{\MixingMatrix}
\newcommand\MixingMatrix{
A mixing matrix is a matrix that encodes the correlation introduced by missing data in azimuthal 
wavenumber space $\ell$. Effectively this is a determination of the aliasing between angular modes caused as 
a result of missing data. The matrix is commonly given the symbol $M_{\ell\ell'}$, 
where $\ell$ and $\ell'$ are angular wavenumber --- {\scshape{\footnotesize see} \gls{l-mode}}. It is also 
referred to as the `mode-mode coupling matrix'. 
Referring to \cite{2011MNRAS.412...65H} the harmonic transform of an angular mask $K(\hat n)$ as a function of angle $\hat n$ is 
\setcounter{equation}{0}
\renewcommand{\theequation}{MM.\arabic{equation}}
\begin{equation}
K_{\ell m}=\int {\rm d}\Omega_{\hat n} K(\hat n)Y^*_{\ell m}(\hat n) \;,
\end{equation} 
where $Y_{\ell m}$ are spherical harmonic transforms and the integral is over angle, $m$ and $\ell$ are angular wavenumbers 
in latitude and longitude projected onto the sky. The covariance, or correlation, of the mask is given by 
${\mathcal K}_{\ell\ell'}\equiv [1/(2\ell+1)]\sum_m K_{\ell m}K^*_{\ell m}$. For an isotropic spin-2 field the 
spherical harmonic transforms are spin-2 functions and $m$ is integrated over. 
In this case aliasing of the E-mode and B-mode correlation functions are 
both affected by an angular mask, and the resulting $E$-mode mixing matrix is defined as 
\begin{eqnarray}
M^{EE, EE}_{\ell\ell'}&=&\frac{2\ell'+1}{8\pi}\sum_{\ell''}(2\ell''+1){\mathcal K}_{\ell\ell'}[1+(-1)^{\ell+\ell'+\ell''}
\left( \begin{array}{ccc}
\ell & \ell' & \ell'' \\
2 & -2 & 0 \end{array} \right)^2.
\end{eqnarray}
In the case that a flat-sky approximation is chosen, or in the case of a three dimensional spherical-Bessel 
representation of the data, this definition changes as described in e.g. \cite{2014MNRAS.442.1326K}.
}%
%%%%%%%%%%%%%%%%
\dictentry{dictionary}{Model Bias}{
Refers to biases in inferred ellipticity measurements of galaxies away from the true ellipticity of an object that 
would have been inferred if an exact model for the light distribution of the galaxies was used.  
The hypothetical bias is caused by the use of a non-optimal model in an algorithm that employs some loss function 
between the data and the model. A particular example, shown in \cite{2010MNRAS.404..458V} is the fitting of a 
2D Gaussian profile to a 2D exponential disk or 2D S\'ersic profile. Related to model biases 
are biases that are caused as a result of have non-optimal weight functions in a 
moment-based ellipticity inference method e.g. KSB \citep[e.g. KSB][]{1995ApJ...449..460K}.
}%
%%%%%%%%%%%%%%%%
\dictentry{dictionary}{Model Fitting (method)}{
In weak lensing `a model fitting method' typically refers to a shape measurement method that 
uses a model, or set of models, that are fit to some galaxy imaging 
data with the aim to use the loss function of the fitting method to infer the ellipticity and size of 
the object in question. Examples include fitting 2D S\'ersic or exponential disk 
profiles to galaxies or a combination (e.g. the lensfit method 
\citep{2007MNRAS.382..315M,2008MNRAS.390..149K,2013MNRAS.429.2858M} and im3shape method \citep{2013MNRAS.434.1604Z}), 
fitting sums of Gaussians (e.g. the NGMIX method \citep{2015ascl.soft08008S}), 
or fitting more complex 2D functions \citep[e.g. shapelets][]{2003MNRAS.338...35R,2003MNRAS.338...48R}.
}%
%%%%%%%%%%%%%%%%
\dictentry{dictionary}{Moment (method)}{
A moment, or `moment based', method is an algorithm that infers the 
quadrupole moments from an image through a direct integration or sum of the pixel values (typically 
with some associated weight function). Examples are the KSB \cite{1995ApJ...449..460K} and DEIMOS \cite{DEIMOS} algorithms.
}%

%%%%%%%%%%%%%%%%
\dictentry{dictionary}{Fermat's principle}{
This principle states that a light ray must traverse a path between two points which takes the least time. 
In gravitational lensing, this is used in field equations describing the light deflection and was described in 
\cite{0264-9381-7-8-011} and \cite{1999math.ph...6023G}.
}%

%%%%%%%%%%%%%%%%
\dictentry{dictionary}{Einstein Radius}{
The angular radius of an Einstein ring. For a point mass lens, this is given by 
\setcounter{equation}{0}
\renewcommand{\theequation}{ER.\arabic{equation}}
\begin{equation}
\theta_E = \sqrt{\frac{4G_{\rm N}M}{c^2}\frac{D_{ls}}{D_lD_s}}\;,
\end{equation} 
where $M$ is the mass of the lens, $D_l$, $D_s$ and $D_{ls}$ are the observer-to-lens, observer-to-source and lens-to-source 
angular diameter distances respectively, and $G_{\rm N}$ and $c$ are
Newton's gravitational constant and the speed of light in a vacuum. 
}%

%%%%%%%%%%%%%%%%
\dictentry{dictionary}{Einstein Ring}{A ring-like
  image of a gravitationally lensed background galaxy that is produced
  when the background galaxy lies along the same line of sight to the
  observer as the lens. In order to observe an Einstein ring, the mass
  distribution of the lens needs to be axially symmetric, as seen from
  the observer, and the source must lie on top of the
  resulting (point-like) caustic. However, note that none of the lenses that
    produce observed Einstein rings are axi-symmetric, nor ar the       
    sources {\it exactly} aligned; such an explanation only applies to
    infinitesimally small sources. Such images can be used to
  determine cosmological parameters through measurement of the angular
  diameter distances, if one has an independent
    determination of the mass (and note that in general angular diamter distances are known to 
    better precision than galaxy masses). Several such objects have
  been studied to date e.g.  \cite{1989AJ.....97.1283L},
  \cite{2002ApJ...575...95C}, \cite{2008ApJ...677.1046G}.
  }%

%%%%%%%%%%%%%%%%
\dictentry{dictionary}{EB-mode Decomposition}{\EBmodedecomposition}
\newcommand\EBmodedecomposition{ Any spin-2 field (such as
  gravitational shear) can be decomposed into curl-free modes (called
  E-modes) and divergence-free modes (called B-modes). However such a 
  decomposition can be non-unique because there are
    also ambiguous modes -- for example, a constant shear is neither
    E- nor B-mode.  A pure E-mode shear field displays radial or
  tangential alignment around matter over- or under-densities. Whereas a
  pure B-mode shear field features curl patterns, where orientations are at 
  45 degrees relative to the line between the lensed image and the matter over- or 
  under-densities. Since cosmological signals only generate E-modes to
  good approximation, the analysis of B-modes is considered a valuable
  test for systematic effects, such as intrinsic galaxy alignments or
  errors in the modelling of PSFs. Note that some ambiguous
  modes exist and are usually discarded after an EB-mode decomposition
  (see e.g.  \cite{2016arXiv160501414L}). For quantitative details on
  the EB-mode decomposition \cite{2002A&A...389..729S} presented a
  flat-sky approach and \cite{2005PhRvD..72b3516C} for a curved-sky
  approach. 
  }%
  
%%%%%%%%%%%%%%%%
\dictentry{dictionary}{II term}{
{\scshape{\footnotesize see} \gls{Intrinsic Alignment}}.
}%

%%%%%%%%%%%%%%%%
\dictentry{dictionary}{GI term}{
{\scshape{\footnotesize see} \gls{Intrinsic Alignment}}.
}%

%%%%%%%%%%%%%%%%
\dictentry{dictionary}{GG term}{
{\scshape{\footnotesize see} \gls{Intrinsic Alignment}}.
}%

%%%%%%%%%%%%%%%%
\dictentry{dictionary}{Born Approximation}{
When considering a weak scattering process, the Born approximation assumes that the total driving 
field, that is causing the scattering, can be approximated as the incident field at each point in the scatter. 
This approximation is used in weak lensing for simplicity \citep{Refsdal1970,1988ApJ3301S}. It approximates a lensing 
mass as being on a 2D plane along the line of sight where the deflection of light occurs instantaneously when the light 
ray intersects the plane. This is assummed to be a good approximation if the extent of the lensing mass in comoving 
distance is much smaller than the comoving distance between the observer and the lens, and the lens and source -- which 
is the case in the majority of gravitational lens systems. Several studies have tested the validity of this approximation 
for example \cite{2012MNRAS.420..455S} and
\cite{2016arXiv160508761M}.}%

%%%%%%%%%%%%%%%%
\dictentry{dictionary}{Background}{
In weak lensing, background can mean either 
\begin{enumerate}
\item 
The emission or radiation which originates at a distance from the observer which is greater than a source in question 
e.g. a ``background galaxy'', ``background radiation'', ``background galaxies''. 
\item 
Unresolved sources of radiation that originate between the source and an observer. For example sub-mm observations of
high-$z$ QSOs are affected by the the Cosmic Infrared Background (CIB), that originates from smaller redshift. 
\item
Used to refer to approximately isotropic sources of radiation. For example the Cosmic Microwave Background. 
\item  
A synonym for `noise' in an image
\end{enumerate}.
}%

%%%%%%%%%%%%%%%%
\dictentry{dictionary}{B-mode}{
The ``magnetic'' or odd-parity component of a spin-2 field, such as the weak
lensing shear field {\scshape{\footnotesize see} EB-mode Decomposition}
}%

%%%%%%%%%%%%%%%%
\dictentry{dictionary}{Arc}{
A strongly distorted image of a background object (i.e. the source) that appears as an arc in an observation caused by 
strong lensing.
}%

%%%%%%%%%%%%%%%%
\dictentry{dictionary}{Arclet}{
A single distorted image of a background object (i.e. the
source) with significant lensing-induced ellipticity. For a discussion of arclet definitions and detection 
steps see e.g. \cite{2004A&A...416..391L}.
}%

%%%%%%%%%%%%%%%%
\dictentry{dictionary}{Top-Hat (correlation function)}{
Refers to an ellipticity correlation function with top-hat functional form as a weight. 
This is defined in \cite{1992ApJ...388..272K} and equations A3 and A4 in \cite{2013MNRAS.430.2200K}.
}%

%%%%%%%%%%%%%%%%
\dictentry{dictionary}{Tangential Shear}{\tshear}
\newcommand\tshear{The components of complex shear $\gamma_1$ and $\gamma_2$ can be defined relative to a local Cartesian 
coordinate frame on the sky. However it is often apt to consider the projected shear components in a 
rotated frame, particularly in the case of galaxy clusters where the centre of the polar coordinate frame 
can be defined as the centre of the cluster. For a lensing cluster the image distortions are 
aligned tangentially about the cluster. If $\phi_C$ specifies the angular position about the centre of the 
coordinate frame then the tangential and cross-component shears (aligned respectively 
perpendicular and parallel to the radius vector) are:
\setcounter{equation}{0}
\renewcommand{\theequation}{TS.\arabic{equation}}
\begin{equation}
\gamma_t = -{\rm Re}[\gamma {\rm e}^{-2{\rm i}\phi_C}]
\end{equation}
and $\gamma_x = {\rm Im}[\gamma {\rm e}^{-2{\rm i}\phi_C}]$; where $\gamma=\gamma_1+{\rm i}\gamma_2$. 
An axisymmetric lensing mass should only produce a tangential signal in the shear so the cross-component shear 
(which should be $\gamma_x = 0$) can be used to estimate the noise on the measurement of the tangential shear.}%

%%%%%%%%%%%%%%%%
\dictentry{dictionary}{Critical Surface Density}{
{\scshape{\footnotesize see} \gls{Surface Density}}.
}%

%%%%%%%%%%%%%%%%
\dictentry{dictionary}{Surface Density}{
The three dimensional density $\rho(\bar{r})$ of an extended lensing mass can be projected onto 
a 2D plane, known as the `lens plane' ({\scshape{\footnotesize see} Born Approximation}), 
to obtain a surface mass density distribution defined as 
\setcounter{equation}{0}
\renewcommand{\theequation}{SD.\arabic{equation}}
\begin{equation}
\Sigma(\bar{\xi}) = \int_0^{D_\text{S}} \rho(\bar{r}) \text{d}r_z \; ,
\end{equation}
with $\bar{r}$ being a 3D vector in space and $\bar{\xi}$ a 2D vector on the lens plane. The 
surface density can then be related to the deflection angle for constant surface mass density within a finite azimuthal aperture
\begin{equation}
\alpha(\theta) = \frac{D_\text{LS} D_\text{L}}{D_\text{S}} \frac{4 \pi G \Sigma}{c^2}\theta \; , 
\end{equation}
where $D_\text{L}$ is the distance to the lens, $D_\text{S}$ is the distance to the source and $D_\text{LS}$ 
is the lens-source distance. Also, $\xi = D_\text{L} \theta$, and $\theta$ is the angle of 
deflection seen by the observer. The critical surface mass density can then be defined as 
\begin{equation}
\Sigma_\text{critical} = \frac{D_\text{S}}{D_\text{LS} D_\text{L}} \frac{c^2}{4 \pi G \Sigma} \; .
\end{equation}
When the condition $\Sigma > \Sigma_\text{critical}$ is satisfied multiple images are produced (i.e. strong lensing).
}%

%%%%%%%%%%%%%%%%
\dictentry{dictionary}{Intrinsic Alignment}{\iadef}
\newcommand\iadef{Intrinsic is a term that is used to denote the individual galaxy or population properties without, or before, gravitational 
lensing effects are included. Intrinsic alignment refers to hypothesised correlated orientations of galaxies in the Universe, 
caused by effects other than gravitational lensing. 
\\
\\
An observed ellipticity for an individual galaxy is written like 
\setcounter{equation}{0}
\renewcommand{\theequation}{IA.\arabic{equation}}
\begin{equation}
\epsilon^{\rm obs}=\epsilon^{\rm int}+\gamma \;,
\end{equation}
where $\epsilon^{\rm int}$ is the unlensed `intrinsic' ellipticity of a galaxy, $\gamma$ is the
weak lensing shear. The intrinsic part is labelled $I$ and the shear part is labelled $G$. 
Assuming that the intrinsic ellipticities of galaxies do not have a random orientation or phase, one can take the correlation 
function of the observed ellipticity 
\begin{equation}
\langle \epsilon^{\rm obs}_i\epsilon^{*,\rm obs}_j\rangle=
\langle \epsilon^{\rm int}_i\epsilon^{*, \rm int}_j\rangle+
\langle\epsilon^{\rm int}_i\gamma^*_j\rangle+
\langle\gamma_i\epsilon^{*, \rm int}_j\rangle+
\langle\gamma_i\gamma^*_j\rangle \;,
\end{equation}
where $i$ and $j$ label different redshifts. The four terms on the right hand side are given the abbreviations 
II, IG, GI, and GG. In the context of power spectrum this can be written as
\begin{equation}
C_{\ell,ij} = C_{\ell,ij}^\textrm{II} + C_{\ell_ij}^\textrm{IG} + C_{\ell_ij}^\textrm{GI}+ C_{\ell_ij}^\textrm{GG}\;.
\end{equation}
Assuming that the $j$ redshift bin is more distant that the $i$ redshift bin the correlation IG should be zero, except in 
the case that galaxy redshifts are incorrect or there are systematic effects in the data. The remaining three terms are described as follows 
\begin{enumerate}
\item GG is the gravitational-gravitational or shear-shear correlation between two galaxies. This is the 
cosmic shear that surveys tend to measure, which is the gravitational lensing of a distant source galaxy by the gravitational matter field of the lens. 
\item GI is the gravitational-intrinsic or shape-shear correlation, which 
is the correlation between the intrinsic ellipticity of one galaxy with the shear of another galaxy. This 
happens when a foreground galaxy ellipticity is correlated via IA to structure of the lens that 
shears a background galaxy. A lens causes background galaxies to be aligned tangentially while galaxies 
physically close to the lens are stretched radially towards the lens due to tidal forces. This produces a negative GI correlation. 
\item II is the intrinsic-intrinsic or shape-shape correlation of two galaxies. This is 
only for the galaxies that are physically close and share the same local environment; as the 
formation of galaxies and their orientation is highly affected by the tidal force of the local environment. 
\end{enumerate}
The contamination due to II can be reduced by removing galaxy pairs at the same 
redshift \citep{2003MNRAS.339..711H,2002A&A...396..411K,2003A&A...398...23K}. The GI 
contribution is harder to remove since galaxy pairs at the same line-of-sight at all angles are affected. 
The contribution due to GI can be reduced in a model-independent way 
within the survey \citep{2008A&A...477...43J,2009A&A...507..105J}, or can be modelled 
and its effect then removed \citep{2005A&A...441...47K,2010A&A...523A...1J}. Each of the methods 
have advantages and disadvantages. 
\\
\\
There are various models for the intrinsic alignment:
\begin{enumerate}
\item Linear alignment model describes the local alignment, or orientation of their semi-major axis, of galaxies due to interactions with their local tidal gravitational field. The model predicts that the ellipticity of 
galaxy is a linear function of the tidal quadrupole. This is normally applied to elliptical galaxies. 
The model is applicable on large scales (i.e. $\gtrsim 30$ Mpc). 
In the model the intrinsic ellipticity of a galaxy is assumed to follow the linear relation
\begin{equation}
e = \frac{C_1}{4\pi G}\left( \nabla^2_x - \nabla^2_y \; , \; 2\nabla_x \nabla_y \right) \; \mathcal{S}[\Psi_P] \; ,
\end{equation}
where $\Psi_P$ is the Newtonian potential at the time of galaxy formation, $\mathcal{S}$ is a smoothing filter that cuts off fluctuations on galactic scales, $\nabla$ is a comoving derivative and $C_1$ is a normalisation constant. The model is described in \cite{2001MNRAS.320L...7C,2004PhRvD..70f3526H,2010PhRvD..82d9901H} 
\item Quadratic alignment model describes the local alignment of galaxies, or orientation of their semi-major axis, due to interactions with their local tidal gravitational field. The model predicts that the 
ellipticity of a galaxy is a quadratic function of the tidal field. This is normally applied to spiral 
galaxies because it is hypothesised that the angular momentum of the spiral galaxy during 
gravitational collapse should influence its orientation. The model requires two tidal quadrupoles 
to describe the orientation. The model is described in \cite{2010PhRvD..82d9901H} 
\end{enumerate}
For a review of {\it intrinsic alignments} see e.g. \cite{2015PhR...558....1T}.
}%

%%%%%%%%%%%%%%%%
\dictentry{dictionary}{3D Mass Map}{
Reconstruction of the total matter
distribution in 3D i.e. celestial angle and redshift, or comoving distance (assumming a cosmology). 
{\scshape{\footnotesize see} \gls{Mass Map}; \gls{Kappa}; \gls{Potential Map}}.
}%

%%%%%%%%%%%%%%%%
\dictentry{dictionary}{2D Cosmic Shear Analysis}{
Analysis of the cosmic shear ({\scshape{\footnotesize see} \gls{Cosmic Shear}}) signal without
employing individual redshift information of the lensed source galaxies.
}% 

%%%%%%%%%%%%%%%%
\dictentry{dictionary}{3D Cosmic Shear}{
The large-scale shear lensing signal, using the photometric redshift
positions of source galaxies as radial information,  
either per galaxy or in fine radial bins. Also refers to a spherical-Bessel transform of a photometric weak lensing 
data set see e.g. \cite{2014MNRAS.442.1326K}.
}%

%%%%%%%%%%%%%%%%
\dictentry{dictionary}{Magnification}{\Magnification}
\newcommand\Magnification{
Gravitational lensing distorts and magnifies the intrinsic light
profile of sources. In the weak lensing regime, the effect on a source
is described as linear mapping by the distortion matrix ${\mathcal A}$
({\scshape{\footnotesize see} \gls{Distortion Matrix}}) that includes an  
isotropic focussing by $\kappa$ and anisotropic
focussing by $\gamma$. Due to the focussing the observed flux of a source
changes relative to the flux of the unlensed source by
\setcounter{equation}{0}
\renewcommand{\theequation}{MG.\arabic{equation}}
\begin{equation}
\mu = 
\frac{1}{\abs{\det {\mathcal{A}}}} = 
\frac{1}{(1-\kappa)^2-|\gamma|^2}\;,
\end{equation}
where $|\gamma|^2=\gamma_1^2+\gamma_2^2$. 

In addition, this so-called
magnification changes the observed local number density of sources on the
sky. If $n_0(> S,z){\rm d}z$ is the unlensed number density of galaxies within ${\rm d}z$, 
and flux larger than $S$, then at an angular position $\theta$, where the 
magnification is $\mu(\theta,z)$, the observed number counts are modified as
\begin{equation}
n(> S,z) {\rm d}z = \frac{1}{\mu(\theta,z)} n_0 \left( >\frac{S}{\mu(\theta,z)} \right) \;.
\end{equation}
This effect is prominent for large, extended lenses such as galaxy
clusters but can also be studied for galactic lenses by devising
cross-correlation techniques \citep{2001PhR...340..291B}.
}%

%%%%%%%%%%%%%%%%
\dictentry{dictionary}{Aperture Mass}{\ApertureMass}
\newcommand\ApertureMass{
A smoothed version of the convergence $\kappa$
({\scshape{\footnotesize see} \gls{Kappa}})  
that can be directly estimated from the ellipticities of
sources. Therefore the aperture 
mass is basically the projected matter density field
$\kappa(\vec\theta)$ after application of a smoothing kernel
$u(x)$. The aperture mass at $\vec\theta_0$ for an aperture with
smoothing scale $\Theta$ is
\setcounter{equation}{0}
\renewcommand{\theequation}{AM.\arabic{equation}}
\begin{equation}
M_{\rm ap}(\vec\theta_0;\Theta):=
\int\frac{{\rm d}\vec\vartheta}{\Theta^2}\; 
u(|\vec\vartheta-\vec\theta_0|\;\Theta^{-1})\;\kappa(\vec\vartheta). 
\end{equation}
In the case that a compensated filter with $\int_0^\infty{\rm
  d}x\;x\;u(x)=0$ is chosen, we equivalently find 
\begin{equation}
M_{\rm ap}(\vec\theta_0;\Theta):=
\int\frac{{\rm d}\vec\vartheta}{\Theta^2}\;
q(|\vec\vartheta-\vec\theta_0|\;\Theta^{-1})\;
\gamma_{\rm t}(\vec\vartheta;\vec\theta_0)\;,
\end{equation}
where $\gamma_{\rm t}(\vec\vartheta;\vec\theta_0)$ denotes the
tangential shear component of $\gamma(\vec\vartheta)$ relative to the
direction of $\vec\vartheta-\vec\theta_0$, and
\begin{equation}
q(x)=
\left(\frac{2}{x^2}\int_0^x{\rm d}y\;y\;(y)\right)-u(x)\;,
\end{equation}
is the corresponding smoothing filter of the shear field. As galaxy
ellipticities are estimators of shear, $M_{\rm ap}$ can be directly
estimated from source ellipticities; up to shot noise and possibly a
bias due to gaps in the data that overlap with the aperture. The
aperture mass is traditionally employed in cosmological studies, in
searches for matter concentrations in the (projected) LSS, and in
studies of galaxy bias. A transformation of the shear field 
on the sky with a compensated filter, which corresponds to a weighting 
of the lensing convergence field on the same patch of sky
}%
 
%%%%%%%%%%%%%%%%
\dictentry{dictionary}{Lensing Potential}{\LensingPotential}
\newcommand\LensingPotential{
The lensing potential (or deflection potential) $\psi(\theta)$ is a projected gravitational potential that can be 
related to the observed quantities of weak lensing. 
If $\vec\beta$ is the arrival direction of a light ray on the sky in a perfectly homogeneous universe 
this defines the position of the light ray in the source plane. 
Gravitational lensing by fluctuations in the matter or energy density shifts $\vec\beta$ to a new direction 
$\vec\theta$, which is the position of the deflected light ray in the lens plane. 
We denote by $\vec\alpha(\vec\theta):=\vec\theta-\vec\beta(\vec\theta)$ ({\scshape{\footnotesize see} Lens Equation \ref{eq:LE3}}) as the deflection angle of a 
light ray observed in the direction of $\vec\theta$. 
The vector field $\vec\alpha(\vec\theta)$ is 
curl-free so that it can be expressed as gradient field of a scalar 
field $\psi(\vec\theta)$. 
In the flat-field approximation, 
a possible choice is $\alpha(\vec\theta)=\nabla\psi(\vec\theta)=(\partial_1+{\rm
  i}\partial_2)\;\psi(\vec\theta)$ with
\setcounter{equation}{0}
\renewcommand{\theequation}{LP.\arabic{equation}}
\begin{equation}
\psi(\vec\theta) = 
\frac{1}{\pi}
\;\int_{{\mathcal R}^2} {\rm d}^2\theta \;\kappa(\vec\theta)\;\ln{|\vec\theta - \vec\theta’|}\;,
\end{equation}
where $\kappa(\vec\theta)$ is the convergence in direction of $\vec\theta$. From the 
previous relation follows the two-dimensional Poisson equation $\nabla\nabla^\ast\psi(\vec\theta)=2\kappa(\vec\theta)$. 
In addition, the local shear $\gamma=\gamma_1+{\rm i}\;\gamma_2$ derives 
from $\psi(\vec\theta)$ through $\gamma=(1/2)\;\nabla\nabla\psi(\vec\theta)$.
}%

%%%%%%%%%%%%%%%%
\dictentry{dictionary}{Light Bundle}{
A bunch of light rays close to a fiducial light ray. Light bundles are
a concept used to study the 
local differences in light deflections, up to linear order, between the fiducial light ray and rays close-by. 
In weak gravitational lensing the observation of an entire image of a distant galaxy can
be approximated by the observation of a single light bundle. 
This means that  
the convergence $\kappa(\vec\theta)$ and shear $\gamma(\vec\theta)$ are normally assumed to be constant over the angular 
extend of a source (also known as a linear distortion approximation); 
however in stronger field regions higher-order image distortions can also occur 
{\scshape{\footnotesize see} \gls{Flexion}}.}%

%%%%%%%%%%%%%%%%
\dictentry{dictionary}{Strong Gravitational Lensing}{ 
  Strong gravitational lensing
  is a classification of gravitational lensing effects which is 
  characterised by multiple lensed images of a single source, an        d
  arcs.  Strong lensing can occur when the mass density of the lens is greater
  than a critical density; {\scshape{\footnotesize see} \gls{Surface Density}}, and when the light
  rays from a source pass within with the critical curve of the lens.}%

%%%%%%%%%%%%%%%%
\dictentry{dictionary}{Morphology}{ 
Galaxy morphology refers to the apparent
  physical shape of a galaxy as it appears in an image. Morphology is
  the analysis of those shapes, typically within classification of a
  shape into a model or category. For example a galaxy may have a
  spiral, elliptical, barred-spiral, disk, or disk-plus-disk
  morphology. There are several schemes in use by which galaxies can
  be classified according to their morphologies, the most famous being
  the Hubble sequence, devised by Edwin Hubble and later expanded by
  Gerard de Vaucouleurs and Allan Sandage. {\scshape{\footnotesize
      see} \gls{Early and Late Type (Galaxies)}}. 
      }%

%%%%%%%%%%%%%%%%
\dictentry{dictionary}{Multi-Bin}{
A bin is a finite range in a variable of interest over which measurements are averaged. Multi-bin 
means that more than one bin has used in an analysis. In cosmic shear ``multi-bin'' is used as a 
short-hand prefix to mean that an analysis was done using multiple redshift slices 
e.g. ``multi-bin tomography'', or a ``multi-bin analysis''.}%

%%%%%%%%%%%%%%%%
\dictentry{dictionary}{Multi-Epoch}{\mep}
\newcommand\mep{Epoch in this context refers to the time at which an observation was performed. Multi-epoch is used as a 
prefix as in ``multi-epoch data'' which would refer to a set of data (typically imaging data) that 
were taken at different real times. This phrase is usually used to refer to times ``epochs'' that are 
typically characterised by a qualitative difference in observing
conditions. See e.g. \cite{2006A&A...449..539S}.}%

%%%%%%%%%%%%%%%%
\dictentry{dictionary}{n-of-z n(z)}{
This is a shorthand way to refer to the number density of galaxies, or
total number of galaxies, in a survey as a function of redshift. The
$n(z)$ can be a functional form, or a binned distribution. It can
refer to a surface number density, physical comoving number density
(which is less often used), or total number of galaxies. 
An $n(z)$ can be derived in several different ways, for example, by using one-point redshift estimates or summing posterior redshift distributions in a photometric survey (as well as other methods), or more directly in a spectroscopic survey. Usually pronounced ``n-of-z''.
}%

%%%%%%%%%%%%%%%%
\dictentry{dictionary}{Narrowband}{
Narrow refers to the width of the wavelength range over which the filter is defined. The definition of narrow vs broad is subjective, but weak lensing literature typically refers to the u, g, r, i, z filters\footnote{\url{https://en.wikipedia.org/wiki/Photometric\_system}} for example as ``broadbands'', and anything that has a wavelength interval smaller than these as narrow band. Examples of surveys that use narrow bands are the COMBO-17 survey \cite{2003A&A...401...73W}, and the PauCAM survey \cite{2009ApJ...691..241B}. The phrase is often used in conjunction as in ``narrowband data'', ``narrowband survey'' etc.. {\scshape{\footnotesize see} \gls{Band}; \gls{Broadband}}.
}%

%%%%%%%%%%%%%%%%
\dictentry{dictionary}{Noise Bias}{\NoiseBias}
\newcommand\NoiseBias{
Noise bias can mean one of two things in the weak lensing context
\begin{enumerate}
\item 
In weak lensing noise bias refers to biases in the measured ellipticity of galaxies caused by the influence of random 
noise in images. It is one of the main sources of systematic errors in weak lensing measurements. 
It arises from high-order noise terms in the measurement of the shapes of galaxies. It increases 
in magnitude as a galaxy's signal-to-noise decrease, and this is due to relative the increase of noise 
in the galaxy images, see e.g. \cite{2004MNRAS.353..529H}, \cite{2012MNRAS.425.1951R}, \cite{2013MNRAS.429.2858M}, 
\cite{2014MNRAS.439.1909V}.
\item 
Noise bias can also, much less commonly in the weak lensing literature, but a standard usage in CMB literature, 
refer to the bias in a power spectrum measured away from what would have been measured in the absence of shot (Poisson) noise. See
e.g. \cite{2013MNRAS.431..609N}
\end{enumerate}.
{\scshape{\footnotesize see} \gls{Bias}; \gls{Accuracy}}.
}%

%%%%%%%%%%%%%%%%
\dictentry{dictionary}{Non-convolutive}{
This refers to any process that may have an impact on an image that cannot be represented as a 
convolution of an original image with a kernel that characterises the process as 
described in \cite{2013MNRAS.429..661M} and \cite{2013MNRAS.431.3103C}. An example would 
be that the effect of a telescopes aperture is convolutive, as its effect can be represented 
as a convolution of an original image with a point spread function (PSF). But the effect of 
CCD charge transfer inefficiency (CTI) and the brighter-fatter effect \citep{2015ExA....39..207N} 
on an image cannot be represented in this way, and are therefore ``non-convolutive''. 
{\scshape{\footnotesize see} \gls{Convolved/Convolutive}}.
}%

%%%%%%%%%%%%%%%%
\dictentry{dictionary}{Nicaea}{
Refers to a public code\footnote{\url{http://www.cosmostat.org/software/nicaea/}} that calculates theoretical  
weak lensing correlation functions.
}%

%%%%%%%%%%%%%%%%
\dictentry{dictionary}{Non-linear (Matter Power Spectrum)}{\nlpk}
\newcommand\nlpk{This refers to the part of the matter power spectrum $P_{\delta}(k,z)$ (as a function of scale $k$ and redshift $z$) 
that is governed by physics that involves the solution of equations that are non-linear. In contrast the 
linear part of the matter power spectrum can be computed from first order perturbation theory of the 
matter over-density field and initial conditions {\scshape{\footnotesize see} \gls{Linear (Matter Power Spectrum)}}. 
The non-linear physics primarily involves gravitational collapse of dark matter structures, but also 
non-dark matter (baryonic) physics. There is no exact and widely accepted demarkation between the 
linear and non-linear parts of the function however $k\simeq 1 h{\rm Mpc}^{-1}$ is an approximate scale at a redshift of $z=0$. 
There are various phenomenological models for this range of scales, for example {\scshape{\footnotesize see} Halo Model}, 
and various algorithms that have been calibrated on simulations to model this range, the most 
recent (c. 2016) of these being from \cite{2015MNRAS.454.1958M}\footnote{\url{https://github.com/alexander-mead/HMcode}} .
}%

%%%%%%%%%%%%%%%%
\dictentry{dictionary}{Number Density}{\NumberDensity}
\newcommand\NumberDensity{
The number density may refer to several different quantities depending on the context 
\begin{enumerate}
\item 
The number density is the three-dimensional volume number density $n=N/V$, 2D area number density $n=N/A$, or 1D line number density $n=N/L$, where $N$ is the total number of objects, $V$ is volume, $A$ is area and $L$ is length. 
\item 
In the lensing community, it is most commonly used as a short-hand expression to mean the number density of galaxies as a function of redshift; e.g. as in ``the number densities of red and blue galaxies are different'' {\scshape{\footnotesize see} \gls{n-of-z n(z)}}.
\item 
It can also be used as a short-hand to refer to the effect that weak lensing magnification can change the observed number density 
of galaxies as a function of redshift. For example as in ``does that statistic include the effect of number density?'' 
may, in context, refer to impact of a statistic on the weak lensing magnification effect
\end{enumerate}.
}%

%%%%%%%%%%%%%%%%
\dictentry{dictionary}{Out-of-Band}{
This refers to any electromagnetic flux that has wavelengths that are shorter or longer than a pre-determined range that labels 
the band in question. For example if a CCD is required to be sensitive over the range $500-800$ nm then any sensitivity of the 
CCD to photons with wavelengths outside of this range would be referred to as out-of-band. {\scshape{\footnotesize see} \gls{Band}; \gls{In-band}}.
}%

%%%%%%%%%%%%%%%%
\dictentry{dictionary}{Over-density}{\ovd}
\newcommand\ovd{This refers in general to an excess over the mean density of a general quantity $\alpha$ and can be defined as 
\setcounter{equation}{0}
\renewcommand{\theequation}{OD.\arabic{equation}}
\begin{equation} 
\delta_{\alpha}= \frac{\alpha - \bar\alpha}{\bar\alpha} \;,
\end{equation} 
where $\delta_{\alpha}$ is the over-density, and $\bar\alpha$ is the mean of some quantity $\alpha$. 
Over-density functions appear in the description of the distribution of the matter field in weak lensing, 
\be 
\delta_{\rho}(x) = \frac{\rho(x) - \bar\rho }{\bar\rho} \;,
\ee
where $\rho$ is the matter density at coordinate $x$; 
the covariance of which leads to the matter power spectrum. However, it can also be used in the 
context of galaxies or stars, as in ``there is a over-density in that part of the image'', 
or other quantities. It can also be used in an graphical setting for example referring 
to an over-density of points in a scatter plot. {\scshape{\footnotesize see} \gls{Structure Formation in the Universe}}. 
}%

%%%%%%%%%%%%%%%%
\dictentry{dictionary}{Peak (counts and statistics)}{
A peak can refer to a maxima in any function, however specifically in weak lensing `peak counts' and `peak statistics' refer 
to the maxima identified in maps of the projected mass or projected gravitational potential -- colloquially 
referred to a `dark matter maps' -- and the statistics derived from those maxima. 
`Peak statistics' is a general term that refers to the n-point statistics that can be done with such maxima. 
`Peak counts' refers to the statistic that can be done with the number density distribution of the maxima. Such statistics can be 
sensitive probes of cosmology as shown in \cite{2010PhRvD..81d3519K,2013MNRAS.430.2896C,2015A&A...576A..24L}.
}%

%%%%%%%%%%%%%%%%
\dictentry{dictionary}{Photometric Redshifts (Photo-z; photo-zee; photo-zed)}{\pzs}
\newcommand\pzs{
Photometric redshifts are estimates of the redshift of an object derived from imaging data from which measurements 
of object fluxes (photometry --- that is when the object is observed through various standard broadband filters) 
have been derived --- i.e. not spectroscopy. Redshifts are then inferred from a set of such 
observations in different wavelengths. There are many methods and algorithms that can perform this task with varying 
degrees of accuracy and precision, 
see \cite{2008MNRAS.386.1219B,2010A&A...523A..31H,2011MNRAS.417.1891A,2013ApJ...775...93D,2014MNRAS.445.1482S} 
for comparisons of different photometric redshift estimation methods. 
Photo-z, can be pronounced ``photo-zee'' or ``photo-zed'' and is a short-hand expression used to refer 
to such photometric redshifts estimates.
}%

%%%%%%%%%%%%%%%%
\dictentry{dictionary}{Pixel}{\pixel}
\newcommand\pixel{
Pixel can refer to a physical object or be used in a grammatic sense in weak lensing literature
\begin{enumerate}
\item 
This is a defined area on the surface of a CCD within which the flux of photons is summed over some 
time-period (defined by the electronics and software of the CCDs and imaging system of a telescope). Pixels are 
typically square or rectangular -- although there sensitivity can become slightly non-square due to electrons and 
electric fields leaking or interfering between neighbouring pixels. In a rendered image pixels are represented by square or 
rectangular areas of brightness. 
\item 
Pixel can also be used as a prefix in weak lensing, for example ``pixel effect'' would refer 
to an effect caused by the fact that images are rendered in pixels; ``pixel noise'' \citep{2012MNRAS.424.2757M} refers to the noise 
in images that have been rendered in pixels; from a combination of sources. Pixelisation refers 
to the process of an objects image being rendered in pixels
\end{enumerate}.
}%

%%%%%%%%%%%%%%%%
\dictentry{dictionary}{Pointing}{
A pointing is an area of sky from which observations have been taken in sequence. For example a single 
pointing --- or one pointing --- may contain multiple exposures (or only one exposure). 
The word refers to the physical act of pointing the telescope at an area of sky. 
If an area of sky is revisited in a non-sequential way, i.e. at a later time between which other observations 
have been made, this would be referred to as multi-epoch imaging where each pointing 
contained imaging from several epochs. {\scshape{\footnotesize see} \gls{Multi-Epoch}}.
}%

%%%%%%%%%%%%%%%%
\dictentry{dictionary}{Alignment}{\Alignment}
\newcommand\Alignment{
In weak lensing this can refer to several different physical effects depending on the context 
\begin{enumerate}
\item Galaxy alignment refers to the 3D orientation of the semi-major axis of a 
galaxy -- characterised by its stellar distribution -- with respect
to some direction.
\item In weak lensing images we observe the projected shapes of galaxies and in this context galaxy alignment refers 
to the observed apparent 2D projected alignment of a galaxy's observed surface
brightness distribution with respect to some direction.
\item Intrinsic alignment refers to the phenomenon that the semi-major axes of a galaxy's stellar distribution may 
exhibit a tendency to be aligned with the local dark matter
over-densities. This is a physical effect, distinct from the previous item which is an observed alignment. 
\item In particular contexts alignment can also refer to the weak lensing shear. For example, around galaxy 
clusters background galaxies tend to be tangentially aligned, due to the weak lensing effect of the cluster. 
\item Alignment can also refer to instrumental properties where a particular aspect of a physical object 
in an instrument or telescope has a degree of freedom which is correlated, or points towards, a particular 
coordinate. For example a mirror may be aligned with the optical axes of a telescope
\end{enumerate}
}%

%%%%%%%%%%%%%%%%
\dictentry{dictionary}{Convolved/Convolutive}{
This refers to any effect that can be represented mathematically as a convolution of an image with some function. 
For example the diffraction effects of a telescopes aperture can be represented as a convolution of the original 
image with the Fourier transforms of 
impulse response function of the telescope (the Point Spread Function
(PSF)). {\scshape{\footnotesize see} \gls{Convolution}; \gls{Non-convolutive}}. 
}%

%%%%%%%%%%%%%%%%
\dictentry{dictionary}{Critical Curve}{
Critical curves, for any general lens mass distribution are closed curves in the lens 
plane where the determinant of the distortion matrix for the lens is singular, i.e within 
which there is no unique single solution of the lens equation, resulting in the possibility 
of multiply-imaged background sources within the critical curve. Caustics are 
the corresponding curves in the image plane. The Einstein radius is a particular 
example of a critical curve for a circularly symmetric lens. {\scshape{\footnotesize see} \gls{Critical Surface Density}; \gls{Distortion Matrix}}.
}%

%%%%%%%%%%%%%%%%
\dictentry{dictionary}{Cross-Correlation Function}{\crosscorr}
\newcommand\crosscorr{ 
In weak lensing studies a cross-correlation \emph{between} two quantities $A$ and $B$ refers to 
the finding the inter-variable variance or covariance of the quantities i.e $\langle A B^*\rangle$ where 
angle brackets refer to an ensemble average, and $^*$ is a complex conjugate. Particular and common examples 
are the cross-correlation between variables relating to the PSF and inferred shear measurements from 
galaxies, and the shear cross-correlation function, or power spectrum, between two redshift bins (known as tomography). 
Correlation can refer to both the intra-variable correlation, or `auto' correlation, whereas cross-correlation refers to 
inter-variable correlation, {\scshape{\footnotesize see} \gls{Correlation Functions and Power Spectra}; \gls{Auto-Correlation Function}; \gls{Correlation Function}; \gls{Covariance}}
}%

%%%%%%%%%%%%%%%%
\dictentry{dictionary}{Current (survey)}{
In weak lensing literature ``current'' is used to refer to an experiment that is in 
operations at the time of the writing of paper, and is used as a shorthand for circa (or ``around about''). 
For example ``ABC is a current survey...'' in a paper in 2015 is
shorthand for ``The ABC survey (circa 2015)...''.}%

%%%%%%%%%%%%%%%%
\dictentry{dictionary}{Curved Sky}{ 
In opposition to flat sky
  ({\scshape{\footnotesize see} Flat Sky}). If an operation, or
  statistic, can work in the `curved sky' regime this refers to the
  operation in question being able to correctly work in the spherical
  coordinate system relevant for the celestial sphere. An example of
  a curved-sky approach would be to perform a spherical harmonic
  transform of function on the sphere which involves spherical
  harmonic functions $Y^m_{\ell}(\theta,\phi)$ (where $\theta$, $\phi$
  are co-latitude and co-longitude and $\ell$ and $m$ are the
  spherical harmonic transforms of these variables); in the flat-sky
  case the equivalent transform would be a Fourier transform on a
  projected flat (tangent) plane.}%

%%%%%%%%%%%%%%%%
\dictentry{dictionary}{Distortion Matrix}{\DistortionMatrix}
\newcommand\DistortionMatrix{The distortion matrix ${\cal A}_{ij}$
  (also known as the linear distortion matrix) is the Jacobian of the
  lens equation ({\scshape{\footnotesize see} Lens Equation}). 
  If the source is much smaller than the angular size
  upon which the properties of the lens change
  ({\scshape{\footnotesize see} Light Bundle}) then the local
  distortion of its image can be described by a
  linear mapping 
  between the source and image planes. The Jacobian matrix of
  a lens is defined as
\setcounter{equation}{0}
\renewcommand{\theequation}{DM.\arabic{equation}}
\begin{equation} 
{\cal A}_{ij}=\frac{\partial\beta_i}{\partial\theta_j}=\frac{\partial}{\partial\theta_j}(\theta_i-\alpha_i)\;,
\end{equation}
or related to the Hessian matrix $\psi_{ij}$ of the 
deflection potential $\psi$ using the relation  
between the deflection angle and the deflection
potential ({\scshape{\footnotesize see} Deflection Angle}) 
\begin{equation}
A_{ij}=\delta_{ij}-\frac{\partial\alpha_i}{\partial\theta_j}=\delta_{ij}-\frac{\partial^2\psi}{\partial\theta_i\partial\theta_j}=[M^{-1}]_{ij}\;,
\end{equation}
where $M_{ij}$  is the magnification matrix, and the Hessian 
matrix $\psi_{ij}=\partial^2\psi/\partial\theta_i\partial\theta_j$ represents the deviation 
of the lens mapping from the identity mapping (that which describes no
distortion). {\scshape{\footnotesize see} \gls{Weak Lensing Equations}}.}%
%%%%%%%%%%%%%%%%%%%%%%

\dictentry{review}{Geodesic Deviation Equation}{\gde}
\newcommand\gde{
In any metric theory of
    gravity, the transverse separation vector $\vec\xi$ of two
    infinitesimally close light rays evolves according to the
    \index{Geodesic deviation equation}{\it
      equation of geodesic deviation}, 
\setcounter{equation}{0}
\renewcommand{\theequation}{GDE.\arabic{equation}}
% 
\begin{equation}
  \frac{\d^2\vec\xi}{\d\lambda^2} = {\mathcal T}\,\vec\xi\;,
\label{eq:GDE1}
\end{equation}
%
%\citep[see, e.g., Chap.\,3 of][]{1992grle.book.....S}, 
where $\lambda$ is the affine parameter
along the ray, and ${\cal T}(\lambda)$ is the {\it optical tidal matrix} 
which depends on the Riemann curvature tensor and the
wave-vector of the rays.
\\
\\
We consider a light bundle with vertex at $\lambda_0$ around a
fiducial ray. Each ray of the bundle is specified by the angle
$\vec\theta$ it encloses with the fiducial ray at the vertex. The
linearity of (\ref{eq:GDE1}) allows us to write $\vec\xi={\cal
  D}\vec\theta$, where the \index{Distance matrix}distance matrix
${\cal D}(\lambda)$ obeys 
% 
\begin{equation}
  \frac{\d^2 {\cal D}}{\d\lambda^2} = {\mathcal T}\,{\cal D}\;,
\label{eq:GDE2}
\end{equation}
%
with ${\cal D}(\lambda_0)=0$ and $(\d{\cal
  D}/\d\lambda)(\lambda_0)={\cal I}$, with ${\cal I}$ being the
$2\times 2$ unit matrix, if $\lambda$ is chosen such that it locally
agrees with proper distance. The {\it angular-diameter
  distance}\index{Angular-diameter distance!general definition} to a
point $\lambda$ as 
measured from the vertex at $\lambda_0$ is defined as the square root
of the ratio of the cross-sectional area of the light bundle and its
solid opening angle; hence, $D_{\rm A}(\lambda)=\sqrt{\det{\cal
 D}(\lambda)}$. For an isotropic space-time, such as the
Friedman--Robertson--Walker model, the tidal matrix is proportional to
the unit matrix, and the matrix equation (\ref{eq:GDE2}) reduces to a
scalar equation. In this case, the angular-diameter as defined here
reduces to the one given in \gls{Friedman--Robertson--Walker Models}.
A point along the light bundle where $\det{\cal
D}(\lambda)=0$ (for $\lambda\ne\lambda_0$) is called a \index{Caustic
point, general definition}{\it caustic point} or {\it conjugate
point}. The caustics defined for a gravitational lens system in 
{\scshape{Lens Equation}} is a special case of this general definition.
}

%%%%%%%%%%%%%%%%
\dictentry{dictionary}{E/B-mode Shear}{
A\index{E- and B-modes of a shear
      field}\index{Shear $\gamma$!E/B-mode decomposition} general
    spin-2 field, such as 
    shear $\gamma(\vec\theta)$, can be decomposed into E-modes,
    B-modes, and ambiguous modes. If the shear field is caused by a
    (geometrically-thin) gravitational lens, its two Cartesian
    components are given as second-order partial derivatives of the
    deflection potential $\psi$ -- see Eq.\,(\ref{eq:LE10}); hence,
    these two components are mutually related. Introducing the complex
    nabla operator $\nabla_{\rm c}:=\partial/\partial\theta_1 + {\rm i}
\partial/\partial\theta_2$, the shear produced by a lens can be
written as $\gamma=\nabla_{\rm c}\nabla_{\rm c}\psi/2$, and the
Poisson equation (\ref{eq:LE6}) becomes $\kappa=\nabla^*_{\rm
  c}\nabla_{\rm c}\psi/2$. The local relation (\ref{eq:LE18}) between
the gradient of $\kappa$ and the derivatives of the shear reads
$\nabla_{\rm c}\kappa=\nabla_{\rm c}^*\gamma$. Taking another
derivative, we find $\nabla_{\rm c}^*\nabla_{\rm c}\kappa
=\nabla_{\rm c}^*\nabla_{\rm c}^*\gamma$. Since the Laplacian
$\nabla_{\rm c}^*\nabla_{\rm c}$ and $\kappa$ are both real, the
imaginary component of $\nabla_{\rm c}^*\nabla_{\rm c}^*\gamma$
vanishes for a shear field due to a gravitational lens.
\\
\\
A general shear field $\gamma(\vec\theta)$ does not have this
property. If a shear field is caused by a lens, meaning that there
exists a real scalar function $\psi^{\rm E}$ such that $\gamma=\nabla_{\rm
c}\nabla_{\rm c}\psi^{\rm E}/2$, it is called an `E-mode' shear field. In
particular, for an E-mode field, the imaginary part of $\nabla_{\rm
c}^*\nabla_{\rm c}^*\gamma$ vanishes identically. A pure B-mode shear
field is obtained from a real scalar function $\psi^{\rm B}$ through
$\gamma={\rm i}\nabla_{\rm
c}\nabla_{\rm c}\psi^{\rm B}/2$, for which the real component of $\nabla_{\rm
c}^*\nabla_{\rm c}^*\gamma$ vanishes identically. A general shear
field is the superposition of both, and can be written as 
$\gamma=\nabla_{\rm
c}\nabla_{\rm c}\rund{\psi^{\rm E}+{\rm i}\psi^{\rm B}}/2$.
There also exist ambiguous modes: For a shear field which is a linear
function of position, $\nabla_{\rm c}^*\nabla_{\rm c}^*\gamma\equiv
0$; hence, such a linear shear field can be either attributed to
E-modes or to B-modes. If a shear field has a B-mode contribution, the
result of the mass reconstruction relation (\ref{eq:LE17}) yields an
imaginary component for $\kappa$.
\\
\\
A division of the shear into E-, B-, and ambiguous modes also must be
considered for second- and higher-order shear statistics
({\scshape{\footnotesize see} \gls{Cosmic Shear}}).
}

%%%%%%%%%%%%%%%%
\dictentry{review}{Weak Lensing Equations}{\wle}
\newcommand\wle{
Consider first the case of a
    localized matter distribution located at an angular-diameter
    distance $D_{\rm d}$, and a source at angular-diameter distance
    $D_{\rm s}$ from us. We assume that the localised matter
    distribution is embedded in an otherwise homogeneous and isotropic
    background universe, described by the Friedman--Robertson--Walker
    metric. Assume that the extent of the matter
    distribution along the line-of-sight is much smaller than $D_{\rm
      d}$ and $D_{\rm ds}$, the angular-diameter distance of the
    source from the lens; such a lens is called
    \index{Geometrically-thin gravitational lens}{\it geometrically thin}.
The lensing properties of such a deflector is
    determined solely by its {\it surface mass density}\index{Surface
      mass density $\Sigma$} 
    $\Sigma(\vec\xi)$
\setcounter{equation}{0}
\renewcommand{\theequation}{LE.\arabic{equation}}
%
\begin{equation}
  \Sigma(\vec\xi) \equiv \int\d r_3\,\rho(\xi_1,\xi_2,r_3)\;,
\label{eq:LE1}
\end{equation}
%
the projection of its volume density $\rho$ along the line-of-sight,
where $\xi$ is a two-dimensional vector perpendicular to the 
line-of-sight, locating a point in the \index{Lens  plane}{\it lens
  plane} (a plane 
at distance $D_{\rm d}$ perpendicular to the line-of-sight) relative
to some conveniently chosen origin. A light ray traversing this matter
distribution is deflected, where the \index{Deflection angle $\hat{\vec\alpha}$}{\it
  deflection angle} 
$\hat{\vec\alpha(\vec \xi)}$, the angle between the incoming and
outgoing light ray at the lens plane, is given by
%
\begin{equation}
  \hat{\vec\alpha}(\vec\xi) = \frac{4G_{\rm N}}{c^2}\,
  \int\d^2\xi'\,\Sigma(\vec\xi')\,
  \frac{\vec\xi-\vec\xi'}{|\vec\xi-\vec\xi'|^2}\;,
\label{eq:LE2}
\end{equation}
%
provided that the gravitational potential of the deflector is small
everywhere, $\abs{\phi_{\rm N}\,c^{-2}}\ll 1$\footnote{This condition
  is satisfied everywhere, except in the immediate vicinity of black
  holes and neutron stars; additionally, the very small fraction of
  light rays that 
  traverse strong gravitational fields are irrelevant in a
  gravitational lens situation owing to their extremely small
  magnification -- see below.} and that the matter distribution causing
the gravitational field of the deflector is moving at velocities $\ll
c$. The condition $\abs{\phi_{\rm N}\,c^{-2}}\ll 1$ also guarantees
that the deflection angle is small, $\abs{\hat{\vec\alpha}}\ll 1$, and
thus we can safely use the approximations
$\tan\rund{\abs{\hat{\vec\alpha}}}\approx \abs{\hat{\vec\alpha}}
\approx \sin\rund{\abs{\hat{\vec\alpha}}}$ in the following.
\\
\\
The {\it lens equation}\index{Lens equation} relates the direction
$\vec\theta$ of a light ray on 
the observer's sky, measured relative to a conveniently chosen origin, 
to the angular position $\vec\beta$ the
corresponding source would have in the absence of lensing, 
%
\begin{equation}
  \vec\beta = \vec\theta-\frac{D_\mathrm{ds}}{D_\mathrm{s}}\,
  \hat{\vec\alpha}(D_\mathrm{d}\vec\theta) =:
  \vec\theta-\vec\alpha(\vec\theta)\;,
\label{eq:LE3}
\end{equation}
%
where in the final step the {\it scaled deflection angle}\index{Scaled
  deflection angle $\vec\alpha$}\index{Deflection angle
  $\hat{\vec\alpha}$!Scaled deflection angle $\vec\alpha$}
$\vec\alpha(\vec\theta)$ was defined. Equation (\ref{eq:LE3}) is
called the {\it lens equation};\footnote{Note that, strictly speaking,
  the lens equation is specified only up to an irrelevant (since
  unobservable) translation in the source plane.} it yields a mapping
of the lens plane onto the source plane. Defining the {\it
  convergence}\index{Convergence $\kappa$} or dimensionless surface
mass density\index{Surface mass density!dimensionless $\kappa$}
%
\begin{equation}
  \kappa(\vec\theta) := 
  \frac{\Sigma(D_\mathrm{d}\vec\theta)}{\Sigma_\mathrm{cr}}
  \quad\hbox{with}\quad
  \Sigma_\mathrm{cr} = \frac{c^2}{4\pi G_{\rm N}}\,
  \frac{D_\mathrm{s}}{D_\mathrm{d}\,D_\mathrm{ds}}\;,
\label{eq:LE4}
\end{equation}
%
where $\Sigma_\mathrm{cr}$ is the \index{Critical surface mass density
$\Sigma_{\rm cr}$}{\it critical surface mass density},
the scaled deflection angle can be expressed as 
%
\begin{equation}
  \vec\alpha(\vec\theta) = \frac{1}{\pi}\,
  \int_{\mathbb{R}^n}\d^2\theta'\,\kappa(\vec\theta')\,
  \frac{\vec\theta-\vec\theta'}{|\vec\theta-\vec\theta'|^2}\;.
\label{eq:LE5}
\end{equation}
%
A mass distribution which has $\kappa\ge1$ somewhere, 
i.e.~$\Sigma\ge\Sigma_\mathrm{cr}$, produces multiple images for some
source positions $\vec\beta$. Hence, $\Sigma_\mathrm{cr}$ is a characteristic
value for the surface mass density which is the dividing line between
\index{Weak and strong lensing}{\it weak and strong lensing}.
\\
\\
The identities $\nabla \ln |\vec\theta|=\vec\theta/|\vec\theta|^2$
and $\nabla^2\ln|\vec\theta|=2\pi\delta_{\rm D}(\vec\theta)$ allow us to
write the scaled deflection angle as the gradient of the {\it
  deflection potential}\index{Deflection potential $\psi$}
%
\be
\psi(\vec\theta)={1\over
\pi}\int_{\mathbb{R}^n}\d^2\theta'\;\kappa(\vec\theta')\,
\ln|\vec\theta-\vec\theta'|\;, \quad \hbox{with} \quad
\nabla^2\psi =2\kappa\;, \quad
\hbox{as}\quad
\vec \alpha =\nabla\psi(\vec\theta)\;.
\label{eq:LE6}
\ee
%
Hence, $\psi(\vec\theta)$ satisfies the two-dimensional Poisson
equation\index{Poisson equation for deflection potential $\psi$} with
source $\kappa(\vec\theta)$. The \index{Fermat potential $\tau_{\rm
    F}$}{\it Fermat potential} 
%
\be
\tau_{\rm F}(\vec\theta;\vec\beta) ={1\over
2}\rund{\vec\theta-\vec\beta}^2-\psi(\vec\theta)\; , 
\label{eq:LE7}
\ee
%
is proportional to the difference of the light travel times of a light ray
from the source $\vec\beta$ via the lens at $\vec\theta$ to the
observer and  that of a straight ray ($\vec\theta=\vec\beta$) in the
absence of a deflector. \index{Fermat's principle in gravitational
  lensing}Fermat's principle in conformally stationary 
space-times, specialised to the Friedman--Robertson--Walker
model, states that actual light rays are those with stationary light
travel times. Indeed, $\nabla\tau_{\rm F}(\vec\theta;\vec\beta)=0$ is
equivalent to the lens equation (\ref{eq:LE3}).
\\
\\
Gravitational light deflection preserves the \index{Surface brightness
  conservation}surface brightness;
hence, the surface brightness of an image $I(\vec\theta)$ is given by
the surface brightness $I^{\rm s}$ at the corresponding source
position,
%
\begin{equation}
  I(\vec\theta) = I^{\rm s}[\vec\beta(\vec\theta)]\;.
\label{eq:LE8}
\end{equation}
%
For lensed images that are much smaller than the angular scale on
which the lens properties change, the lens mapping can be
\index{Linearised lens equation}linearised
locally. The distortion of images is then described by the {\it Jacobi
matrix of the lens equation}\index{Distortion matrix ${\cal A}$}\index{Jacobi
matrix of the lens equation} (or {\it lens distortion matrix}) 
\begin{equation}
  \mathcal{A}(\vec\theta) =
  \frac{\partial\vec\beta}{\partial\vec\theta} =
  \left(\delta_{ij} -
    \frac{\partial^2\psi(\vec\theta)}{\partial\theta_i\partial\theta_j}
  \right) = \left(
    \begin{array}{cc}
      1-\kappa-\gamma_1 & -\gamma_2 \\ 
      -\gamma_2 & 1-\kappa+\gamma_1 \\
    \end{array}
  \right)
=:(1-\kappa)
 \left(
    \begin{array}{cc}
      1-g_1 & -g_2 \\ 
      -g_2 & 1+g_1 \\
    \end{array}
  \right)\;,
\label{eq:LE9}
\end{equation}
%
where the (Cartesian) components of the {\it shear}\index{Shear
  $\gamma$}\index{Shear $\gamma$!Cartesian components}
%
\be
\gamma\equiv\gamma_1+\mathrm{i}\gamma_2 =
\frac{1}{2}(\psi_{,11}-\psi_{,22})+\mathrm{i}\psi_{,12}=
|\gamma|\mathrm{e}^{2\mathrm{i}\varphi}
\label{eq:LE10}
\ee
%
have been defined, where indices separated by a comma denote partial
derivatives w.r.t. the coordinates $\theta_i$. Here and below, the
shear is written as a complex quantity to emphasise its mathematical
property as a spin-2 quantity: Rotating the coordinate system by an
angle $\phi$ transforms the shear\index{Shear $\gamma$!transformation
  behaviour under rotations} as
$\gamma\to\gamma{\rm e}^{-2{\rm i}\phi}$. Since the shear represents
the traceless part of ${\cal A}$, this transformation behaviour follows
from that of matrices under the action of rotations. In the final
expression of Eq.\,(\ref{eq:LE10}), the shear is written as product of
its magnitude $|\gamma|=\sqrt{\gamma_1^2+\gamma_2^2}$ and its phase
factor $\mathrm{e}^{2\mathrm{i}\varphi}$. The final expression in
Eq.\,(\ref{eq:LE9}) defines the {\it reduced shear}\index{Reduced
  shear $g$}
%
\be
g=g_1+{\rm i}g_2=\gamma\,(1-\kappa)^{-1}\;,
\label{eq:LE11}
\ee
%
which describes the deviation of ${\cal A}$ from an isotropic matrix,
responsible for changing the shape of images compared to the shape of
sources ({\scshape{\footnotesize see} \gls{Ellipticity}}). Instead of the
Cartesian components of the shear, one is frequently interested in the
shear components relative to a specific orientation, characterised by
an angle $\vp$; for example, $\vp$ could be the direction of the
separation vector to the center of a lens (relevant in galaxy-galaxy
lensing -- {\scshape{\footnotesize see} \gls{Galaxy-Galaxy Lensing}}) or the
orientation of the separation vector between two sources (relevant in
cosmic shear -- {\scshape{\footnotesize see} \gls{Cosmic Shear}}). One
therefore defines the {\it tangential and cross-components of the
  shear}\index{Shear $\gamma$!tangential and cross
  components}\index{Tangential component of the shear $\gamma_{\rm
    t}$}\index{Cross-component of the shear $\gamma_\times$}
at $\vec\theta$ relative to the given orientation $\vp$ as 
%
\be
\gamma_{\rm t}(\vec\theta,\vp)+{\rm i}\gamma_\times(\vec\theta,\vp)
:=-{\rm e}^{-2{\rm i}\vp}\,\gamma(\vec\theta)\;; \quad
g_{\rm t}(\vec\theta,\vp)+{\rm i}g_\times(\vec\theta,\vp)
:=-{\rm e}^{-2{\rm i}\vp}\,g(\vec\theta)\;,
\label{eq:LE12}
\ee
%
where the corresponding definition is also applied to the reduced
shear. For an axially-symmetric matter distribution
$\kappa(|\vec\theta|)$, the \index{Tangential shear of an
  axi-symmetric lens}tangential shear is constant on circles
around the centre, given by 
%
\be
\gamma_{\rm
  t}(|\vec\theta|)=\bar\kappa(|\vec\theta|) - \kappa(|\vec\theta|)
\;,\quad \hbox{where}\quad
\bar\kappa(\theta)={2\over\theta^2}\int_0^\theta \d\theta'\;\theta'
\kappa(|\vec\theta'|) 
\label{eq:LE13}
\ee
%
is the mean convergence inside radius $\theta$, and the cross shear
vanishes identically, $\gamma_\times\equiv 0$. A similar result holds
for a general mass distribution: The mean of the tangential shear on a
circle of radius $\theta$ around a point $\vec\theta_0$ can be
expressed as
%
\be
\ave{\gamma_{\rm t}}(\theta;\vec\theta_0)
:= \int_0^{2\pi}{\d\vp\over 2\pi}\;\gamma_{\rm
  t}(\vec\theta_0+\vec\theta,\vp) 
=\bar\kappa(\theta;\vec\theta_0) - \ave{\kappa}(\theta;\vec\theta_0)
\;,
\label{eq:LE14}
\ee
%
where $\vp$ is the polar angle of $\vec\theta$, and we defined
%
\be
\ave{\kappa}(\theta;\vec\theta_0):=\int_0^{2\pi}{\d\vp\over
  2\pi}\;\kappa(\vec\theta_0+\vec\theta) \; ; \quad
\bar\kappa(\theta;\vec\theta_0)
:={2\over\theta^2}\int_0^\theta \d\theta'\;\theta'\,
\ave{\kappa}(\theta';\vec\theta_0)\;,
\label{eq:LE15}
\ee
%
i.e., the {\it mean convergence on a circle} and the {\it mean
  convergence inside a circle} of radius $\theta$ around
$\vec\theta_0$. Correspondingly, mean cross component of the shear on
a circle vanishes identically. The relation (\ref{eq:LE14}) forms the
basis of galaxy-galaxy lensing ({\scshape{\footnotesize see} \gls{Galaxy-Galaxy Lensing}}).
%
The shear is linearly related to the surface mass density;
combining Eqs.\, (\ref{eq:LE9}) and (\ref{eq:LE6}), one
finds\index{Shear $\gamma$!relation to convergence $\kappa$}
%
\be
  \gamma(\vec\theta) = \frac{1}{\pi}\int_{\mathbb{R}^2}\d^2\theta'\,
  \mathcal{D}(\vec\theta-\vec\theta')\,
  \kappa(\vec\theta')\;,\quad \hbox{with} \quad
  \mathcal{D}(\vec\theta) \equiv
  \frac{\theta_2^2-\theta_1^2-2\mathrm{i}\theta_1\theta_2}
  {|\vec\theta|^4}
  = \frac{-1}{(\theta_1-\mathrm{i}\theta_2)^2}\;.
\label{eq:LE16}
\ee
%
In Fourier space, the relation between shear and convergence
reads\index{Shear $\gamma$!relation to convergence $\kappa$} 
$\tilde\gamma(\vec\ell)=\tilde\kappa(\vec\ell){\rm e}^{2{\rm
    i}\vp(\vec\ell)}$, where $\vp(\vec\ell)$ is the polar angle of
$\vec\ell$, valid for $\vec\ell\ne \vec 0$. For $\ell=\vec 0$, the
relation between $\tilde\gamma$ and $\tilde\kappa$ is undefined,
related to the fact that a uniform convergence yields no shear. These
relations can be inverted,
$\tilde\kappa(\vec\ell)=\tilde\gamma(\vec\ell){\rm e}^{-2{\rm
    i}\vp(\vec\ell)}$, yielding\index{Convergence $\kappa$!relation to
  shear $\gamma$} 
%
\be
\kappa(\vec\theta)=\kappa_0+\int_{\mathbb{R}^2}\d^2\theta'\,
\mathcal{D}^*(\vec\theta-\vec\theta')\,\gamma(\vec\theta')\;,
\label{eq:LE17}
\ee
%
which forms the basis of mass reconstructions from weak lensing data;
here, $\kappa_0$ is an undetermined additive constant. 
For a general shear field, e.g., if the shear field is estimated from
noisy data, the result of the integral in (\ref{eq:LE17}) may have an
imaginary component. This is related to the E/B-modes of a general
shear field ({\scshape{\footnotesize see} \gls{E/B-mode Shear}}).
Therefore, for practical purposes, one takes the
real part of the integral.
A local
relation between shear (and reduced shear) and
convergence\index{Convergence $\kappa$!local relation to shear
  $\gamma$ and reduced shear $g$} is
provided by 
%
 \begin{equation}
  \nabla\kappa = \left(\begin{array}{c}
    \gamma_{1,1}+\gamma_{2,2} \\
    \gamma_{2,1}-\gamma_{1,2} \\
  \end{array}\right) \equiv \vec u_\gamma(\vec\theta)\;; \quad
  \nabla K(\vec\theta) = \frac{1}{1-g_1^2-g_2^2}\,
  \left(\begin{array}{cc}
    1-g_1 & -g_2 \\
    -g_2 & 1+g_1 \\
  \end{array}\right)\,
  \left(\begin{array}{c}
    g_{1,1}+g_{2,2} \\
    g_{2,1}-g_{1,2} \\
  \end{array}\right) \equiv\vec u_g(\vec\theta)\;,
\label{eq:LE18}
\end{equation}
where $K(\vec\theta)=-\ln[1-\kappa(\vec\theta)]$. The latter
equation shows that from an estimate of the reduced shear, $(1-\kappa)$
can be determined only up to a multiplicative constant.
\\
\\
Assuming the applicability of the linearised lens equation,
the surface brightness of the image follows from (\ref{eq:LE8}),
%
\begin{equation}
  I(\vec\theta) = I^{(\rm s)}\left[
    \vec\beta_0+\mathcal{A}(\vec\theta_0)
    \cdot(\vec\theta-\vec\theta_0)
  \right]\;.
\label{eq:LE19}
\end{equation}
%
For larger images for which the linearised lens equation no longer
provides an accurate approximation, higher-order expansions of the
lens equation need to be considered ({\scshape{\footnotesize see}
  \gls{Flexion}}). The flux of an image $S$ is obtained by integrating the
surface brightness $I(\vec\theta)$ over $\vec\theta$, and is related
to the flux $S_0$ of the unlensed source through $S=|\mu|S_0$. The
{\it magnification}\index{Magnification} (for an infinitesimally small
source) $\mu_{\rm p}$ is 
obtained from the distortion matrix 
(\ref{eq:LE9}), 
%
\be
\mu_{\rm p}={1\over \det{\cal A}}={1\over (1-\kappa)^2-|\gamma|^2}
={1\over (1-\kappa)^2(1-|g|^2)}\;.
\label{eq:LE20}
\ee
%
The sign of $\mu_{\rm p}$ is called the \index{Parity of lensed
images}{\it parity of an image}. Strong lenses can have regions with
negative parity. Curves in the lens plane on which $\mu_{\rm p}=0$ are
called \index{Critical curves}{\it critical curves}; they form in
general closed, smooth curves. Mapping critical curves onto the source
plane through the lens equation yields the \index{Caustics}{\it
caustics}; these are in general also closed curves which are smooth
except for a finite number of \index{Cusps}{\it cusps} where the
tangent vector of the caustic vanishes . The divergence of $\mu_{\rm
p}$ on a critical curve is just a formal one; for sources of finite
extent, the linearised lens equation fails at critical curves, and the
magnification of real (finite) sources, obtained by averaging the
point-source magnification $|\mu_{\rm p}|$ weighted by the surface
brightness, remains finite. In weak lensing, where $\kappa\ll 1$ and
$|\gamma|\ll 1$, $\mu_{\rm p}$ is positive and close to unity.
\\
\\
If the lensing effect of a generic 3-dimensional matter distribution is
considered (as in the case of lensing by the large-scale matter
distribution in the Universe -- {\scshape{\footnotesize see} \gls{Cosmic
  Shear}}), one needs to refer to the laws of light propagation in a
curved space-time. This is governed by the {\it equation of geodesic
deviation} ({\scshape{\footnotesize see} \gls{Geodesic Deviation Equation}}).
We now specialize Eq.\,(\ref{eq:GDE2}) to a
perturbed \index{Metric of a perturbed Friedman--Robertson--Walker model}Friedman--Robertson--Walker metric of the form
%
\begin{equation}
  \d s^2 = a^2(\tau)\left[
    \left(1+\frac{2\Phi_{\rm N}}{c^2}\right)\,c^2\d\tau^2-
    \left(1-\frac{2\Phi_{\rm N}}{c^2}\right)
    \left(\d \chi^2+f_K^2(\chi)\d\omega^2\right)
  \right]\;,
\label{eq:LE22}
\end{equation}
%
where $\chi$ is the comoving radial distance ({\scshape{\footnotesize
    see} \gls{Friedman--Robertson--Walker Models}}), $a=(1+z)^{-1}$ the
scale factor, 
$\tau$ is the \index{Conformal time}conformal time, related to the cosmic
time $t$ through $\d t=a(t)\;\d \tau$, $f_K(\chi)$ is the comoving
angular diameter 
distance, and $\Phi_{\rm N}(\vec x,\chi)$
denotes the Newtonian peculiar gravitational potential which depends on the
comoving position $\vec x$ and cosmic time (or comoving distance $\chi$).
In this metric, the tidal matrix ${\cal T}$ can be calculated in terms of the
Newtonian potential $\Phi_{\rm N}$, and correspondingly,
the equation of geodesic deviation (\ref{eq:GDE1})
yields the evolution equation for the comoving separation vector $\vec
x(\vec\theta,\chi)$ of a ray in a ray bundle, specified by the angle
$\vec\theta$ from the fiducial ray at the vertex, located at the 
observer, 
%
\begin{equation} 
\frac{\d^2\vec x}{\d \chi^2} +
K\,\vec x = -\frac{2}{c^2} \eck{ \nabla_\perp\Phi_{\rm N}\rund{\vec
x(\vec\theta,\chi),\chi} -\nabla_\perp\Phi_{\rm N}^{(0)}\rund{\chi}} \;,
\label{eq:LE23}
\end{equation}
%
with $\vec x(\vec\theta,0)=\vec 0$ and $(\d\vec
x/\d\chi)(\vec\theta,0)=\vec\theta$,  
where $K=(H_0/c)^2\,(\Omega_{\rm tot}-1)$ is the spatial curvature of
the Universe, $\nabla_\perp=(\partial/\partial x_1,\partial/\partial
x_2)$ is the transverse {\it comoving} gradient operator, and
$\Phi^{(0)}_{\rm N}(\chi)$ is the potential along the fiducial ray.
The solution of the homogeneous version of Eq.\, (\ref{eq:LE23})
yields the comoving angular-diameter distance $f_K(\chi)$ (see
Eq.\,\ref{eq:FRW2}). The
formal solution of Eq.\,(\ref{eq:LE23}) is obtained by the method of Green's
function, to yield
%
\begin{equation}
  \vec x(\vec\theta,\chi) = f_K(\chi)\vec\theta - 
  \frac{2}{c^2}\!\int_0^\chi\!\!\! \d \chi'\,f_K(\chi-\chi')\! 
  \eck{ \nabla_\perp\Phi_{\rm N}\rund{\vec
x(\vec\theta,\chi'),\chi'} -\nabla_\perp\Phi_{\rm N}^{(0)}\rund{\chi'}} .
\label{eq:LE24}
\end{equation} 
%
A source at comoving distance $\chi$ with comoving separation $\vec x$
from the fiducial light ray would be seen, in the absence of light
deflection, at the angular separation $\vec \beta=\vec x/f_K(\chi)$
from the fiducial ray (this statement is nothing but the definition of
the comoving angular diameter distance).  Hence, $\vec\beta$ is the
unlensed angular position in the `comoving source plane' at distance
$\chi$, relative to that of the fiducial ray.  Therefore, in analogy with
standard lens theory, we define the Jacobi matrix\index{Distortion
  matrix ${\cal A}$!for a 3-D matter distribution}\index{Jacobi matrix of the
  lens equation!for a 3-D
  matter distribution} 
%
\be
{\cal A}(\vec\theta,\chi)={\partial\vec \beta\over\partial\vec\theta}
={1\over f_K(\chi)}{\partial\vec x\over\partial\vec\theta}\;,
\label{eq:LE25}
\ee
%
and obtain from (\ref{eq:LE24})
%
\be
{\cal A}_{ij}(\vec\theta,\chi)=\delta_{ij}-\frac{2}{c^2}\int_0^\chi \!\! \d\chi'\,
{f_K(\chi-\chi') f_K(\chi')\over f_K(\chi)}\,\Phi_{{\rm N},ik}\rund{\vec
x(\vec\theta,\chi'),\chi'}\,{\cal A}_{kj}(\vec\theta,\chi') \;,
\label{eq:LE26}
\ee
%
which describes the locally linearised mapping introduced by LSS
lensing. This equation still is exact in the limit of validity of the
weak-field metric, and the pair of Eqs.\, (\ref{eq:LE24}) and
(\ref{eq:LE26}) are followed typically in ray-tracing
simulations\index{Ray-tracingsimulations} 
\citep[e.g.,][and references
therein]{2000ApJ...530..547J,2009A&A...499...31H}.
\\
\\
Expanding ${\cal A}$ in powers of $\Phi_{\rm N}/c^2$, and truncating the
series after the linear term yields
%
\be
{\cal A}_{ij}(\vec\theta,\chi)
=\delta_{ij}-\frac{2}{c^2}\int_0^\chi \!\! \d \chi'\,
{f_K(\chi-\chi') f_K(\chi')\over f_K(\chi)}\,
\Phi_{{\rm N},ij}\rund{f_K(\chi')\vec\theta,\chi'} \;,
\label{eq:LE27}
\ee
%
and provides two essential simplification compared to
Eq.\,(\ref{eq:LE26}): First, the distortion is obtained by integrating
along the unperturbed ray $\vec x=f_K(\chi)\,\vec\theta$; this is also
called the \index{Born approximation}{\it Born approximation}. Second, in (\ref{eq:LE26}), the
matrix product in the integrand yields a coupling of tidal field
matrices $\Phi_{{\rm N},ij}$ at different distances $\chi'$, which
disappears in the linear version (\ref{eq:LE27}), rendering ${\cal A}$
symmetric. This approximation is frequently referred to as
\index{Lens-lens coupling}`neglecting
lens-lens coupling'. In the literature, one often sees both
approximations as being collectively called `Born approximation'.
Corrections to the Born approximation are necessarily of order
$\Phi_{\rm N}^2/c^2$ \citep[see][and references therein for
higher-order terms]{2010A&A...523A..28K}.
\\
\\
Employing the Born approximation from here on, we now define the
deflection potential
%
\be
\psi(\vec\theta,\chi):=\frac{2}{c^2}\int_0^\chi \!\! \d \chi'\,
{f_K(\chi-\chi') \over f_K(\chi)\;f_K(\chi')}\,
\Phi_{\rm N}\rund{f_K(\chi')\vec\theta,\chi'} \;,
\label{eq:LE28}
\ee
%
yielding ${\cal A}_{ij}=\delta_{ij}-\psi_{,ij}$, just as in
geometrically-thin lens theory. 
In this approximation, lensing by the 3-D matter
distribution can be treated as an \index{Equivalent lens
  plane}equivalent lens plane with 
deflection potential $\psi$, convergence $\kappa=\nabla^2\psi/2$, and
shear $\gamma=(\psi_{,11}-\psi_{,22})/2+{\rm i}\psi_{,12}$.
\\
\\
One can relate $\kappa$ to fractional density contrast $\delta$ of
matter fluctuations in the Universe by using the Poisson equation, 
to yield\index{Convergence $\kappa$!for 3-D lensing}
%
\be
\kappa(\vec\theta,\chi)=\frac{3H_0^2\Omega_{\rm m}}{2c^2}\,
  \int_0^\chi\,\d \chi'\,\frac{f_K(\chi')f_K(\chi-\chi')}{f_K(\chi)}\,
  \frac{\delta\rund{f_K(\chi')\vec\theta,\chi'}}{a(\chi')}\;.
\label{eq:LE29}
\end{equation}
%
Note that $\kappa$ is proportional to $\Omega_{\rm m}$, since lensing is
sensitive to $\Delta\rho\propto \Omega_{\rm m}\,\delta$, not just to
the density contrast $\delta=\Delta\rho/\bar\rho$ itself.
An equivalent expression for the convergence reads\index{Convergence
  $\kappa$!for 3-D lensing} 
%
\be
\kappa(\vec\theta,z_{\rm s})=\int_0^{z_{\rm s}} \d z\;
{\d D_{\rm prop}\over \d z}\;
{4\pi G\over c^2}\,
{D_{\rm A}(z) D_{\rm A}(z,z_{\rm s})\over D_{\rm A}(z_{\rm s})}\,
\Delta\rho(D_{\rm A}(z)\vec\theta,z)\;,
\label{eq:LE30}
\ee
%
which has an immediate intuitive interpretation: the density contrast
$\Delta\rho = \rho-\bar\rho$ (as a function of proper transverse
separation $D_{\rm A}(z)\vec\theta$)
as the source of the gravitational field,
multiplied by the proper distance interval $\d D_{\rm prop}$, yields a
surface mass density $\d\Sigma$ at redshift $z$, which is multiplied
by the inverse of the critical surface density in the integrand,
resulting in an infinitesimal convergence $\d\kappa$ at redshift
$z$. The convergence is then obtained by line-of-sight integration
of this infinitesimal convergence. If $\delta\rho$ is assumed to be
non-zero only on a small redshift interval centred on $z_{\rm d}$,
the convergence of the geometrically-thin lens equation (\ref{eq:LE4})
is reobtained.
\\
\\
For a redshift (or distance) distribution of sources with $p_z(z)\,\d
z=p_\chi(\chi)\,\d \chi$, 
the effective surface mass density becomes
\be
\kappa(\vec\theta)=\int\d \chi\;p_\chi(\chi)\,\kappa(\vec\theta,\chi)
= \frac{3H_0^2\Omega_{\rm m}}{2c^2}\,
  \int_0^{\chi_{\rm h}}\d \chi\;g_{\rm LE}(\chi)\,f_K(\chi)\,
\frac{\delta\rund{f_K(\chi)\vec\theta,\chi}}{a(\chi)} \;,
\label{eq:LE31}
\ee
%
with
%
\be
g_{\rm LE}(\chi)=\int_\chi^{\chi_{\rm h}}\d \chi'\;p_\chi(\chi')
{f_K(\chi'-\chi)\over f_K(\chi')}\;,
\label{eq:LE32}
\ee
%
which is the source-redshift weighted lens efficiency factor
$D_{\rm ds}/D_{\rm s}$ for a
density fluctuation at distance $\chi$, and
$\chi_{\rm h}$ is the \index{Comoving horizon distance}comoving horizon
distance, obtained from $\chi(a)$ by letting $a\to 0$.
}

%%%%%%%%%%%%%%%%
\dictentry{dictionary}{Fourier Transform}{
Let $f(\vec x)$ be a function on
$\mathbb{R}^n$. We define its Fourier transform\index{Fourier transform} as
\setcounter{equation}{0}
\renewcommand{\theequation}{FT.\arabic{equation}}
%
\be
\tilde f(\vec k):=\int_{\mathbb{R}^n}\d^n x\;f(\vec x)\,
\exp\rund{{\rm i}\,\vec x\cdot \vec k} \;.
\label{eq:FT1}
\ee
%
The inverse Fourier transform then reads
%
\be
f(\vec x)={1\over (2\pi)^n} \int_{\mathbb{R}^n}{\d^n k}\;
\tilde f(\vec k)\,\exp\rund{-{\rm i}\,\vec x\cdot \vec k} \;.
\label{eq:FT2}
\ee
%
The Fourier transform of a function yields a decomposition of this
function into plane waves.
}

%%%%%%%%%%%%%%%%
\dictentry{dictionary}{Correlation Functions and Power Spectra}{
Let $f_i(\vec x)$, $i=1,2$, be
homogeneous and isotropic random fields defined on
$\mathbb{R}^n$. We assume that the expection value of the fields
vanishes, $\ave{f_i(\vec x)}=0$ for all $\vec x$.\footnote{This is not a
  strong restiction, as one can consider the modified field $f'(\vec
  x)=f(\vec x)-\ave{f(\vec x)}$, whose expection value vanishes at each
  point be construction.}
\\
\\
The {\it two-point correlation function} $\xi$ of $f_i$ (often simply
called `correlation function') is defined through
\setcounter{equation}{0}
\renewcommand{\theequation}{CFS.\arabic{equation}}
%
\be
\ave{f_i(\vec x)\,f_j(\vec y)}=:\xi_{ij}\rund{\abs{\vec x-\vec y}}\;,
\label{eq:CFS1}
\ee
%
where the angular brackets denote the enseble average (or expectation
value). Statistical
homogeneity ensures that the correlator depends only on the separation
vector $\vec x-\vec y$, statistical isotropy yields a dependence only
on the absolute value of this separation vector. If $i=j$ in
(\ref{eq:CFS1}), $\xi$ is called the {\it auto-correlation function},
for $i\ne j$ it is the {\it cross-correlation function} of the two
fields. 
\\
\\
The {\it power spectrum} $P$ of $f$ is defined as the correlator in
Fourier space ({\scshape{\footnotesize see} \gls{Fourier Transform}}),
%
\be
\ave{\tilde f_i(\vec k) \,\tilde f_j(\vec k')}
=(2\pi)^n\,\delta_{\rm D}\rund{\vec k+\vec k'}\,P_{ij}\rund{\abs{\vec
    k}}\;.
\label{eq:CFS2}
\ee
%
The Dirac delta `function' is a consequence of the statistical
homogeneity, the dependence of $P$ only on $\abs{\vec k}$ is due to
statistical isotropy. For $i=j$, $P$ is the {\it auto-power spectrum}
(or simply power spectrum), 
whereas for $i\ne j$ it is called {\it cross-power spectrum}.
Correlation function and power spectrum are
Fourier transform pairs, i.e.,
%
\be
P_{ij}\rund{\abs{\vec k}}=\int_{\mathbb{R}^n}\d^n x\;\xi_{ij}\rund{\abs{\vec
    x}}\, \exp\rund{{\rm i}\,\vec x\cdot\vec k}\;.
\label{eq:CFS3}
\ee
Higher-order correlation functions and spectra are defined
analogously. The general $m$-point correlation function $\xi^{(m)}$ of 
homogeneous and isotropic real fields $f_i(\vec x)$ with zero mean is
defined through 
%
\be
\ave{f_i(\vec x_1)\,f_j(\vec x_2)\dots f_p(\vec
  x_m)}=:\xi^{(m)}_{ij\dots p} \;.
\label{eq:CFS4}
\ee
%
Due to statistical homogeneity, $\xi^{(m)}$ depends only on the
separation vectors $\vec x_i-\vec x_1$, $2\le i\le m$. Furthermore,
statistical isotropy implies that $\xi^{(m)}$ is invariant under
rotations of these separation vectors in $\mathbb{R}^n$.  Note that
$\xi^{(2)}\equiv \xi$. 
\\
\\
The $m$-th order polyspectra $P^{(m)}$ are defined as the correlators of the
Fourier transforms of the fields,
%
\be
\ave{\tilde f_i(\vec k_1)\,\tilde f_j(\vec k_2)\dots \tilde f_p(\vec
  k_m)}=:(2\pi)^n\,\delta_{\rm D}\rund{\sum_{i=1}^m \vec k_i}\,
P^{(m)}_{ij\dots p} \;,
\label{eq:CFS5}
\ee
%
where again the Dirac delta `function' appears due to homogeneity, and
the polyspectrum $P^{(m)}$ is invariant under spatial
rotations of the $\vec k_j$.
}

%%%%%%%%%%%%%%%%
\dictentry{review}{Friedman--Robertson--Walker Models}{\frw} 
\newcommand\frw{\index{Friedman--Robertson--Walker models}Homogeneous
    and isotropic models of the universe are described by the
    metric\index{Friedman--Robertson--Walker
      models!metric}\index{Metric of homogeneous and isotropic universes} 
\setcounter{equation}{0}
\renewcommand{\theequation}{FRW.\arabic{equation}}
%
\be
\d s^2=c^2\,\d t^2 - a^2(t)\eck{\d \chi^2
+f_K^2(\chi)\rund{\d\theta^2+\sin^2\theta\, \d\vp^2}}\; ,
\label{eq:FRW1}
\ee
%
where $t$ is the \index{Cosmic time}cosmic time, $a(t)$ the
(dimensionless) cosmic scale\index{Cosmic scale factor
$a$}\index{Scale factor $a$} factor, normalized that today, i.e., for
$t=t_0$, $a(t_0)=1$, $\chi$ the \index{Comoving
distance}\index{Comoving radial coordinate}comoving radial coordinate,
$\theta$ and $\vp$ are the angular coordinates on a unit sphere, and
$f_K(\chi)$ the \index{Comoving angular diameter distance
$f_K$}\index{Angular-diameter distance!comoving}comoving angular
diameter distance, which depends on the \index{Curvature parameter
$K$}curvature parameter $K$ in the following way:
%
\be
  f_K(\chi) = |K|^{-1/2}\,{\rm sinn}\rund{|K|^{1/2} \chi} 
:=
  \left\{
  \begin{array}{ll}
    K^{-1/2}\sin(K^{1/2}\chi) & (K>0) \\
    \chi & (K=0) \\
    (-K)^{-1/2}\sinh[(-K)^{1/2}\chi] & (K<0) \\
  \end{array}\right.\;.
\label{eq:FRW2}
\ee
%
The term in brackets in Eq.\,(\ref{eq:FRW1}) describes a homogeneous,
isotropic three-dimensional space of constant curvature $K$. The
spatial coordinates $\chi,\theta,\vp$ in Eq.\,(\ref{eq:FRW1}) are
called \index{Comovingcoordinates}{\it comoving coordinates}. Object
or observers whose worldlines are characterized by constant values of
$\chi$, $\theta$ and $\vp$ are called \index{Comoving observers}{\it
comoving sources} or {\it comoving observers}. Comoving observers all
experience the same history of the universe and observe the universe
to be isotropic. Their proper time coincides with the cosmic time.  If
we choose coordinates such that we are at $\chi=0$, then light rays we
receive follow spatially radial paths with $c\,\d t=-a(t)\,\d\chi$,
yielding
%
\be
\chi=\int_t^{t_0}{c\,\d t'\over a(t')}\;.
\label{eq:FRW3}
\ee
%
Hence, the light from a source at comoving distance $\chi$ that we
receive today was emitted at cosmic time $t$.
From Einstein's field equations follow the \index{Friedman
  equations}\index{Friedman--Robertson--Walker models!Friedman
  equations}Friedman equations,  
relating the scale factor $a(t)$ to the matter/energy contents of the
universe,\footnote{Note that the Friedman equations are frequently
  written by explicitly including a term from the cosmological
  constant $\Lambda$. In this case, $\rho$ and $p$ refer to the
  density and pressure of matter and radiation only. Here, the
  cosmolgical constant is included as a special case of dark energy,
  with an equivalent density $\rho_\Lambda$ and pressure
  $p_\Lambda=-\rho_\Lambda c^2$.}
%
\be
H^2(a) := \rund{\dot a\over a}^2={8\pi G_{\rm N}\over
  3}\rho(a)-{K\,c^2\over a^2}\; ;\quad
 {\ddot a \over a} = - {4 \pi G_{\rm N} \over 3} \rund{\rho + {3 p \over
 c^2}} \;,
\label{eq:FRW4}
\ee
%
with the {\it Hubble function}\index{Hubble function $H$} $H(a)$,
energy density $\rho(a)\,c^2$, and isotropic pressure $p(a)$, as
measured by comoving observers. The current value of the Hubble
function is the {\it Hubble constant}\index{Hubble constant} $H_0$.
The Hubble constant yields the overall scale of the Universe, and all
lengths scale like $c\,H_0^{-1}$ when 
redshifts are used as a measure of distance. The uncertainty in the
value of the Hubble constant is typically parametrized by writing
%
\be
H_0=h_x\, x\,{\rm km\,s^{-1}\,Mpc^{-1}}\;,
\label{eq:FRW4.1}
\ee
%
where $x$ is a number. For example, traditionally one used $h_{100}$, 
so that $H_0=h_{100}\, 100\,{\rm km\,s^{-1}\,Mpc^{-1}}$.
\\
\\
Density $\rho$ and pressure $p$ contain the various energy components
in the universe, characterized by an {\it
equation-of-state}\index{Equation-of-state}
$p_i=w_i\,\rho_i\,c^2$. Pressureless matter (or `dust'; ofter just
called `matter') has $w_{\rm m}=0$, $p_{\rm m}=0$; radiation is
characterized by $w_{\rm r}=1/3$, $p_{\rm r}=\rho_{\rm r}\,c^2/3$. For
dark energy, we write $w_{\rm DE}\equiv w$, and $p_{\rm
DE}=w\,\rho_{\rm DE}\,c^2$. In the special case that dark energy is
described by a constant vacuum energy density, or equivalently, a
\index{Cosmological constant}cosmological constant, $w=-1$, and
$p_\Lambda=-\rho_\Lambda\,c^2$. The \index{Friedman--Robertson--Walker
  models!adiabatic equation}\index{Adiabatic equation in FRW
  models}adiabatic equation, 
%
\be
\d(\rho c^2 a^3) = -p\, \d(a^3) \;,
\label{eq:FRW5}
\ee
%
then yields for non-interacting energy components
%
\be
\rho_{\rm m}(a)=\rho_{\rm m, 0}\,a^{-3}\; ,\quad
\rho_{\rm r}(a)=\rho_{\rm r, 0}\,a^{-4}\; ,\quad
\rho_{\rm DE}(a)=\rho_{\rm DE, 0}\,\exp\rund{-3\int_1^a\d
  a'\;{1+w(a')\over a'}}\;,
\label{eq:FRW6}
\ee
%
where the additional subscript `0' indicates densities taken at the
current epoch. These are converted into the dimensionless
\index{Density parameters $\Omega_i$}{\it density
parameters} $\Omega_i$ by dividing them by the \index{Critical density
$\rho_{\rm cr}$}{\it critical density}
%
\be
\rho_{\rm cr}={3 H_0^2\over 8 \pi G_{\rm N}} \;,
\label{eq:FRW7}
\ee
%
to obtain
\begin{equation}
  \Omega_{\rm r}={\rho_{\rm r, 0}\over \rho_{\rm cr}}\;,\quad
  \Omega_{\rm m}={\rho_{\rm m, 0}\over \rho_{\rm cr}}\;,\quad
  \Omega_{\rm DE}={\rho_{\rm DE, 0}\over \rho_{\rm cr}}\;.
\label{eq:FRW8}
\end{equation}
Analogous expressions hold for baryons (subscript `b'), cold dark
matter (subscript `c'), hot dark matter (subscript `HDM' or `$\nu$',
if it refers to neutrinos only), etc. The \index{Density parameters
  $\Omega_i$!total density parameter $\Omega_{\rm tot}$}total density
parameter is $\Omega_{\rm tot}=\sum_i \Omega_i$, and is related to the
curvature through \index{Curvature parameter $K$!relation to
  $\Omega_{\rm tot}$}$K=(\Omega_{\rm tot}-1) H_0^2/c^2$. The Hubble
function $H(a)$ \index{Hubble function $H$!in terms of density
  parameters}in terms of the density parameters is given by
%
\be
E^2(a)\equiv \rund{H(a)\over H_0}^2
={\Omega_{\rm r}\over a^{4}}+{\Omega_{\rm m}\over a^{3}}
+{(1-\Omega_{\rm tot})\over a^{2}} +\Omega_{\rm DE} \,
\exp\rund{-3\int_1^a \d a'\;{1+w(a')\over a'}} \;,
\label{eq:FRW9}
\ee
%
The Friedman equation (\ref{eq:FRW9}) is a first-order differential
equation for $a(t)$, and its solution is specified by the current
expansion rate $H_0$. Classifying the general properties of its
solutions in terms of the density parameters, together with
lower limits on $\Omega_{\rm m}$ obtained from cosmological
observations, yields the result that $a\to 0$ at a finite time in the
past. That epoch is called \index{Big Bang}{\it Big Bang}. Relics of
the Big Bang, such as the helium abundance in the Universe and the
existence of the cosmicmicrowave background, confirm this prediction
of the cosmological model.
\\
\\
The \index{Cosmological redshift}{\it cosmological redshift} $z$ of a
comoving source seen by a 
comoving observer today
is related to the scale factor through
$(1+z)=1/a$. The scale factor at which the energy density of matter
and radiation are equal is $a_{\rm eq}={\Omega_{\rm r} / \Omega_{\rm
    m}}$, corresponding to the \index{Redshift of matter-radiation
  equality $z_{\rm eq}$}{\it redshift of matter-radiation
  equality} $z_{\rm eq}=a_{\rm eq}^{-1}-1$.\footnote{Health warning: Cosmologists
  often use the same symbol for meaning totally different mathematical
  functions. For example, they write $H(a)$ to mean the function given
  in (\ref{eq:FRW9}), but also write $H(z)$ to imply the function
  obtained from $H(a)$ by replacing $a$ by $(1+z)^{-1}$. of course,
  this yields a {\it mathematically} very different function, but
  usually no confusion arises by this practice.}
\\
\\
From the metric (\ref{eq:FRW1}) we see that the comoving
angular-diameter distance $f_K$ is the ratio of the comoving
transverse extent of an object, $f_K(\chi)\,\d\theta$ and the angle
$\d\theta$ as observed by an observer at $\chi=0$.  The {\it angular
diameter distance}\index{Angular-diameter distance!in an FRW model
$D_A$} $D_{\rm A}(z)$ out to redshift $z$ relates the 
physical size $\d R=a(t)\,f_K(\chi)\,\d\theta$ of a comoving object to
its angular size $\d\theta$ on the sky, $D_{\rm A}=\d R/\d\theta$. It
is thus related to the comoving angular diameter $f_K$ though $D_{\rm
A}(z)=a\, f_K(\chi)=(1+z)^{-1}\,f_K(\chi)$. 
\\
\\
The angular-diameter distance of a source at redshift $z_2$ seen by an
observer along the same line-of-sight at redshift $z_1<z_2$ is given
as 
%
\be
D_{\rm A}(z_1,z_2)={1\over 1+z_2}\,f_k(\chi_2-\chi_1) \;,
\label{eq:FRW10}
\ee
%
where $\chi_i$ is the comoving distance out to redshift $z_i$.
\\
\\
The luminosity distance\index{Luminosity distance}
$D_{\rm L}(z)$ yields the relation between the bolometric luminosity
$L$ of an isotropically radiating comoving source at redshift $z$ to
the observed bolometric flux $S$, 
%
\be
D_{\rm L}(z)\equiv \sqrt{L\over 4\pi S}\;.
\label{eq:FRW11}
\ee
%
For any metric theory of gravity, $D_{\rm L}(z)=(1+z)^2 D_{\rm
  A}(z)$, sometimes called the `duality relation' of distances.
The comoving distance $\chi$ of a source as a function of
its cosmological redshift is obtained from (\ref{eq:FRW3}) and
(\ref{eq:FRW4}),\index{Comoving distance!asfunction of redshift}
%
\be
\chi=\int_{(1+z)^{-1}}^1{c\;\d a\over H(a)\,a^2}\;.
\label{eq:FRW12}
\ee
%
The proper and comoving volume elements corresponding to a
solid angle $\d\omega$ and a redshift interval $\d z$ at redshift $z$
are 
%
\be
\d V_{\rm prop}=D_{\rm A}^2(z)\,\d\omega\,{\d D_{\rm prop}\over \d
  z}\,\d z \; ;\quad
\d V_{\rm com}=a^{-3}\,\d V_{\rm prop} =
f_K^2(\chi(z))\,\d\omega\,{\d\chi\over\d z}\,\d z \;,
\label{eq:FRW13}
\ee
%
where $\d D_{\rm prop}=c\,\d t=a(z)\,\d\chi$. Finite volumes are
obtained from these expression by integration.
\\
\\
Since the speed of light is finite, light can only propagate a finite
distance from the Big Bang until a given epoch. This distance is
called {\it horizon}. The comoving horizon size at redshift $z$ is
given as 
%
\be
r_{\rm H,com}(z)=\int_0^{(1+z)^{-1}}{c\;\d a\over a^2\,H(a)}\;.
\label{eq:FRW14}
\ee
}

%%%%%%%%%%%%%%%%
\dictentry{review}{Structure Formation in the Universe}{\sfm}
\newcommand\sfm{
The formation
 of structure in the Universe is a complex 
research field that cannot be described in the framework of this
document. Here, we provide the basic quantities for the simplest case,
namely the growth of density perturbations of pressureless matter on
scales smaller than the horizon scale.
\\
\\
In the framework of an inhomogeneous universe, the mean density of
matter as function of epoch is denoted by $\bar{\rho}_{\rm m}(t)$ (or with
scale factor $a$, redshift $z$, or comoving distance $\chi$ as
argument). The \index{Fractional matter density contrast $\delta_{\rm
    m}$}fractional matter density contrast is defined as  
\setcounter{equation}{0}
\renewcommand{\theequation}{LSS.\arabic{equation}}
%
\be
\delta_{\rm m}(\vec x,t):={\rho_{\rm m}(\vec x,t)-\bar{\rho}_{\rm m}(t)\over
\bar{\rho}_{\rm m}(t)}\;,
\label{eq:LSS1}
\ee
%
where $\vec x$ is the comoving spatial position, and the
\index{Peculiar velocity}peculiar
velocity is denoted by $\vec v(\vec x,t)$. Describing the behaviour of
the matter field using the fluid approximation, and neglecting
pressure (which is a valid approximation for describing (cold) dark
matter in the epochs of the Universe
for which $a\gg a_{\rm eq}=\Omega_{\rm r}/\Omega_{\rm m}\ll 1$, as
long as the density contrast $\delta_{\rm m}\lesssim 1$), one obtains
the coupled set of continuity equation, pressure-less Euler equation,
and Poisson equation, 
%
\be
{\partial \delta_{\rm m} \over \partial t}+{1\over
  a}\nabla\cdot\eck{(1+\delta_{\rm m})\vec v}=0 \; ; \quad
{\partial \vec v\over \partial t}+{\dot a\over a}\vec v+{1\over a}\rund{\vec
v\cdot \nabla} \vec v=-{1\over a}\nabla \Phi_{\rm N} \; ; \quad
\nabla^2\Phi_{\rm N} ={3 H_0^2 \Omega_{\rm m} \over 2 a}\delta_{\rm m} \;,
\label{eq:LSS2}
\ee
%
where the spatial derivatives are with respect to comoving coordinates
$\vec x$, and $\dot a\equiv \d a/\d t$. Assuming the fraction density
contrast to be small, $|\delta_{\rm m}|\ll 1$, this set of equations
can be linearized in $\delta_{\rm m}$ and $|\vec v|/c$, and then
combined into a single equation,
%
\be
\frac{\partial^2{\delta_{\rm m}}}{\partial t^2}
+{2\dot a\over a}{\partial \delta_{\rm m}\over \partial t}-{3 H_0^2 \Omega_{\rm
m} \over 2 a^3 } \delta_{\rm m} = 0\;.
\label{eq:LSS3}
\ee
%
Since the coefficients of this partial differential equation have no
dependence on $\vec x$, its solution can then be factorized,
$\delta_{\rm m}(\vec x,t)=D_+(t)\,\Delta_+(\vec x)+D_-(t)\,\Delta_-(\vec x)$,
where the $\Delta_\pm(\vec x)$ are arbitrary functions of $\vec x$, to be
determined from the initial conditions, and 
$D_\pm(t)$ satisfy the ordinary differential equation
%
\be
\ddot{D} + \frac{2 \dot{a}}{a} \dot{D} = \frac{3 H_0^2 \Omega_{\rm
m}}{2 a^3}\, D \;.
\label{eq:LSS4}
\ee
%
$D_-$ is a decreasing function of time, thus these modes decay during
cosmic evolution. The solution $D_+$, called the \index{Growth factor
  $D_+$}{\it growth factor}, 
inceases in time and hence 
describes the growth of density fluctuations. It is normalized to
unity today, $D_+(t_0)=1$. If the dark energy is described by a
cosmological constant, then
%
\be
D_+(t)\propto{H(t)  H_0^2}\int^t_0 {\d t'\over a^2(t') H^2(t')} \;,\quad
\hbox{or}\quad
D_+(a)\propto {H(a)\over H_0}\int_0^a {\d a' \over \eck{\Omega_{\rm m}/a'
+\Omega_\Lambda a'^2-(\Omega_{\rm m}+\Omega_\Lambda-1)}^{3/2}}\;.
\label{eq:LSS5}
\ee
%
The solution $\delta_{\rm m}(\vec x,t)=D_+(t)\delta_{\rm m,0}(\vec
x)$, where $\delta_{\rm m,0}(\vec x)$ is the {\it linearly
  extrapolated density contrast today}, with the function $D_+(t)$
being the growing solution of Eq.\,(\ref{eq:LSS4}), is valid only for
subhorizon perturbations in
the matter-dominated epoch ($a\gg a_{\rm eq}$). Superhorizon
perturbations grow $\propto a$ for $a\gg a_{\rm eq}$, which is the same
behavior as $D_+$ for epochs where matter dominates over the dark
energy and the curvature term in the Friedman eqation (\ref{eq:FRW9}).
In the radiation-dominated epoch ($a\ll a_{\rm eq}$), superhorizon
fluctuations grow $\propto a^2$, whereas subhorizon matter
fluctuations are stalled: their amplitude stays constant. In fact, for
sub-horizon fluctuations one finds $\delta_{\rm m}\propto
\rund{1+a_{\rm eq}^{-1}a}$. This different growth behavior of super-
and subhorizon scale fluctuations in the radiation-dominated epoch
suppresses density fluctuations on scales smaller than  the comoving
horizon scale at matter-radiation 
equality, $r_{\rm H,com}(a_{\rm eq})$ (see Eq.\,\ref{eq:FRW14}), called
\index{M\'esz\'aros effect}M\'esz\'aros effect. Furthermore, the
growth of structure is affected by the presence of baryons in the epoch
before recombination ($z\gtrsim 1000$), since the ionized baryon fluid
is strongly coupled to the photons and together form a
pressure-dominated fluid in which \index{Baryonic acoustic
osciallations}{\it baryonic acoustic osciallations} develop. In
addition, the presence of neutinos (or other forms of \index{Hot dark
matter}hot dark matter) can suppress structure growth due to
free-streaming of fast particles.
\\
\\
If $\tilde\delta_{\rm m}(\vec k,t)$ denotes  the Fourier transform of
$\delta_{\rm m}(\vec x,t)$, where $\vec k$ is the comoving wave
vector, then
%
\be
\ave{\tilde\delta_{\rm m}(\vec k,t)\,\tilde\delta_{\rm m}(\vec k',t)}
=(2\pi)^3\,\delta_{\rm D}(\vec k + \vec k')\,P_{\rm m}(|\vec k|,t)\;,
\label{eq:LSS6}
\ee
%
where the Dirac delta `function' is due to the assumed statistical
homogeneity of the density field, and the \index{Matter power
spectrum}{\it matter power spectrum} $P_{\rm m}$ depends only on the
absolute value of $\vec k$, owing to the assumed statistical isotropy.
\\
\\
The primordial power spectrum as predicted from simple inflationary
models follows a power law, $P_{\rm m}\propto k^{n_{\rm s}}$, where
the \index{Spectral index of primordial fluctuations $n_{\rm
s}$}spectral index of primordial fluctuations $n_{\rm s}$ is predicted
to be slightly less than unity.  In the framework of linear
perturbation theory, the matter power spectrum then becomes
%
\be
P(k,t)=A\,k^{n_{\rm s}}\,T_k^2\,D_+^2(t)\;,
\label{eq:LSS7}
\ee
%
valid for the matter-dominated epoch. The \index{Transfer function
$T_k$}{\it transfer function $T_k$} accounts for the M\'esz\'aros
effect, the presence of baryonic acoustic oscillations and the effects
of hot dark matter and can be obtained from solving the linearized
coupled Boltzmann equations for the various components (cold dark
matter, baryons, photons, neutrinos) in the perturbed
Friedman--Robertson--Walker metric (\ref{eq:LE22}); there are publicly
available codes for obtaining accurate transfer functions (see e.g. \citep{1999ApJ...511....5E}).}


%----------------------------------------------------------------------------------------
% Part I --- Dictionary, short definitions
%----------------------------------------------------------------------------------------
\newpage
\printglossary[style=list,type=dictionary,title= \Large\textbf{Short Definitions}]%

%----------------------------------------------------------------------------------------
% Part II --- Review, long definitions
%----------------------------------------------------------------------------------------
\newpage
\printglossary[style=list,type=review,title= \Large\textbf{Long Definitions}]%

%----------------------------------------------------------------------------------------
% Bibliography
%----------------------------------------------------------------------------------------
\newpage

\bibliographystyle{apalike}
\bibliography{biblio}

%\printindex
\end{document}
